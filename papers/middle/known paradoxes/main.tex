\documentclass[11pt]{article}

\usepackage[T1]{fontenc}
\usepackage[utf8]{inputenc}
\usepackage{lmodern}
\usepackage{microtype}
\usepackage{amsmath,amssymb,amsthm}
\usepackage{graphicx}
\usepackage{hyperref}
\usepackage{url}
\usepackage{booktabs}
\usepackage{enumitem}

\usepackage{../../../templates/cop/cop-macros}

\title{Known Paradoxes: Non-Canonical Recognition Anchors}
\author{Pascal Sparidaens}
\date{}

\begin{document}
\maketitle

\begin{abstract}
This document lists non-canonical recognition anchors
that readers may associate with the COP framework.
These anchors are not sources of COP or ICOP structure
and do not constrain admissibility or legibility.
They exist solely to aid orientation
and must not be used for derivation, explanation, or interpretation.
\end{abstract}

\section{Status and Role}

This document is \textbf{not} canonical.
It is not structural.
It is a recognition layer only.

Known paradox anchors:
\begin{itemize}
  \item do not define admissibility or legibility,
  \item do not precede or generate COP or ICOP structure,
  \item do not override any COP corpus section,
  \item and must not be treated as equivalent to COP or ICOP entries.
\end{itemize}

Similarity indicates reader recognition only.
It implies no conceptual, historical, or logical relation.

\section{Recognition Anchors by Shadow Axis}

The anchors below are arranged to mirror the Shadow Axes
for navigational convenience only.
COP and ICOP references are provided strictly as index locators.
No derivation or explanation is licensed.

\subsection*{A1 (S1) — Loss Required for Continuation}
\textbf{Related Axis:} S1 (COP-01 \texttt{<->} ICOP-01)

Common reader recognition:
systems often require loss, discard, or pruning
to avoid cascading failure.

\subsection*{A2 (S2) — Contradictions That Persist}
\textbf{Related Axis:} S2 (COP-02 \texttt{<->} ICOP-02)

Common reader recognition:
incompatibilities may persist without resolution;
effects do not justify causal reconstruction.

\subsection*{A3 (S3) — Coherence Without Signaling; Maintenance Without Origin}
\textbf{Related Axis:} S3 (COP-03 \texttt{<->} ICOP-03)

Common reader recognition:
systems remain aligned without communication;
repeated upkeep does not reveal beginnings.

\subsection*{A4 (S4) — Operation Within Bands; Responsibility Diffuses}
\textbf{Related Axis:} S4 (COP-04 \texttt{<->} ICOP-04)

Common reader recognition:
stability operates within bounded ranges;
systemic impact cannot be assigned to single actors.

\subsection*{A5 (S5) — One Frame, Many Functions; Agency Is Situational}
\textbf{Related Axis:} S5 (COP-05 \texttt{<->} ICOP-05)

Common reader recognition:
unity does not imply sameness;
agency appears only under enabling conditions.

\subsection*{A6 (S6) — Binding Without Agreement; Not All Actions Count}
\textbf{Related Axis:} S6 (COP-06 \texttt{<->} ICOP-06)

Common reader recognition:
commitment can stabilize without consensus;
motion without consequence is null.

\subsection*{A7 (S7) — Coordination Without Fusion; Limits Preserve Health}
\textbf{Related Axis:} S7 (COP-07 \texttt{<->} ICOP-07)

Common reader recognition:
integration beyond limits produces brittleness;
constraints preserve viability.

\subsection*{A8 (S8) — Invisibility of What Works}
\textbf{Related Axis:} S8 (COP-08 \texttt{<->} ICOP-08)

Common reader recognition:
what functions correctly disappears from attention;
background conditions must not enter the foreground.

\section{Non-Equivalence Rule}

No anchor listed here:
\begin{itemize}
  \item is a COP or ICOP entry,
  \item substitutes for a COP or ICOP,
  \item explains COP or ICOP structure or meaning,
  \item or licenses interpretation or intervention.
\end{itemize}

\section{Use}

Permitted use:
\begin{itemize}
  \item reader orientation,
  \item teaching entry points,
  \item translation for unfamiliar audiences.
\end{itemize}

Forbidden use:
\begin{itemize}
  \item derivation of COP or ICOP structure,
  \item explanation of COP meaning,
  \item justification of actions, policies, or interventions.
\end{itemize}

\end{document}
