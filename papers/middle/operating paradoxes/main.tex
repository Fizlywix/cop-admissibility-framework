\documentclass[11pt]{article}

\usepackage[T1]{fontenc}
\usepackage[utf8]{inputenc}
\usepackage{lmodern}
\usepackage{microtype}
\usepackage{amsmath,amssymb,amsthm}
\usepackage{graphicx}
\usepackage{hyperref}
\usepackage{url}
\usepackage{booktabs}
\usepackage{enumitem}

\usepackage{../../../templates/cop/cop-macros}

\title{Operating Paradoxes: Active Constraints in the COP Framework}
\author{Pascal Sparidaens}
\date{}

\begin{document}
\maketitle

\begin{abstract}
This document lists the Operating Paradoxes of the COP framework.
Operating Paradoxes are active admissibility constraints.
They introduce no new structure, notation, or claims.
Their sole purpose is to specify the complete operational paradox set
governing admissible operation and legibility.
\end{abstract}

\section{Status and Role}

This document is not explanatory.
It is canonical.

Operating Paradoxes:
\begin{itemize}
  \item are decision constraints, not states,
  \item operate as admissibility boundaries, not outcomes,
  \item are neither problems nor solutions,
  \item and do not form a narrative or temporal sequence.
\end{itemize}

They define where operation is admissible
and where legibility collapses.

\section{Operating Paradoxes}

The Operating Paradoxes consist of eight paired constraints.
Each paradox has:
\begin{itemize}
  \item a forward admissibility constraint (COP),
  \item and a reverse legibility constraint (ICOP).
\end{itemize}

The pairing is fixed and non-negotiable.
Shadow Axes bind each pair.

\subsection*{OP-01 — Survivability / Residue}
\textbf{COP-01 \texttt{<->} ICOP-01}

Permits loss to prevent cascade.
Constrains inference to irreversible residue.

\subsection*{OP-02 — Routed Tension / Effects}
\textbf{COP-02 \texttt{<->} ICOP-02}

Permits coexistence without resolution.
Forbids causal reconstruction from effects.

\subsection*{OP-03 — Coherence / Maintenance}
\textbf{COP-03 \texttt{<->} ICOP-03}

Permits coherence without communication.
Forbids recovery of history from maintenance.

\subsection*{OP-04 — Stability Band / Responsibility}
\textbf{COP-04 \texttt{<->} ICOP-04}

Permits operation only within bounded ranges.
Forbids local attribution of network effects.

\subsection*{OP-05 — Structural Unity / Agency}
\textbf{COP-05 \texttt{<->} ICOP-05}

Permits functional plurality within one structure.
Forbids treating agency as inherent or granted.

\subsection*{OP-06 — Commitment / Action}
\textbf{COP-06 \texttt{<->} ICOP-06}

Permits binding without consensus.
Constrains action to write-capable operations.

\subsection*{OP-07 — Coordinated Multiplicity / Limitation}
\textbf{COP-07 \texttt{<->} ICOP-07}

Permits unity without total integration.
Forbids removal of systemic limiters.

\subsection*{OP-08 — Non-Local Constraint / Unassimilable Return}
\textbf{COP-08 \texttt{<->} ICOP-08}

Forbids further articulation when admissibility is exhausted.
Constrains return to be non-local, unassimilable, and non-narrativizable.

\section{Global Notes}

\begin{itemize}
  \item Operating Paradoxes do not include COP-00 or ICOP-00.
  \item No new Operating Paradoxes may be added.
  \item Ordering is canonical and non-narrative.
  \item Interpretation and misuse prevention are governed by Shadow Axes.
\end{itemize}

\section{Placement in the COP Corpus}

Operating Paradoxes form the active constraint core.

They are positioned:
\begin{itemize}
  \item below Caps (global gates),
  \item above Light Axes (usable forward surfaces),
  \item and bound by Shadow Axes (admissibility and legibility limits).
\end{itemize}

They define admissibility.
They do not explain it.

\end{document}
