\documentclass[11pt]{article}

\usepackage[T1]{fontenc}
\usepackage[utf8]{inputenc}
\usepackage{lmodern}
\usepackage{microtype}
\usepackage{amsmath,amssymb,amsthm}
\usepackage{graphicx}
\usepackage{hyperref}
\usepackage{url}
\usepackage{booktabs}
\usepackage{enumitem}

\usepackage{../../../templates/cop/cop-macros}

\title{Operating Paradoxes: Active Constraints in the COP Framework}
\author{Pascal Sparidaens}
\date{}

\begin{document}
\maketitle

\begin{abstract}
This document lists the Operating Paradoxes of the COP framework.
Operating Paradoxes are active admissibility constraints.
They introduce no new structure, notation, or claims.
Their sole purpose is to specify the canonical paradox set
that governs admissible operation.
\end{abstract}

\section{Status and Role}

This document is not explanatory.
It is canonical.

Operating Paradoxes:
\begin{itemize}
  \item are decision constraints, not states,
  \item operate as choice boundaries, not outcomes,
  \item are neither problems nor solutions,
  \item and do not form a narrative sequence.
\end{itemize}

They define where admissible operation is possible.

\section{Operating Paradoxes}

The following paradoxes constitute the active COP set.
Each paradox is listed by identifier and canonical label.
Forward (F), Reverse (R), or Reverse-Inverse (RI) status
is specified for orientation only.

\subsection*{COP-01 — Survivability \hfill (F)}
Permits loss to prevent cascade.

\subsection*{COP-02 — Residue Primacy \hfill (R)}
Constrains inference to irreversible residue.

\subsection*{COP-03 — Routed Tension \hfill (F)}
Permits coexistence without resolution.

\subsection*{COP-04 — Coherence Without Communication \hfill (F)}
Permits alignment without signaling.

\subsection*{COP-05 — Stability Band \hfill (F)}
Permits operation only within bounded ranges.

\subsection*{COP-06 — Effects-First Ontology \hfill (R)}
Forbids causal reconstruction from outcomes.

\subsection*{COP-07 — Structural Unity, Functional Plurality \hfill (F)}
Permits multiple functions within one structure.

\subsection*{COP-08 — Commitment Replaces Consensus \hfill (F)}
Permits binding without agreement.

\subsection*{COP-09 — Agency Is Produced, Not Granted \hfill (F)}
Permits agency only under write conditions.

\subsection*{COP-10 — Not All Actions Exist \hfill (R)}
Constrains action to write-capable operations.

\subsection*{COP-11 — Maintenance Is Cyclic; History Is Directional \hfill (R)}
Forbids recovery of origin from upkeep.

\subsection*{COP-12 — Unity Is Coordinated Multiplicity \hfill (F)}
Permits unity without total integration.

\subsection*{COP-13 — Responsibility Must Be Network-Routed \hfill (R)}
Forbids local attribution of network effects.

\subsection*{COP-14 — Healthy Systems Require Limiters \hfill (F)}
Permits viability only with constraints.

\section{Global Notes}

\begin{itemize}
  \item Operating Paradoxes do not include COP-00 or COP-15.
  \item No new Operating Paradoxes may be added.
  \item Ordering is canonical and may not be altered.
  \item Interpretation is governed by Shadow Axes.
\end{itemize}

\section{Placement in the COP Corpus}

Operating Paradoxes form the core constraint layer.

They are positioned:
\begin{itemize}
  \item below Caps (COP-00, COP-15),
  \item above Light Axes,
  \item and bound by Shadow Axes.
\end{itemize}

They define admissibility.
They do not explain it.

\end{document}
