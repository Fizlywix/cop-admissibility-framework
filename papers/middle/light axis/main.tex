\documentclass[11pt]{article}

\usepackage[T1]{fontenc}
\usepackage[utf8]{inputenc}
\usepackage{lmodern}
\usepackage{microtype}
\usepackage{amsmath,amssymb,amsthm}
\usepackage{graphicx}
\usepackage{hyperref}
\usepackage{url}
\usepackage{booktabs}
\usepackage{enumitem}

\usepackage{../../../templates/cop/cop-macros}

\title{Light Axes: Usable Surfaces in the COP Framework}
\author{Pascal Sparidaens}
\date{}

\begin{document}
\maketitle

\begin{abstract}
This document specifies the Light Axes of the COP framework.
Light Axes define the usable, forward-facing surfaces of the corpus.
They introduce no new structure, notation, or claims.
Their sole purpose is to constrain legitimate operation
without permitting interpretation or reconstruction.
\end{abstract}

\section{Status and Role}

This document is not a paper.
It is an interface index.

The Light Axes:
\begin{itemize}
  \item specify how the framework may be used forward,
  \item do not justify or explain the framework,
  \item do not override Shadow or Cap constraints,
  \item and do not introduce new admissibility rules.
\end{itemize}

Light Axes are operational.
They are not authoritative.

\section{Light Axes}

Each Light Axis corresponds to a Light paper (L1--L8).
Each axis defines a single usable surface.
No axis subsumes another.

\subsection*{L1 — Reading Discipline}

L1 specifies how the COP corpus may be read.

It constrains:
\begin{itemize}
  \item reading order,
  \item illegitimate inference,
  \item narrative reconstruction.
\end{itemize}

L1 introduces no structure.
It only blocks misuse.

\subsection*{L2 — Coordinate System}

L2 defines the admissibility coordinate system.

It introduces:
\begin{itemize}
  \item axes,
  \item roles,
  \item admissibility constraints.
\end{itemize}

L2 contains no guidance on interpretation or application.

\subsection*{L3 — Reading Without Reconstruction}

L3 forbids reconstruction of history, intent, or origin.

It constrains:
\begin{itemize}
  \item narrative inference,
  \item causal backfilling,
  \item interpretive synthesis.
\end{itemize}

Any use inconsistent with L3 is invalid.

\subsection*{L4 — Diagrammatic Addressing}

L4 specifies the grammar of diagrams.

It constrains:
\begin{itemize}
  \item what diagrams may do,
  \item how they may be read,
  \item how they may not explain.
\end{itemize}

Figures function as constraint objects only.

\subsection*{L5 — Operational Admissibility}

L5 demonstrates admissibility under constraint.

It provides:
\begin{itemize}
  \item an operational surface,
  \item a bounded example,
  \item no interpretation.
\end{itemize}

The example is a host surface, not a model.

\subsection*{L6 — Survivability Table}

L6 clarifies the role of the periodic table analogy.

It constrains:
\begin{itemize}
  \item misreadings of enumeration,
  \item semantic interpretation,
  \item ontological projection.
\end{itemize}

The table records survivability, not elements.

\subsection*{L7 — Visibility Under Stress}

L7 specifies why failure becomes visible.

It constrains:
\begin{itemize}
  \item observer-centric explanations,
  \item narrative salience,
  \item success-as-feature assumptions.
\end{itemize}

Visibility is a consequence of constraint pressure.

\subsection*{L8 — Applied Constraint Surfaces}

L8 specifies legitimate use without explanation.

It constrains:
\begin{itemize}
  \item application scope,
  \item stopping conditions,
  \item misuse as theory.
\end{itemize}

No examples are given by design.

\section{Global Reading Discipline}

The Light Axes are subject to the following rules:

\begin{itemize}
  \item Light Axes do not override Shadow Axes.
  \item Light Axes do not override Caps.
  \item Light Axes may be used only forward.
  \item Light Axes do not justify interpretation.
\end{itemize}

Any conflict is resolved against the Light Axes.

\section{Placement in the COP Corpus}

The COP corpus consists of eight structural sections:

\begin{enumerate}
  \item Full COP List
  \item Caps
  \item Operating Paradoxes
  \item Light Axes
  \item Shadow Axes
  \item Known Paradoxes
  \item Forbidden Moves
  \item Extension / Non-Extension Rules
\end{enumerate}

The Light Axes define the usable surface.
They do not define meaning.

\end{document}
