\documentclass[11pt]{article}

\usepackage[T1]{fontenc}
\usepackage[utf8]{inputenc}
\usepackage{lmodern}
\usepackage{microtype}
\usepackage{amsmath,amssymb,amsthm}
\usepackage{graphicx}
\usepackage{hyperref}
\usepackage{url}
\usepackage{booktabs}
\usepackage{enumitem}

\usepackage{../../../templates/cop/cop-macros}

\title{The Canonical Operating Paradoxes (Full COP / ICOP List)}
\author{Pascal Sparidaens}
\date{}

\begin{document}
\maketitle

\begin{abstract}
This document enumerates the full Canonical Operating Paradoxes
of the COP framework.
It is a canonical index.
It introduces no explanations, examples, or interpretations.
All entries listed here are admissibility constraints,
not states, steps, or claims.
\end{abstract}

\section{Definition}

A Canonical Operating Paradox is a structural choice boundary.
It constrains admissibility (forward) or legibility (reverse)
under operational limits.

Operating Paradoxes:
\begin{itemize}
  \item are not states,
  \item are not steps,
  \item are not explanations,
  \item are not narratives.
\end{itemize}

\section{Directional Roles}

Each Operating Paradox has exactly one role:

\begin{itemize}
  \item \textbf{COP} — Forward admissibility (operable)
  \item \textbf{ICOP} — Reverse legibility (diagnostic only)
\end{itemize}

No paradox performs both roles.

\section{Caps (Outside the Operating Cycle)}

Caps are absolute structural gates.
They are not Operating Paradoxes.

\begin{itemize}
  \item \textbf{COP-00} — Null Dominance (Entry Gate)
  \item \textbf{ICOP-00} — Return Gate
\end{itemize}

Caps are not traversable.
They are not paired.

\section{Operating Paradoxes}

The COP framework contains exactly eight Operating Paradoxes.
Each paradox consists of one COP and one ICOP,
bound by a Shadow Axis.

\subsection*{OP-01}
\begin{itemize}
  \item \textbf{COP-01} — Manifestation
  \item \textbf{ICOP-01} — Almost-Nothing (Final Residue)
\end{itemize}

\subsection*{OP-02}
\begin{itemize}
  \item \textbf{COP-02} — 1 $\rightarrow$ MANY
  \item \textbf{ICOP-02} — MANY $\rightarrow$ 1
\end{itemize}

\subsection*{OP-03}
\begin{itemize}
  \item \textbf{COP-03} — Boundary (Inside/Outside)
  \item \textbf{ICOP-03} — Deboundarying (Field Homogenization)
\end{itemize}

\subsection*{OP-04}
\begin{itemize}
  \item \textbf{COP-04} — Relation (Selective Interaction)
  \item \textbf{ICOP-04} — Decoupling (Network Collapse)
\end{itemize}

\subsection*{OP-05}
\begin{itemize}
  \item \textbf{COP-05} — Dynamics (Change Admitted)
  \item \textbf{ICOP-05} — Shutdown (Energy Leak / Termination)
\end{itemize}

\subsection*{OP-06}
\begin{itemize}
  \item \textbf{COP-06} — Selection (Pruning Required)
  \item \textbf{ICOP-06} — Deselection (History Pruned)
\end{itemize}

\subsection*{OP-07}
\begin{itemize}
  \item \textbf{COP-07} — Saturation (Knife Edge)
  \item \textbf{ICOP-07} — Destructuring (Bindings Break)
\end{itemize}

\subsection*{OP-08}
\begin{itemize}
  \item \textbf{COP-08} — Non-Local Constraint
  \item \textbf{ICOP-08} — Unassimilable Return
\end{itemize}

\section{Shadow Binding (Non-Optional)}

Each Operating Paradox is bound by exactly one Shadow Axis.
This pairing is structural and mandatory.

\begin{itemize}
  \item S1: COP-01 \texttt{<->} ICOP-01
  \item S2: COP-02 \texttt{<->} ICOP-02
  \item S3: COP-03 \texttt{<->} ICOP-03
  \item S4: COP-04 \texttt{<->} ICOP-04
  \item S5: COP-05 \texttt{<->} ICOP-05
  \item S6: COP-06 \texttt{<->} ICOP-06
  \item S7: COP-07 \texttt{<->} ICOP-07
  \item S8: COP-08 \texttt{<->} ICOP-08
\end{itemize}

\section{Ordering Rules}

\subsection*{Forward Presentation Order (Surface)}

This order may be used for external presentation.
It is not a narrative and does not imply traversal.

\begin{enumerate}
  \item COP-00
  \item COP-01
  \item COP-02
  \item COP-03
  \item COP-04
  \item COP-05
  \item COP-06
  \item COP-07
  \item COP-08
\end{enumerate}

\subsection*{Parallel Reverse Index (Mandatory, Not Sequential)}

ICOPs are not optional.
They are co-present with COPs via Shadow Axes
and constrain interpretation in parallel.

This list is an index only.
It is not a second phase and must not be read as a sequence.

\begin{enumerate}
  \item ICOP-01
  \item ICOP-02
  \item ICOP-03
  \item ICOP-04
  \item ICOP-05
  \item ICOP-06
  \item ICOP-07
  \item ICOP-08
\end{enumerate}

\subsection*{Caps Placement}

Caps bound the index and may be shown as endpoints of presentation only:

\begin{itemize}
  \item Entry Gate: COP-00
  \item Return Gate: ICOP-00
\end{itemize}

\section{Hard Reading Rules}

\begin{itemize}
  \item Never read Operating Paradoxes as a narrative.
  \item Never resolve a paradox.
  \item Never treat ICOPs as design rules.
  \item Never foreground Caps.
  \item Never list ICOPs as a post-OP sequence.
\end{itemize}

\section{Lock Statement}

Operating Paradoxes do not describe systems.
They constrain admissibility and legibility.

What cannot exist
determines everything that can.

\end{document}
