\documentclass[11pt]{article}

\usepackage[T1]{fontenc}
\usepackage[utf8]{inputenc}
\usepackage{lmodern}
\usepackage{microtype}
\usepackage{amsmath,amssymb,amsthm}
\usepackage{graphicx}
\usepackage{hyperref}
\usepackage{url}
\usepackage{booktabs}
\usepackage{enumitem}

\usepackage{../../../templates/cop/cop-macros}

\title{The Canonical Operating Paradoxes (Full COP List)}
\author{Pascal Sparidaens}
\date{}

\begin{document}
\maketitle

\begin{abstract}
This document enumerates the full Canonical Operating Paradoxes (COP).
It is a canonical index.
It introduces no explanations, examples, or interpretations.
All COP entries are admissibility constraints, not states, steps, or claims.
\end{abstract}

\section{Definition}

A Canonical Operating Paradox (COP) is a structural choice boundary.
It constrains what may exist or persist under admissibility limits.

COPs:
\begin{itemize}
  \item are not states,
  \item are not steps,
  \item are not explanations,
  \item are not narratives.
\end{itemize}

\section{Directional Flags}

Each COP carries exactly one directional status:

\begin{itemize}
  \item \textbf{F} — Forward admissibility (operable)
  \item \textbf{R} — Reverse legibility (inferential only)
  \item \textbf{RI} — Reverse-Inverse boundary (non-traversable)
\end{itemize}

\section{Caps (Outside the Operating Cycle)}

Caps bound the framework before and after use.
They are not operating paradoxes.

\begin{itemize}
  \item \textbf{COP-00} — Null Dominance \hfill (RI)
  \item \textbf{COP-15} — Background Operation \hfill (RI)
\end{itemize}

\section{Operating Paradoxes}

\subsection*{Forward Admissibility (F)}

These COPs may be used in forward operation.
They open admissible option space.

\begin{itemize}
  \item \textbf{COP-01} — Survivability \hfill (F)
  \item \textbf{COP-03} — Routed Tension \hfill (F)
  \item \textbf{COP-04} — Coherence Without Communication \hfill (F)
  \item \textbf{COP-05} — Stability Band \hfill (F)
  \item \textbf{COP-07} — Structural Unity, Functional Plurality \hfill (F)
  \item \textbf{COP-08} — Commitment Replaces Consensus \hfill (F)
  \item \textbf{COP-09} — Agency Is Produced, Not Granted \hfill (F)
  \item \textbf{COP-12} — Unity Is Coordinated Multiplicity \hfill (F)
  \item \textbf{COP-14} — Healthy Systems Require Limiters \hfill (F)
\end{itemize}

\subsection*{Reverse Legibility (R)}

These COPs are never used forward.
They constrain inference after the fact.

\begin{itemize}
  \item \textbf{COP-02} — Residue Primacy \hfill (R)
  \item \textbf{COP-06} — Effects-First Ontology \hfill (R)
  \item \textbf{COP-10} — Not All Actions Exist \hfill (R)
  \item \textbf{COP-11} — Maintenance Is Cyclic; History Is Directional \hfill (R)
  \item \textbf{COP-13} — Responsibility Must Be Network-Routed \hfill (R)
\end{itemize}

\section{Shadow Pairing (Non-Optional)}

Each COP participates in exactly one Shadow Axis.
Caps pair only with each other.

\begin{itemize}
  \item S1: COP-01 ↔ COP-02
  \item S2: COP-03 ↔ COP-06
  \item S3: COP-04 ↔ COP-11
  \item S4: COP-05 ↔ COP-13
  \item S5: COP-07 ↔ COP-09
  \item S6: COP-08 ↔ COP-10
  \item S7: COP-12 ↔ COP-14
  \item S8: COP-00 ↔ COP-15
\end{itemize}

\section{Ordering Rules}

\subsection*{Canonical Internal Order}

This order governs internal reasoning and validation.

\begin{itemize}
  \item Caps: COP-00, COP-15
  \item Forward chain: COP-01 → 03 → 04 → 05 → 07 → 08 → 09 → 12 → 14
  \item Reverse constraints: COP-02, COP-06, COP-10, COP-11, COP-13
\end{itemize}

\subsection*{Public Presentation Order}

This order may be used externally.
Reverse logic remains implicit.

\begin{enumerate}
  \item COP-00
  \item COP-01
  \item COP-03
  \item COP-04
  \item COP-05
  \item COP-07
  \item COP-08
  \item COP-09
  \item COP-12
  \item COP-14
  \item COP-15
\end{enumerate}

\section{Hard Reading Rules}

\begin{itemize}
  \item Never read COPs as a narrative.
  \item Never resolve a paradox.
  \item Never use Reverse COPs as design rules.
  \item Never foreground Caps.
\end{itemize}

\section{Lock Statement}

COPs do not describe systems.
They constrain what systems may do.

What cannot exist
determines everything that can.

\end{document}
