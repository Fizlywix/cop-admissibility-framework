\documentclass[11pt]{article}

\usepackage[T1]{fontenc}
\usepackage[utf8]{inputenc}
\usepackage{lmodern}
\usepackage{microtype}
\usepackage{amsmath,amssymb,amsthm}
\usepackage{graphicx}
\usepackage{hyperref}
\usepackage{url}
\usepackage{booktabs}
\usepackage{enumitem}

\usepackage{../../../templates/cop/cop-macros}

\title{Forbidden Moves: Illegitimate Inference in the COP Framework}
\author{Pascal Sparidaens}
\date{}

\begin{document}
\maketitle

\begin{abstract}
This document enumerates explicitly forbidden inferential moves
within the COP framework.
It introduces no new structure, notation, or claims.
Its sole purpose is to prevent predictable misuse,
overextension, and illegitimate interpretation.
\end{abstract}

\section{Status}

This document is not explanatory.
It is prohibitive.

Each item listed here is invalid by construction.
References indicate where the move is structurally blocked
within the COP corpus.

\section{Forbidden Moves Index}

\subsection*{Reconstruction Fallacy}
Attempting to reconstruct events, histories, or origins
from residue, maintenance, or aftermath.

\textit{Blocked by:} S1, S3

---

\subsection*{Narrative Closure}
Treating the framework as a story with a beginning,
progression, and conclusion.

\textit{Blocked by:} L3, S8

---

\subsection*{Totalization Error}
Treating the COP framework as complete, exhaustive,
or capable of explaining everything.

\textit{Blocked by:} S8, Caps

---

\subsection*{Causal Inversion}
Inferring causes from observed effects
or treating outcomes as explanatory origins.

\textit{Blocked by:} S2

---

\subsection*{Resolution Demand}
Treating structural tension or contradiction
as a problem that must be solved or eliminated.

\textit{Blocked by:} S2

---

\subsection*{Consensus Substitution}
Treating agreement, participation, or discussion
as equivalent to binding commitment.

\textit{Blocked by:} S6

---

\subsection*{Intentional Inflation}
Treating intention, expression, or declaration
as equivalent to action or effect.

\textit{Blocked by:} S6

---

\subsection*{Agency Naturalization}
Treating agency as an inherent property
rather than a produced operational condition.

\textit{Blocked by:} S5

---

\subsection*{Authority Projection}
Treating stability, coherence, or persistence
as evidence of control or legitimacy.

\textit{Blocked by:} S4

---

\subsection*{Responsibility Localization}
Assigning responsibility to individual nodes
for effects produced by networks.

\textit{Blocked by:} S4

---

\subsection*{Uniformity Inference}
Treating structural unity
as functional sameness.

\textit{Blocked by:} S5

---

\subsection*{Integration Absolutism}
Treating unity as requiring total integration
or elimination of remainder.

\textit{Blocked by:} S7

---

\subsection*{Limit Removal}
Treating constraints or limiters
as design flaws to be eliminated.

\textit{Blocked by:} S7

---

\subsection*{Visibility Privileging}
Treating what is visible, articulated,
or foregrounded as more important or real.

\textit{Blocked by:} S8

---

\subsection*{Background Foregrounding}
Treating infrastructure, background processes,
or enabling conditions as actors or objects.

\textit{Blocked by:} S8, Caps

---

\subsection*{Interpretive Overreach}
Assigning meaning, ontology, or explanation
where only admissibility constraints are specified.

\textit{Blocked by:} L3, S8

---

\section{Use}

This index is exhaustive in intent.

Any inferential move resembling those listed here
is illegitimate by default,
regardless of novelty or motivation.

\section{Placement in the COP Corpus}

The Forbidden Moves Index is a mandatory discipline layer.
It applies to:
\begin{itemize}
  \item the Full COP List,
  \item the Operating Paradoxes,
  \item the Light Axes,
  \item the Shadow Axes,
  \item and all extensions or applications.
\end{itemize}

Violations do not indicate misunderstanding.
They indicate inadmissible use.

\end{document}
