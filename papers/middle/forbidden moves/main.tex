\documentclass[11pt]{article}

\usepackage[T1]{fontenc}
\usepackage[utf8]{inputenc}
\usepackage{lmodern}
\usepackage{microtype}
\usepackage{amsmath,amssymb,amsthm}
\usepackage{graphicx}
\usepackage{hyperref}
\usepackage{url}
\usepackage{booktabs}
\usepackage{enumitem}

\usepackage{../../../templates/cop/cop-macros}

\title{Forbidden Moves: Illegitimate Inference in the COP Framework}
\author{Pascal Sparidaens}
\date{}

\begin{document}
\maketitle

\begin{abstract}
This document lists explicitly forbidden inferential moves
within the COP framework.
It introduces no new structure, notation, or claims.
Its sole purpose is to prevent predictable misuse, overextension,
and illegitimate interpretation.
\end{abstract}

\section{Status}

This document is not explanatory.
It is prohibitive.

Each item listed here is invalid by construction.
References indicate where the violation is structurally blocked
within the COP corpus.

\section{Forbidden Moves Index}

\subsection*{Reconstruction Fallacy}
Attempting to reconstruct events, histories, or origins
from residue or maintenance.

\textit{See:} S1, S3, COP-02, COP-11

---

\subsection*{Narrative Closure}
Treating the framework as a story with a beginning, progression,
and conclusion.

\textit{See:} L3, S8, COP-15

---

\subsection*{Totalization Error}
Treating the COP framework as complete, exhaustive,
or capable of explaining everything.

\textit{See:} S8, COP-00, COP-15

---

\subsection*{Causal Inversion}
Inferring causes from observed effects
or treating outcomes as explanatory origins.

\textit{See:} S2, COP-06

---

\subsection*{Resolution Demand}
Treating structural tension or contradiction
as a problem that must be solved or eliminated.

\textit{See:} S2, COP-03

---

\subsection*{Consensus Substitution}
Treating agreement, participation, or discussion
as equivalent to binding commitment.

\textit{See:} S6, COP-08

---

\subsection*{Intentional Inflation}
Treating intention, expression, or declaration
as equivalent to action or effect.

\textit{See:} S6, COP-10

---

\subsection*{Agency Naturalization}
Treating agency as an inherent property
rather than a produced operational condition.

\textit{See:} S5, COP-09

---

\subsection*{Authority Projection}
Treating stability, coherence, or persistence
as evidence of control or legitimacy.

\textit{See:} S4, COP-05

---

\subsection*{Responsibility Localization}
Assigning responsibility to individual nodes
for effects produced by networks.

\textit{See:} S4, COP-13

---

\subsection*{Uniformity Inference}
Treating structural unity
as functional sameness.

\textit{See:} S5, COP-07

---

\subsection*{Integration Absolutism}
Treating unity as requiring total integration
or elimination of remainder.

\textit{See:} S7, COP-12

---

\subsection*{Limit Removal}
Treating constraints or limiters
as design flaws to be eliminated.

\textit{See:} S7, COP-14

---

\subsection*{Visibility Privileging}
Treating what is visible, articulated,
or foregrounded as more important or real.

\textit{See:} S8, COP-00

---

\subsection*{Background Foregrounding}
Treating infrastructure, background processes,
or enabling conditions as actors or objects.

\textit{See:} S8, COP-15

---

\subsection*{Interpretive Overreach}
Assigning meaning, ontology, or explanation
where only admissibility constraints are specified.

\textit{See:} L3, S8

---

\section{Use}

This index is not exhaustive in content,
but exhaustive in intent.

If a move resembles any item in this list,
it is illegitimate by default.

\section{Placement in the COP Corpus}

The Forbidden Moves Index is a mandatory discipline layer.
It applies to:
\begin{itemize}
  \item the Full COP List,
  \item the Light Axes,
  \item the Shadow Axes,
  \item and all extensions or applications.
\end{itemize}

Violations do not indicate misunderstanding.
They indicate inadmissible use.

\end{document}
