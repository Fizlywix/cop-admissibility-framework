\documentclass[11pt]{article}

\usepackage[T1]{fontenc}
\usepackage[utf8]{inputenc}
\usepackage{lmodern}
\usepackage{microtype}
\usepackage{amsmath,amssymb,amsthm}
\usepackage{graphicx}
\usepackage{hyperref}
\usepackage{url}
\usepackage{booktabs}
\usepackage{enumitem}

\usepackage{../../../templates/cop/cop-macros}

\title{Shadow Axes: Admissibility Boundaries in the COP Framework}
\author{Pascal Sparidaens}
\date{}

\begin{document}
\maketitle

\begin{abstract}
This document specifies the Shadow Axes of the COP framework.
Shadow Axes do not extend the framework.
They bind forward admissibility to reverse legibility and prevent illegitimate inference.
This document functions as a mandatory middle layer in the COP corpus.
\end{abstract}

\section{Status and Role}

This document is not a paper.
It is an index layer.

The Shadow Axes:
\begin{itemize}
  \item do not introduce new concepts,
  \item do not explain COP entries,
  \item do not form a narrative or sequence,
  \item and do not permit resolution.
\end{itemize}

They are binding constraints on how COP entries may be read.

\section{Shadow Axes}

Each Shadow Axis pairs a forward admissibility constraint with a reverse legibility constraint.
The pairing is structural and non-negotiable.

\subsection*{S1 — Survivability / Residue}
\textbf{COP-01 ↔ COP-02}

Survivability permits loss to prevent cascade.
Residue constrains inference without reconstructing events.

\textit{Boundary:}
What survives cannot preserve everything.
What remains cannot be replayed.

\subsection*{S2 — Routed Tension / Effects}
\textbf{COP-03 ↔ COP-06}

Routed tension permits coexistence without resolution.
Effects do not license causal reconstruction.

\textit{Boundary:}
Tension must not be resolved.
Effects must not be explained.

\subsection*{S3 — Coherence / Maintenance}
\textbf{COP-04 ↔ COP-11}

Coherence may persist without communication.
Maintenance is cyclic; history is not recoverable.

\textit{Boundary:}
Coherence does not imply signaling.
Upkeep does not encode origin.

\subsection*{S4 — Stability Band / Responsibility}
\textbf{COP-05 ↔ COP-13}

Operation is viable only within bounded ranges.
Responsibility propagates through networks, not nodes.

\textit{Boundary:}
Stability is not control.
Responsibility is not local.

\subsection*{S5 — Structural Unity / Agency}
\textbf{COP-07 ↔ COP-09}

One structure may support multiple distinct functions.
Agency is produced by operational conditions.

\textit{Boundary:}
Unity does not imply sameness.
Agency does not precede structure.

\subsection*{S6 — Commitment / Action}
\textbf{COP-08 ↔ COP-10}

Commitment stabilizes without consensus.
Only write-capable operations count as actions.

\textit{Boundary:}
Agreement is not binding.
Intention is not action.

\subsection*{S7 — Coordinated Multiplicity / Limitation}
\textbf{COP-12 ↔ COP-14}

Unity arises from selective coordination.
Healthy systems require limiters.

\textit{Boundary:}
Coordination must not become total integration.
Limits must not be removed.

\subsection*{S8 — Null Dominance / Background Operation}
\textbf{COP-00 ↔ COP-15}

Correct operation leaves no trace.
Sustaining infrastructure must not appear as an actor.

\textit{Boundary:}
What enables operation must not dominate it.
What sustains structure must not enter it.

\section{Global Reading Discipline}

The Shadow Axes impose the following rules:

\begin{itemize}
  \item Forward use proceeds only along visible admissible operations (\texttt{+}).
  \item Reverse constraints are diagnostic only.
  \item No Shadow Axis licenses reconstruction, explanation, or closure.
  \item The configuration \texttt{---} is never locally traversed,
        but may be globally detectable as system non-response.
\end{itemize}

\section{Placement in the COP Corpus}

The COP corpus consists of three structural layers:

\begin{enumerate}
  \item \textbf{Caps:} COP-00 and COP-15
  \item \textbf{Operating Paradoxes:} COP-01 through COP-14
  \item \textbf{Shadow Axes:} S1 through S8
\end{enumerate}

The Shadow Axes are mandatory.
Without them, the COP list is vulnerable to illegitimate interpretation.

\end{document}
