% D1.tex — Paper D-1: Structural Equivalence Under Admissibility Constraints
% OSF / arXiv compatible
\documentclass[11pt]{article}

% ---------- Basics ----------
\usepackage[T1]{fontenc}
\usepackage[utf8]{inputenc}
\usepackage{lmodern}
\usepackage{microtype}
\usepackage{geometry}
\geometry{margin=1in}

\usepackage{amsmath,amssymb}
\usepackage{hyperref}
\hypersetup{
  colorlinks=true,
  linkcolor=blue,
  citecolor=blue,
  urlcolor=blue,
  pdftitle={Structural Equivalence Under Admissibility Constraints},
  pdfauthor={Pascal Sparidaens}
}

\usepackage{enumitem}
\setlist{nosep}

% ---------- Title ----------
\title{\textbf{Structural Equivalence Under Admissibility Constraints}}
\author{
  Pascal Sparidaens\\
  \small Independent Researcher\\
  \small \texttt{pascalsparidaens@gmail.com}
}
\date{\today}

\begin{document}
\maketitle

\begin{abstract}
This paper presents a derived demonstration (D-1) within the
2$\times$4$\times$8 admissibility framework.
It shows that a broad class of bounded branching structures exhibit
admissibility behavior equivalent to that defined by the framework,
without introducing new operators, interpretations, or semantic claims.

The purpose of this paper is not explanation, but structural correspondence.
\end{abstract}

\section{Scope and Relation to Prior Work}
This paper is a derived study within the constrained sequence of
2$\times$4$\times$8 admissibility papers.

The admissibility coordinate system is defined in
\emph{A 2$\times$4$\times$8 Admissibility Framework}.
Rules governing illegal reconstruction are specified in
\emph{Reading Without Reconstruction}.
Diagrammatic addressing conventions are defined in
\emph{Diagrammatic Addressing in the COP Framework}.

The present paper introduces no new admissibility rules, no new symbols,
and no additional diagrammatic conventions. It derives a structural
correspondence using only the existing framework.

\section{Purpose of the Demonstration}
Admissibility frameworks are frequently criticized for being overly abstract.
This paper addresses that criticism by demonstrating how the framework
constrains a familiar class of bounded structures, while maintaining
strict discipline against interpretation.

No claims are made regarding physical causation, biological function,
cognition, or ontology. The structure considered serves only as a
constraint surface.

\section{Definition of the Constraint Surface}
Consider a bounded structure characterized by:
\begin{itemize}
  \item a finite set of positions,
  \item branching adjacency relations,
  \item directed throughput between adjacent positions,
  \item unavoidable local loss,
  \item and selective removal of relations over time.
\end{itemize}

No semantic interpretation of these elements is assumed.
They are defined purely by their constraint properties.

\section{Forward Admissibility and Restriction}
Within the admissibility framework, forward operations irreversibly
restrict the set of future admissible configurations.

In a bounded branching structure, unrestricted branching is not sustainable.
As forward restriction accumulates, some relations must be excluded in order
to maintain coherence under finite constraints.

This selective restriction does not negate prior configurations.
It limits only what may follow.

\section{Reverse Admissibility and Exclusion}
Reverse admissibility operates concurrently with forward restriction.
Reverse operations do not reconstruct earlier states.
They exclude histories that are incompatible with present constraints.

As forward restriction accumulates, multiple forward trajectories may remain
admissible while certain past interpretations become impossible.
This asymmetry does not imply reversibility or symmetry between directions.

\section{Structural Equivalence}
The behaviors described above correspond directly to admissibility constraints
already defined in the framework:
\begin{itemize}
  \item branching corresponds to multiplicity without disorder,
  \item throughput corresponds to admissible process,
  \item local loss enforces selectivity,
  \item and pruning corresponds to restriction under bounded capacity.
\end{itemize}

No additional operators are required to account for these behaviors.
They are enforced by admissibility alone.

Any bounded branching structure exhibiting irreversible restriction,
local loss, and selective persistence will therefore display the same
admissibility behavior, independent of domain, scale, or material composition.

\section{Limits of Interpretation}
This demonstration does not license interpretation.
It does not explain the structure under consideration, nor does it
represent any external system.

Any attempt to read this correspondence as causal, explanatory, or
ontological constitutes an illegal reconstruction and falls outside
the admissibility of the framework.

\section{Conclusion}
This paper has demonstrated a structural equivalence between bounded
branching structures and the admissibility framework, without introducing
new rules, semantics, or interpretations.

The result is not an explanation, but a confirmation of constraint
compatibility. That suffices for the purpose of a derived admissibility study.

\section*{References}
\begin{enumerate}
\item Pascal Sparidaens,
\emph{A 2$\times$4$\times$8 Admissibility Framework: A Periodic-Table View of Stable System Interactions},
OSF Preprint, 2026.
\item Pascal Sparidaens,
\emph{Reading Without Reconstruction: Illegal Inference in the 2$\times$4$\times$8 Admissibility Framework},
OSF Preprint, 2026.
\item Pascal Sparidaens,
\emph{Diagrammatic Addressing in the COP Framework},
OSF Preprint, 2026.
\end{enumerate}

\end{document}
