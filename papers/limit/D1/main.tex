\documentclass[11pt]{article}

\usepackage[T1]{fontenc}
\usepackage[utf8]{inputenc}
\usepackage{lmodern}
\usepackage{microtype}
\usepackage{amsmath,amssymb,amsthm}
\usepackage{graphicx}
\usepackage{hyperref}
\usepackage{url}
\usepackage{booktabs}
\usepackage{enumitem}

\usepackage{../../../templates/cop/cop-macros}
\usepackage{../../../templates/cop/cop-diagrams}

\title{D1: COP as a Persistence Grammar}
\author{Pascal Sparidaens}
\date{}

\begin{document}
\maketitle

\begin{abstract}
This paper introduces no new COP structure.
It formalizes the observation that any system exhibiting persistence under
constraint necessarily conforms to the Canonical Operating Paradoxes (COP).
The result is not a theory of life, systems, or reality, but a grammar of
admissibility shared by all branching-flow structures once semantic content
is removed.
\end{abstract}

\section{Purpose}

This paper serves a single role:
to expose the structural equivalence between the COP framework and a minimal,
semantics-free persistence skeleton.

It does not:
\begin{itemize}
  \item define life,
  \item explain systems,
  \item propose mechanisms,
  \item or introduce ontological claims.
\end{itemize}

Its sole contribution is to show that COP is not domain-specific,
but structurally unavoidable for any system that persists.

\section{Preliminary Constraint}

All references to empirical domains, metaphors, or interpretations
(tree structures, organisms, networks, cognition, etc.)
are explicitly excluded.

Only abstract structure is admissible.

\section{Minimal Persistence Skeleton}

Consider a system $S$ defined only by:

\begin{itemize}
  \item a directed graph $G = (V,E)$,
  \item a positive flow on edges,
  \item unavoidable loss at nodes,
  \item and a pruning operation removing excess structure.
\end{itemize}

No node possesses global knowledge.
No element carries semantic meaning.

Persistence is defined minimally as:
\[
\exists t \to \infty \;\; \text{s.t.} \;\; G_t \neq \emptyset.
\]

No optimization, purpose, or explanation is assumed.

\section{Structural Necessity of COP}

The persistence skeleton above enforces the following constraints:

\begin{itemize}
  \item branching must occur,
  \item flow must be continuous,
  \item loss must be bounded,
  \item excess structure must be pruned,
  \item total possibility cannot be saturated.
\end{itemize}

These constraints map directly onto the COP operators:

\begin{itemize}
  \item branching enforces COP02 (1 $\rightarrow$ VEEL),
  \item flow enforces COP05 (dynamiek),
  \item bounded loss enforces COP06 (selectie),
  \item pruning enforces COP06 shadow behavior,
  \item saturation pressure enforces COP07 (knife-edge).
\end{itemize}

No alternative admissible operators exist.
Any system violating these constraints collapses.

\section{ICOP as Reverse Legibility}

When pruning exceeds flow,
the system enters reverse legibility.

This behavior corresponds to ICOP traversal:
structure dissolves without reconstructable history.
Reverse inspection yields diagnostics only,
never generative inference.

ICOP does not describe destruction.
It marks the loss of admissible continuation.

\section{Non-Implications}

This equivalence does not imply that:

\begin{itemize}
  \item all persistent systems are identical,
  \item COP explains persistence,
  \item COP models reality,
  \item or COP predicts system behavior.
\end{itemize}

COP functions strictly as an admissibility grammar.
It constrains which operations remain coherent.
It does not explain why systems exist.

\section{Why This Is Not a Theory}

The mapping demonstrated here is forced, not explanatory.

COP is not derived from persistence.
Persistence is not derived from COP.

Rather:
any system that persists while respecting constraint
must already obey COP,
or cease to exist as a system.

This is a structural tautology,
not a theoretical claim.

\section{Closure}

Once semantics are stripped,
persistent systems share a common grammar.

That grammar is COP.

No further interpretation is admissible.

\bibliographystyle{unsrt}
\bibliography{references}

\end{document}
