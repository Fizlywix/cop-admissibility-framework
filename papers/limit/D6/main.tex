\documentclass[11pt]{article}

\usepackage[T1]{fontenc}
\usepackage[utf8]{inputenc}
\usepackage{lmodern}
\usepackage{microtype}
\usepackage{amsmath,amssymb,amsthm}
\usepackage{graphicx}
\usepackage{hyperref}
\usepackage{url}
\usepackage{booktabs}
\usepackage{enumitem}

\usepackage{../../../templates/cop/cop-macros}
\usepackage{../../../templates/cop/cop-diagrams}

\title{D5: Control Without Central Authority}
\author{Pascal Sparidaens}
\date{}

\begin{document}
\maketitle

\begin{abstract}
This demonstration examines control as an emergent property of distributed constraint
rather than as the action of a central authority.
Using classical free-will paradoxes and brain-based decision models as reference,
it shows how coordinated behavior remains admissible under COP forward operation,
and how collapse occurs when agency is localized into a single controlling center.
No resolution is proposed.
The paradoxes are used solely to orient admissible perception.
\end{abstract}

\section{Purpose}

This paper demonstrates a single admissibility distinction:
control may arise from distributed constraint
without requiring a central decision-maker.

Within COP terms, this corresponds to visible coordination under constraint
without escalation into a privileged control node.

The purpose is not to deny agency,
but to show how agency remains coherent
until it is forced into a centralized authority.

\section{What This Demonstration Is Not}

This demonstration does not:
\begin{itemize}
  \item deny subjective experience of choice,
  \item assert determinism or indeterminism,
  \item localize control in neural structures,
  \item reduce behavior to a single cause,
  \item or resolve the free-will debate.
\end{itemize}

Any reading that treats control as requiring a single governing center
misreads the demonstration.

\section{What Becomes Visible}

\subsection{Distributed Constraint and Coordination}

When read admissibly:
\begin{itemize}
  \item behavior emerges from interacting constraints,
  \item coordination occurs without central command,
  \item local adjustments propagate without global oversight.
\end{itemize}

Under COP forward admissibility,
control is visible as constraint satisfaction,
not as executive decision.

\subsection{Why Central Authority Is Not Required}

Central authority is a global attribution.
Control is a local phenomenon.

Nothing in coordinated behavior
requires a single point where decisions are made.
Stability arises from compatibility of constraints,
not from command.

\section{Demonstration: Free Will and the Brain}

Free-will paradoxes demonstrate a tension,
not a contradiction.

\begin{itemize}
  \item Decisions appear unified.
  \item Neural activity is distributed and parallel.
\end{itemize}

The paradox arises only when:
\begin{itemize}
  \item unity of action is mapped to a single cause,
  \item control is localized into a point of origin,
  \item agency is treated as a substance.
\end{itemize}

When control is read as distributed constraint optimization,
the paradox loses its force.

\section{How Collapse Occurs}

Collapse occurs when:
\begin{itemize}
  \item agency is forced into a central controller,
  \item decisions are traced to a single origin,
  \item coordination is re-narrated as command,
  \item distributed processes are ignored.
\end{itemize}

At this point, admissible coordination
is replaced by explanatory fixation.

\section{Admissible Reading}

This demonstration permits the following reading:

\begin{quote}
Control may emerge from distributed constraints
without requiring a central authority or decision point.
\end{quote}

Agency remains intelligible
without localization.

\section{Stop Condition}

This demonstration stops at the point where:
\begin{itemize}
  \item control is attributed to a single governing center,
  \item agency is treated as a causal substance,
  \item explanation replaces structural observation.
\end{itemize}

Beyond this point, coordination collapses into command narrative.

\section{Conclusion}

Free-will paradoxes do not negate control.
They expose the error of centralizing authority.

Control persists through constraint.
Authority is optional.

Nothing further is claimed.

\bibliographystyle{unsrt}
\bibliography{references}

\end{document}
