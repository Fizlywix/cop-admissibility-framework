\documentclass[11pt]{article}

\usepackage[T1]{fontenc}
\usepackage[utf8]{inputenc}
\usepackage{lmodern}
\usepackage{microtype}
\usepackage{amsmath,amssymb,amsthm}
\usepackage{graphicx}
\usepackage{hyperref}
\usepackage{url}
\usepackage{booktabs}
\usepackage{enumitem}

\usepackage{../../../templates/cop/cop-macros}
\usepackage{../../../templates/cop/cop-diagrams}

\title{D5: Why VEEL $\rightarrow$ 1 Is Not Collapse}
\author{Pascal Sparidaens}
\date{}

\begin{document}
\maketitle

\begin{abstract}
This paper introduces no new structure.
It clarifies why the transition from multiplicity to unity within the COP
framework does not constitute collapse, failure, or annihilation.
The reduction from VEEL to 1 is shown to be a necessary consequence of
admissibility under constraint.
\end{abstract}

\section{Purpose}

This paper addresses a recurring misinterpretation:
that reduction of multiplicity implies destruction or loss of validity.

It does not.

This paper specifies why consolidation is admissible,
while collapse is a narrative misreading.

\section{Multiplicity Under Constraint}

VEEL denotes constrained multiplicity.
It exists only while admissibility remains available.

As constraint accumulates:
\begin{itemize}
  \item distinctions lose viability,
  \item resource demands increase,
  \item and continuation becomes restricted.
\end{itemize}

Reduction is therefore not optional.

\section{Reduction as Structural Selection}

The transition from VEEL to 1 occurs through selection,
not through annihilation.

Configurations are not destroyed.
They are rendered inadmissible.

Inadmissibility is a property of articulation,
not of existence.

\section{Why Collapse Is a Narrative Error}

Collapse implies:
\begin{itemize}
  \item sudden failure,
  \item catastrophic loss,
  \item or total negation.
\end{itemize}

None of these are structurally required.

Reduction proceeds incrementally
through constraint enforcement.

Calling this collapse introduces narrative drama
where only pruning occurs.

\section{Persistence of Grammar}

Although multiplicity is reduced,
the admissibility grammar remains intact.

The resulting unity is not privileged,
complete, or final.

It is simply the remaining admissible configuration.

\section{Non-Reversibility}

The transition from VEEL to 1 is not reversible.

Eliminated configurations cannot be restored,
enumerated, or recovered.

This irreversibility is structural,
not temporal.

\section{Forbidden Interpretations}

The following interpretations are inadmissible:

\begin{itemize}
  \item equating reduction with destruction,
  \item interpreting unity as final truth,
  \item treating consolidation as resolution,
  \item attributing agency to selection.
\end{itemize}

All reintroduce narrative closure.

\section{Relation to Return}

Reduction to 1 does not preclude return.

The remaining configuration may itself
reach saturation and force re-entry.

Reduction therefore prepares continuation,
not termination.

\section{Closure}

VEEL $\rightarrow$ 1 is not collapse.

It is the minimal condition
under which admissibility persists.

No further interpretation is admissible.

\bibliographystyle{unsrt}
\bibliography{references}

\end{document}
