\documentclass[11pt]{article}

\usepackage[T1]{fontenc}
\usepackage[utf8]{inputenc}
\usepackage{lmodern}
\usepackage{microtype}
\usepackage{amsmath,amssymb,amsthm}
\usepackage{graphicx}
\usepackage{hyperref}
\usepackage{url}
\usepackage{booktabs}
\usepackage{enumitem}

\usepackage{../../../templates/cop/cop-macros}
\usepackage{../../../templates/cop/cop-diagrams}

\title{D2: Motion Without Completion}
\author{Pascal Sparidaens}
\date{}

\begin{document}
\maketitle

\begin{abstract}
This demonstration examines motion as a form of continuity
that does not require global completion or final arrival.
Using Zeno’s paradoxes as a concrete reference,
it shows how motion remains intelligible when read locally,
and how it collapses when forced into total traversal.
No resolution is proposed.
The paradox is used solely to orient admissible perception.
\end{abstract}

\section{Purpose}

This paper demonstrates a simple distinction:
motion may persist without completion.

Its purpose is not to solve Zeno’s paradoxes,
but to show what becomes visible
when motion is read as local progression
rather than as a demand for total traversal.

\section{What This Demonstration Is Not}

This demonstration does not:
\begin{itemize}
  \item explain motion metaphysically,
  \item deny infinite divisibility,
  \item privilege mathematical limits,
  \item assert that arrival must occur,
  \item or dissolve the paradox.
\end{itemize}

Any attempt to turn this paper into a solution
misreads its intent.

\section{What Becomes Visible}

\subsection{Motion as Local Continuity}

When motion is read locally:
\begin{itemize}
  \item each step is complete in itself,
  \item progression does not depend on future states,
  \item continuity is preserved without reference to an endpoint.
\end{itemize}

Motion is experienced and operational
without requiring knowledge of the full path.

\subsection{Why Completion Is Not Required}

Completion is a global property.
Motion is a local phenomenon.

Nothing in the experience or operation of motion
requires that the total distance be exhausted
or that a final position be reached.

Motion does not borrow legitimacy from arrival.

\section{Demonstration: Zeno’s Paradoxes}

Zeno’s paradoxes demonstrate a tension,
not an impossibility.

\begin{itemize}
  \item Motion proceeds step by step without contradiction.
  \item The paradox appears only when infinite subdivision
        is treated as a requirement for total completion.
\end{itemize}

When motion is forced into a global accounting problem,
progress collapses into abstraction.
When motion is allowed to remain local,
the paradox loses its force.

\section{How Collapse Occurs}

Collapse occurs when:
\begin{itemize}
  \item motion is evaluated only by its endpoint,
  \item traversal is treated as an all-or-nothing condition,
  \item infinite divisibility is interpreted as denial of progress.
\end{itemize}

In these readings, motion is no longer observed;
it is replaced by a demand for totalization.

\section{Admissible Reading}

This demonstration permits the following reading:

\begin{quote}
Motion is valid wherever local progression occurs,
regardless of whether global completion is possible or defined.
\end{quote}

This reading preserves continuity
without resolving infinity.

\section{Stop Condition}

This demonstration stops at the point where:
\begin{itemize}
  \item motion is treated as invalid without arrival,
  \item paradox is used to deny lived continuity,
  \item explanation replaces observation.
\end{itemize}

Beyond this point, narration replaces demonstration.

\section{Conclusion}

Zeno’s paradoxes do not negate motion.
They expose the error of demanding completion.

Motion persists locally.
Completion is optional.

Nothing further is claimed.

\bibliographystyle{unsrt}
\bibliography{references}

\end{document}
