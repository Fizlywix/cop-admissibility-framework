\documentclass[11pt]{article}

\usepackage[T1]{fontenc}
\usepackage[utf8]{inputenc}
\usepackage{lmodern}
\usepackage{microtype}
\usepackage{amsmath,amssymb,amsthm}
\usepackage{graphicx}
\usepackage{hyperref}
\usepackage{url}
\usepackage{booktabs}
\usepackage{enumitem}

\usepackage{../../../templates/cop/cop-macros}
\usepackage{../../../templates/cop/cop-diagrams}

\title{D2: ICOP as Loss of Legibility}
\author{Pascal Sparidaens}
\date{}

\begin{document}
\maketitle

\begin{abstract}
This paper introduces no new structure.
It clarifies the status of ICOP within the COP framework.
ICOP is defined not as negation, destruction, or absence,
but as progressive loss of legibility under constraint.
\end{abstract}

\section{Purpose}

This paper addresses a systematic misinterpretation:
that ICOP represents negation, death, or non-existence.

It does not.

ICOP marks the breakdown of admissible articulation.
It specifies what can no longer be read,
not what no longer exists.

\section{Forward Operation and Reverse Legibility}

COP operators govern admissible forward operations.
They specify how structure may be written under constraint.

ICOP operators govern reverse legibility.
They specify what can be inspected after admissibility
has been exceeded.

Reverse legibility is diagnostic only.
It cannot generate admissible forward moves.

\section{Legibility Versus Existence}

Loss of legibility must not be conflated with loss of existence.

A structure may persist while becoming unreadable.
Conversely, readable structure may disappear
without implying annihilation.

ICOP tracks only the former.

No claim is made about ontological status.

\section{Why ICOP Is Not Inverse COP}

ICOP operators are not inverses of COP operators.

An inverse would require:
\begin{itemize}
  \item preservation of structure,
  \item reversible operations,
  \item and recoverable history.
\end{itemize}

None of these are admissible once saturation is reached.

ICOP therefore does not undo COP.
It marks the erosion of admissibility itself.

\section{Asymmetry of Inspection}

Forward operation accumulates constraint.
Reverse inspection removes distinction.

This asymmetry is structural and non-negotiable.

No operation exists that reconstructs forward admissibility
from ICOP traversal.

Any such attempt constitutes narrative reconstruction.

\section{Why ICOP Appears Destructive}

ICOP is often described in destructive terms
because loss of legibility removes recognizable structure.

This appearance is misleading.

ICOP does not destroy structure.
It renders structure indistinguishable
with respect to admissible articulation.

\section{Forbidden Reconstructions}

The following interpretations are illegitimate:

\begin{itemize}
  \item treating ICOP as causal history,
  \item reconstructing origins from ICOP descent,
  \item equating ICOP with annihilation or nothingness,
  \item attributing agency to reverse traversal.
\end{itemize}

ICOP does not explain what happened.
It only constrains what can still be read.

\section{Diagnostic Function}

ICOP serves a single function:
to indicate that admissible continuation has been exceeded.

At this point:
\begin{itemize}
  \item explanation fails,
  \item interpretation collapses,
  \item and further articulation becomes illegitimate.
\end{itemize}

No corrective action is defined.

\section{Closure}

ICOP does not negate COP.
It preserves the framework by enforcing opacity
where articulation would otherwise be forced.

Loss of legibility is not failure.
It is a structural boundary.

\bibliographystyle{unsrt}
\bibliography{references}

\end{document}
