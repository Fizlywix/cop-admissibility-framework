\documentclass[11pt]{article}

\usepackage[T1]{fontenc}
\usepackage[utf8]{inputenc}
\usepackage{lmodern}
\usepackage{microtype}
\usepackage{amsmath,amssymb,amsthm}
\usepackage{graphicx}
\usepackage{hyperref}
\usepackage{url}
\usepackage{booktabs}
\usepackage{enumitem}

\usepackage{../../../templates/cop/cop-macros}
\usepackage{../../../templates/cop/cop-diagrams}

\title{D6: Misuse Taxonomy}
\author{Pascal Sparidaens}
\date{}

\begin{document}
\maketitle

\begin{abstract}
This paper introduces no new structure.
It catalogues predictable misuse patterns of the COP framework.
These patterns are not corrected or repaired.
They function as diagnostic markers indicating loss of admissibility.
\end{abstract}

\section{Purpose}

This paper exists to expose recurrent failure modes
that arise when the COP framework is extended beyond admissibility.

Misuse is treated as expected behavior.
The framework does not attempt to prevent it.
It only renders misuse visible.

No corrections are provided.

\section{Scope of Misuse}

Misuse occurs when the framework is employed as anything other than
an admissibility constraint.

In particular, misuse arises when COP is treated as:

\begin{itemize}
  \item an explanatory theory,
  \item a descriptive model,
  \item an ontological claim,
  \item a methodological authority,
  \item or a source of meaning.
\end{itemize}

Each misuse corresponds to a distinct structural violation.

\section{Narrativization}

Narrativization occurs when COP operators are arranged
into a temporal or causal story.

Indicators include:
\begin{itemize}
  \item origin myths,
  \item progressions toward completion,
  \item cycles with preserved identity,
  \item or causal chains across operators.
\end{itemize}

Narrativization violates reading constraints
by reconstructing continuity where none is admissible.

\section{Explanation Substitution}

Explanation substitution occurs when admissibility is treated
as explanation.

Typical forms include:
\begin{itemize}
  \item claims that COP explains why systems exist,
  \item claims that COP explains behavior or outcomes,
  \item claims that COP reveals hidden mechanisms.
\end{itemize}

Admissibility excludes illegitimate moves.
It does not provide causes.

\section{Ontological Elevation}

Ontological elevation occurs when COP structures
are treated as entities.

Examples include:
\begin{itemize}
  \item treating operators as real objects,
  \item treating boundaries as metaphysical layers,
  \item treating diagrams as representations of reality.
\end{itemize}

COP does not enumerate what exists.
It constrains what can be coherently articulated.

\section{Authority Projection}

Authority projection occurs when COP is used
to justify decisions, claims, or actions.

Indicators include:
\begin{itemize}
  \item appeals to COP as validation,
  \item exclusion of alternatives on COP grounds,
  \item invocation of COP as intellectual leverage.
\end{itemize}

The framework confers no authority.
Using it as such constitutes misuse.

\section{Optimization Attempts}

Optimization misuse occurs when COP is treated
as something to be improved, completed, or made more efficient.

This includes:
\begin{itemize}
  \item attempts to add operators,
  \item attempts to close the framework,
  \item attempts to resolve paradoxes.
\end{itemize}

The framework is not incomplete.
Attempts at optimization violate its stopping conditions.

\section{Identity Attachment}

Identity attachment occurs when adherence to COP
becomes part of personal or collective identity.

Indicators include:
\begin{itemize}
  \item belief statements,
  \item defensive reactions to critique,
  \item evangelization or protection of the framework.
\end{itemize}

COP does not sustain identity.
Attachment signals misuse.

\section{Diagnostic Status}

The misuse patterns listed above are not errors to be fixed.

They are diagnostic signals indicating that admissibility
has been exceeded.

At this point, continued use of the framework is illegitimate.

\section{Closure}

Misuse does not invalidate the COP framework.
It indicates where its use must stop.

No further response is admissible.

\bibliographystyle{unsrt}
\bibliography{references}

\end{document}
