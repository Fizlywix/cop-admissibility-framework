\documentclass[11pt]{article}

\usepackage[T1]{fontenc}
\usepackage[utf8]{inputenc}
\usepackage{lmodern}
\usepackage{microtype}
\usepackage{amsmath,amssymb,amsthm}
\usepackage{graphicx}
\usepackage{hyperref}
\usepackage{url}
\usepackage{booktabs}
\usepackage{enumitem}

\usepackage{../../../templates/cop/cop-macros}
\usepackage{../../../templates/cop/cop-diagrams}

\title{D8: The Right Way to Stop}
\author{Pascal Sparidaens}
\date{}

\begin{document}
\maketitle

\begin{abstract}
This paper introduces no new structure.
It specifies the correct termination conditions for the use of the COP framework.
Stopping is treated not as failure, mystery, or completion, but as an admissible
and necessary operation once constraint surfaces are reached.
\end{abstract}

\section{Purpose}

This paper exists to prevent a specific misuse:
continued operation beyond admissibility.

The COP framework does not converge.
It does not resolve.
It does not culminate.

Correct use therefore requires a disciplined stopping condition.

This paper specifies that condition.

\section{Stopping Is an Operation}

Within the COP framework, stopping is not the absence of action.
It is an admissible action in its own right.

Continued articulation past constraint does not add information.
It produces illegibility.

Stopping preserves coherence.

\section{When to Stop}

Use of the framework must cease when any of the following occur:

\begin{itemize}
  \item further articulation repeats existing distinctions,
  \item additional structure fails to introduce new admissibility constraints,
  \item interpretation pressure replaces operational discipline,
  \item explanation is demanded where only exclusion remains,
  \item or the framework itself becomes the object of justification.
\end{itemize}

At this point, continuation is illegitimate.

\section{What Stopping Is Not}

Stopping must not be interpreted as:

\begin{itemize}
  \item completion,
  \item resolution,
  \item explanation,
  \item transcendence,
  \item or epistemic humility.
\end{itemize}

Stopping introduces no new insight.
It merely prevents error.

\section{Why Further Extension Is Forbidden}

Any extension beyond the defined COP and ICOP operators necessarily introduces:

\begin{itemize}
  \item unbounded totality,
  \item narrative reconstruction,
  \item explanatory closure,
  \item or hidden ontology.
\end{itemize}

These moves violate admissibility.

The framework does not fail at this point.
It has been correctly exhausted.

\section{Abandonment as Correct Use}

The COP framework does not require maintenance.

It may be:
\begin{itemize}
  \item set aside,
  \item ignored,
  \item replaced,
  \item or forgotten.
\end{itemize}

Abandonment does not negate correctness.
Retention beyond usefulness does.

\section{Non-Authority Clause}

The framework confers no authority.

It does not:
\begin{itemize}
  \item demand belief,
  \item justify decisions,
  \item prescribe actions,
  \item or arbitrate truth.
\end{itemize}

Any attempt to wield COP as authority constitutes misuse.

\section{Final Constraint}

The only invariant enforced by the framework is this:

\begin{quote}
If continued articulation is required for coherence,
coherence has already been lost.
\end{quote}

No further statement is admissible.

\bibliographystyle{unsrt}
\bibliography{references}

\end{document}
