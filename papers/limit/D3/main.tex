\documentclass[11pt]{article}

\usepackage[T1]{fontenc}
\usepackage[utf8]{inputenc}
\usepackage{lmodern}
\usepackage{microtype}
\usepackage{amsmath,amssymb,amsthm}
\usepackage{graphicx}
\usepackage{hyperref}
\usepackage{url}
\usepackage{booktabs}
\usepackage{enumitem}

\usepackage{../../../templates/cop/cop-macros}
\usepackage{../../../templates/cop/cop-diagrams}

\title{D3: Return Without Cycle}
\author{Pascal Sparidaens}
\date{}

\begin{document}
\maketitle

\begin{abstract}
This paper introduces no new structure.
It clarifies the status of return within the COP framework.
Return is shown to be neither recurrence, repetition, nor cyclic restoration,
but a structural re-entry condition without state preservation.
\end{abstract}

\section{Purpose}

This paper addresses a persistent misreading:
that return within the COP framework implies cyclicity,
recurrence, or repetition of states.

It does not.

This paper specifies why return is admissible,
while cycles, recurrence, and eternal repetition are not.

\section{Definition of Return}

Within the COP framework, return denotes a forced re-entry
into admissible structure following saturation.

Return is not:
\begin{itemize}
  \item a temporal loop,
  \item a restoration of prior states,
  \item a repetition of history,
  \item or a reversible operation.
\end{itemize}

Return marks the loss of admissible continuation
and the initiation of a new write.

\section{Why Cycles Are Illegitimate}

A cycle implies that a prior configuration can be re-entered
with its structure intact.

This requires:
\begin{itemize}
  \item preservation of state,
  \item preservation of distinctions,
  \item and retrievable history.
\end{itemize}

All three are forbidden under saturation.
Once admissibility is exhausted,
no structure remains that could support recurrence.

Cycles therefore constitute narrative reconstruction,
not admissible operation.

\section{Structural Asymmetry of Return}

Return is asymmetric by construction.

Forward traversal accumulates constraint.
Reverse traversal removes legibility.

No mapping exists from a post-return configuration
to a pre-return configuration.

The similarity of form across returns
does not imply identity of state.

\section{Why Return Appears Cyclic}

Return is often misread as cyclic
because structural constraints are reused.

The COP framework enforces the same admissibility grammar
on every re-entry.

This produces resemblance without repetition.

Structural similarity must not be confused with recurrence.

\section{Non-Temporal Character of Return}

Return does not occur \emph{in} time.

Time is a property of admissible structure.
When admissibility collapses,
temporal ordering is no longer defined.

Return therefore cannot be located
as an event in a temporal sequence.

\section{Forbidden Interpretations}

Return must not be interpreted as:

\begin{itemize}
  \item rebirth,
  \item renewal,
  \item reset,
  \item oscillation,
  \item or eternal return.
\end{itemize}

All such interpretations introduce narrative continuity
where none is admissible.

\section{Closure}

Return preserves the framework by preventing illegitimate continuation.

It does not restore.
It does not repeat.
It does not conclude.

It merely permits admissibility to resume.

\bibliographystyle{unsrt}
\bibliography{references}

\end{document}
