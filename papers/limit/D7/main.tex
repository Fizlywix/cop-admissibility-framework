\documentclass[11pt]{article}

\usepackage[T1]{fontenc}
\usepackage[utf8]{inputenc}
\usepackage{lmodern}
\usepackage{microtype}
\usepackage{amsmath,amssymb,amsthm}
\usepackage{graphicx}
\usepackage{hyperref}
\usepackage{url}
\usepackage{booktabs}
\usepackage{enumitem}

\usepackage{../../../templates/cop/cop-macros}
\usepackage{../../../templates/cop/cop-diagrams}

\title{D7: Why This Is Not a Theory}
\author{Pascal Sparidaens}
\date{}

\begin{document}
\maketitle

\begin{abstract}
This paper introduces no new structure.
It specifies why the COP framework must not be interpreted as a theory,
model, or explanatory system.
The distinction between admissibility and explanation is made explicit,
and the consequences of conflating the two are enumerated.
\end{abstract}

\section{Purpose}

This paper exists to terminate a recurrent misclassification:
treating the COP framework as a theory.

It is not.

This paper does not argue against theories.
It clarifies category boundaries.

\section{What a Theory Requires}

A theory minimally requires:

\begin{itemize}
  \item explanatory claims,
  \item descriptive correspondence,
  \item predictive scope,
  \item or ontological commitment.
\end{itemize}

A theory answers questions of the form:
\emph{what exists}, \emph{how it behaves}, or \emph{why it occurs}.

The COP framework answers none of these.

\section{What COP Provides Instead}

The COP framework provides:

\begin{itemize}
  \item admissibility constraints,
  \item exclusion of illegitimate moves,
  \item stopping conditions under saturation,
  \item and discipline against narrative completion.
\end{itemize}

These functions are orthogonal to explanation.

COP does not describe systems.
It constrains articulation about systems.

\section{Why Explanation Is Inadmissible}

Explanation attempts to stabilize meaning
by reconstructing causality or structure.

Within the COP framework, such reconstruction:

\begin{itemize}
  \item exceeds admissibility,
  \item reintroduces narrative continuity,
  \item and violates stopping conditions.
\end{itemize}

Explanation is therefore not postponed.
It is excluded.

\section{Non-Comparability to Other Frameworks}

COP must not be compared to:

\begin{itemize}
  \item physical theories,
  \item mathematical models,
  \item philosophical systems,
  \item or computational formalisms.
\end{itemize}

Comparison presupposes shared explanatory goals.

COP has none.

Any comparison produces category error.

\section{Why COP Cannot Be Evaluated as True or False}

Truth evaluation requires propositions.

COP provides constraints, not propositions.

Statements such as:
\begin{itemize}
  \item “COP is true,”
  \item “COP is false,”
  \item or “COP is correct”
\end{itemize}

are therefore ill-typed.

The framework can only be used correctly or incorrectly.

\section{The Role of Failure}

Failure within COP is not refutation.

When admissibility is exceeded:
\begin{itemize}
  \item legibility collapses,
  \item misuse becomes visible,
  \item and further articulation is blocked.
\end{itemize}

This behavior confirms the framework’s role.
It does not undermine it.

\section{Non-Substitution Clause}

COP does not replace:

\begin{itemize}
  \item theories,
  \item models,
  \item methods,
  \item or domains of inquiry.
\end{itemize}

It may be applied alongside them
to constrain illegitimate extension.

It may also be set aside without consequence.

\section{Closure}

The COP framework is not a theory.

It does not explain.
It does not predict.
It does not describe.

It only enforces where articulation must stop.

No further classification is admissible.

\bibliographystyle{unsrt}
\bibliography{references}

\end{document}
