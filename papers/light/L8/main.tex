\documentclass[11pt]{article}

\usepackage[T1]{fontenc}
\usepackage[utf8]{inputenc}
\usepackage{lmodern}
\usepackage{microtype}
\usepackage{amsmath,amssymb,amsthm}
\usepackage{graphicx}
\usepackage{hyperref}
\usepackage{url}
\usepackage{booktabs}
\usepackage{enumitem}

\usepackage{../../../templates/cop/cop-macros}
\usepackage{../../../templates/cop/cop-diagrams}

\title{L8: Applied Constraint Surfaces}
\author{Pascal Sparidaens}
\date{}

\begin{document}
\maketitle

\begin{abstract}
This paper demonstrates how the COP framework may be used without becoming an
interpretive or explanatory model.
Rather than applying the framework to domains, it treats admissibility as a
constraint surface on which operations may be performed.
The emphasis is on disciplined use, not outcome.
\end{abstract}

\section{What It Means to Apply a Constraint}

In the COP framework, application does not mean explanation.
It does not consist in mapping concepts, behaviors, or meanings onto the table.

To apply the framework is to:
\begin{itemize}
  \item place an interaction on a constraint surface,
  \item attempt admissible operations,
  \item observe exclusions,
  \item and stop where admissibility ends.
\end{itemize}

No interpretation is required.

\section{Constraint Surfaces}

A constraint surface is any structure on which admissibility can be evaluated.
The surface itself may be abstract, physical, computational, or conceptual.

The framework does not privilege any surface.
It requires only that:
\begin{itemize}
  \item the surface be bounded,
  \item constraints be enforceable,
  \item and illegitimate moves be detectable.
\end{itemize}

Surfaces do not encode meaning.
They host admissibility.

\section{Operational Use Without Interpretation}

An admissible use proceeds as follows:
\begin{enumerate}
  \item Identify a bounded surface.
  \item Locate admissible writes.
  \item Apply forward restriction until saturation.
  \item Apply reverse exclusion where incompatibility appears.
  \item Observe return when no further distinction is admissible.
\end{enumerate}

At no point is the surface interpreted.
The procedure terminates naturally when admissibility is exhausted.

\section{Stopping Conditions}

Correct application includes knowing when to stop.

Reasoning must cease when:
\begin{itemize}
  \item only exclusion remains,
  \item no new admissible distinctions can be introduced,
  \item further moves would require reconstruction.
\end{itemize}

Continuing beyond this point produces artifacts, not insight.

\section{Why No Examples Are Given}

This paper contains no worked examples.
This is intentional.

Examples invite interpretation.
They encourage readers to treat admissibility as explanation or prediction.

The framework remains valid without exemplars.
Its correctness is structural, not illustrative.

\section{Relation to the COP Corpus}

This paper assumes:
\begin{itemize}
  \item the coordinate system of \emph{L2: A 2$\times$4$\times$8 Admissibility Framework},
  \item the reading discipline of \emph{L3: Reading Without Reconstruction},
  \item the diagrammatic grammar of \emph{L4: Diagrammatic Addressing in the COP Framework},
  \item the operational behavior shown in \emph{L5: Operational Admissibility in a Bounded Structure},
  \item the survivability framing of \emph{L6: The Periodic Table Is Not About Elements},
  \item and the visibility analysis of \emph{L7: Structural Visibility Under Stress}.
\end{itemize}

No additional structure is introduced.

\section{Conclusion}

The COP framework does not explain systems.
It constrains interaction.

Applied correctly, it disappears into the surface on which it operates,
leaving only excluded impossibilities.

\bibliographystyle{unsrt}
\bibliography{references}

\end{document}

