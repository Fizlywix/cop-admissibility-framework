\documentclass[11pt]{article}

\usepackage[T1]{fontenc}
\usepackage[utf8]{inputenc}
\usepackage{lmodern}
\usepackage{microtype}
\usepackage{amsmath,amssymb,amsthm}
\usepackage{graphicx}
\usepackage{hyperref}
\usepackage{url}
\usepackage{booktabs}
\usepackage{enumitem}

\usepackage{../../../templates/cop/cop-macros}
\usepackage{../../../templates/cop/cop-diagrams}

\title{L4: Diagrammatic Addressing in the COP Framework}
\author{Pascal Sparidaens}
\date{}

\begin{document}
\maketitle

\begin{abstract}
This paper specifies the diagrammatic addressing rules used in the COP framework.
It does not explain, interpret, or derive the diagrams. Instead, it defines the admissible
operations encoded by the visual language and the conditions under which a diagrammatic
reading is valid.
\end{abstract}

\section{Purpose and Scope}

This paper serves as a diagrammatic legend for the COP framework.
Its function is to define how diagrams are read, not what they mean. :contentReference[oaicite:1]{index=1}

The figures associated with this paper (Figures 1--3) are constraint objects.
They do not illustrate concepts introduced here. Rather, the present text specifies
the addressing rules that make those diagrams admissible. :contentReference[oaicite:2]{index=2}

No semantic interpretation, narrative explanation, or domain reconstruction is permitted. :contentReference[oaicite:3]{index=3}

\section{Diagrammatic Objects}

A COP diagram consists of the following visual objects: :contentReference[oaicite:4]{index=4}
\begin{itemize}
  \item Write boxes
  \item Arrows
  \item Copter structures
  \item Signs ($+$, $-$)
  \item Seams and return regions
\end{itemize}

Each object has a single operational role.
No object carries semantic meaning beyond its admissible function. :contentReference[oaicite:5]{index=5}

\section{Write Boxes}

Write boxes denote loci of commitment.
A write does not represent a state, value, or proposition.
It represents the act of restriction itself. :contentReference[oaicite:6]{index=6}

All writes are additive.
No write erases or overwrites a prior write.
What is excluded remains excluded without reconstruction. :contentReference[oaicite:7]{index=7}

\section{Arrows}

Arrows indicate addressing relations to a write.
They do not indicate flow, time, causation, hierarchy, or direction in space. :contentReference[oaicite:8]{index=8}

An arrow always refers to the write it touches.
Arrow orientation is local and does not encode global ordering. :contentReference[oaicite:9]{index=9}

Arrows do not initiate operations.
They mark where admissibility is evaluated. :contentReference[oaicite:10]{index=10}

\section{Signs}

The symbols $+$ and $-$ denote admissible pressure types: :contentReference[oaicite:11]{index=11}
\begin{itemize}
  \item $+$ indicates forward restriction
  \item $-$ indicates reverse exclusion
\end{itemize}

Signs are local.
They do not cancel, balance, or negate each other.
Multiple signs may be active concurrently. :contentReference[oaicite:12]{index=12}

No sign encodes success, failure, progress, or regression. :contentReference[oaicite:13]{index=13}

\section{Copter Structures}

Copter structures encode resolution change.
They do not encode traversal, motion, or direction. :contentReference[oaicite:14]{index=14}

A primary copter admits both upward and downward resolution addressing.
Branch copters are resolution-restricted and may only resolve downward. :contentReference[oaicite:15]{index=15}

Copter symmetry does not imply operational equivalence. :contentReference[oaicite:16]{index=16}

\section{Return and Seams}

Return is not a destination and is not commanded by arrows.
Return is revealed when no further admissible distinction is possible. :contentReference[oaicite:17]{index=17}

Seams denote adjacency without traversal.
They enforce return structurally rather than dynamically. :contentReference[oaicite:18]{index=18}

Return may occur concurrently with both forward restriction and reverse exclusion. :contentReference[oaicite:19]{index=19}

\section{Progressive Exposure of Operations}

Figures 1--3 differ only by admissible operation exposure: :contentReference[oaicite:20]{index=20}
\begin{itemize}
  \item Figure 1 admits forward restriction ($+$).
  \item Figure 2 admits forward and reverse operations ($+$, $-$).
  \item Figure 3 admits forward restriction, reverse exclusion, and return.
\end{itemize}

No new rules are introduced across figures.
Only admissible operations are progressively exposed. :contentReference[oaicite:21]{index=21}

\begin{figure}[h]
\centering
\includegraphics[width=\linewidth]{figures/fig1-forward-only.jpg}
\caption{Constraint diagram admitting forward restriction ($+$) only. Arrows denote addressing relations to writes; they do not encode causation, flow, or traversal.}
\end{figure}

\begin{figure}[h]
\centering
\includegraphics[width=\linewidth]{figures/fig2-forward-reverse.jpg}
\caption{Constraint diagram admitting both forward restriction ($+$) and reverse exclusion ($-$). Signs are local pressure types and do not encode success, failure, progress, or regression.}
\end{figure}

\begin{figure}[h]
\centering
\includegraphics[width=\linewidth]{figures/fig3-return-exposed.jpg}
\caption{Constraint diagram with return exposed. Return is structural (not causal) and becomes legible only when no further admissible distinction is possible. Seams denote adjacency without traversal.}
\end{figure}

\section{Illegitimate Readings}

The following readings are explicitly inadmissible: :contentReference[oaicite:22]{index=22}
\begin{itemize}
  \item Treating arrows as causal flow
  \item Treating copters as hierarchy
  \item Treating return as reset or origin
  \item Assigning domain semantics to visual objects
  \item Interpreting diagrams as explanations
\end{itemize}

Any reading requiring reconstruction violates the framework. :contentReference[oaicite:23]{index=23}

\section{Relation to the COP Corpus}

This paper depends on:
\begin{itemize}
  \item \emph{L2: A 2$\times$4$\times$8 Admissibility Framework}
  \item \emph{L3: Reading Without Reconstruction}
\end{itemize}
No additional assumptions are introduced. :contentReference[oaicite:24]{index=24}

\section{Conclusion}

The diagrams associated with this paper are complete insofar as they obey the addressing rules specified here.
They do not require explanation. They constrain admissible reading. :contentReference[oaicite:25]{index=25}

\end{document}
