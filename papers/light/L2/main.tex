\documentclass[11pt]{article}

\usepackage[T1]{fontenc}
\usepackage[utf8]{inputenc}
\usepackage{lmodern}
\usepackage{microtype}
\usepackage{amsmath,amssymb,amsthm}
\usepackage{graphicx}
\usepackage{hyperref}
\usepackage{url}
\usepackage{booktabs}
\usepackage{enumitem}

\usepackage{../../templates/cop/cop-macros}
\usepackage{../../templates/cop/cop-diagrams}

\title{L2: A 2×4×8 Admissibility Framework}
\author{Pascal Sparidaens}
\date{}

\begin{document}
\maketitle

\begin{abstract}
This paper introduces a minimal 2×4×8 coordinate system for classifying system interactions by admissibility.
The framework does not predict behavior, reconstruct histories, or assign semantics.
Instead, it specifies which configurations remain coherent under constraint and excludes incompatible ones.
The framework serves as the structural basis for all subsequent papers in the COP corpus.
\end{abstract}

\section{Purpose and Non-Claims}

This paper defines the coordinate system used throughout the COP corpus.
Its purpose is to establish a stable address space for admissible operations.

The framework explicitly does not:
\begin{itemize}
  \item predict outcomes,
  \item reconstruct causal histories,
  \item assume global observability,
  \item rely on semantic interpretation,
  \item or require a privileged observer.
\end{itemize}

Reading discipline and illegitimate inference are addressed separately in
\emph{L3: Reading Without Reconstruction}.

\section{The Three Orthogonal Axes}

Every system interaction is addressed using three independent constraints:
direction, ledger relation, and functional role.

\subsection{Direction (2)}

There are exactly two admissible temporal readings:
\begin{itemize}
  \item Constructive (→): an operation that restricts future admissibility.
  \item Inferential (←): an operation that excludes incompatible pasts.
\end{itemize}

Reverse operations do not reconstruct events; they only prune impossibilities.
The limits of reverse reasoning are treated in detail in
\emph{L3: Reading Without Reconstruction}.

\subsection{Ledger Relation (4)}

Relative to commitment, there are exactly four stable surfaces:
\begin{enumerate}
  \item Latent / Option-space
  \item Write / Commit (↔ Backward Exclusion)
  \item Propagation (↔ Dependency Trace)
  \item Closure / Lock-in (↔ Invariant Inference)
\end{enumerate}

These surfaces are structural necessities.
Systems that violate them lose coherence.

\subsection{Functional Role (8)}

Across domains, eight functional roles recur symmetrically under forward and reverse reading:
\begin{enumerate}
  \item Emergence / Admission
  \item Stabilization / Persistence
  \item Coupling / Connectivity
  \item Optimization / Selectivity
  \item Saturation / Limitation
  \item Maintenance / Support
  \item Backgrounding / Infrastructure
  \item Exit / Elimination
\end{enumerate}

Roles do not encode meaning or domain semantics.
They specify admissible function only.

\section{Coordinate Notation (D.W.R)}

Each interaction is addressed as a coordinate of the form:
\[
\text{D.W.R}
\]
where D is direction, W is ledger relation, and R is functional role.

Examples:
\begin{itemize}
  \item 1.2.3 — forward commit of coupling
  \item 2.2.3 — backward exclusion of impossible connectivity
  \item 1.4.7 — closure into infrastructure
  \item 2.4.7 — inference of infrastructure invariants
\end{itemize}

Notation is purely addressing.
Interpretation beyond admissibility is explicitly forbidden.

\section{Admissibility Rules}

All admissible operations obey the following minimal rules:
\begin{itemize}
  \item Writes are additive; no operation deletes history.
  \item Reverse operations exclude histories; they do not replay them.
  \item No global decider or total audit is assumed.
\end{itemize}

Operational behavior under saturation is demonstrated in
\emph{L5: Operational Admissibility in a Bounded Structure}.

\section{Null Dominance and Observability}

A central consequence of the framework is null dominance:
constraints are invisible while satisfied.

Most of the table is structurally dark:
latent states, infrastructure, and reverse structure are typically legible only
under violation or closure.

Structural visibility under stress is examined in
\emph{L7: Structural Visibility Under Stress}.

\section{Table Interpretation}

The 2×4×8 table is not an enumeration of elements.
It is a pruning snapshot of survivability.

A “stable element” corresponds to regions of trajectories that repeatedly:
\begin{itemize}
  \item close cleanly,
  \item infer invariants,
  \item background into infrastructure,
  \item and disappear without forcing reset.
\end{itemize}

This interpretation is developed further in
\emph{L6: The Periodic Table Is Not About Elements}.

\section{Relation to the COP Corpus}

This paper provides the coordinate system used by:
\begin{itemize}
  \item \emph{L3: Reading Without Reconstruction}
  \item \emph{L4: Diagrammatic Addressing in the COP Framework}
  \item \emph{L5: Operational Admissibility in a Bounded Structure}
  \item \emph{L6: The Periodic Table Is Not About Elements}
  \item \emph{L7: Structural Visibility Under Stress}
  \item \emph{L8: Applied Constraint Surfaces}
\end{itemize}

Paradoxes and non-totality limits are treated separately in the Shadow corpus.

\section{Summary of Admissible Use}

The 2×4×8 framework is an addressing system, not an explanatory model.
Correct use consists in excluding incompatible configurations.
When constraints are satisfied, the structure disappears into background infrastructure.

\bibliographystyle{unsrt}
\bibliography{references}

\end{document}
