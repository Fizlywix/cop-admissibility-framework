\documentclass[11pt]{article}

\usepackage[T1]{fontenc}
\usepackage[utf8]{inputenc}
\usepackage{lmodern}
\usepackage{microtype}
\usepackage{amsmath,amssymb,amsthm}
\usepackage{graphicx}
\usepackage{hyperref}
\usepackage{url}
\usepackage{booktabs}
\usepackage{enumitem}

\usepackage{../../../templates/cop/cop-macros}
\usepackage{../../../templates/cop/cop-diagrams}

\title{L6: The Periodic Table Is Not About Elements}
\author{Pascal Sparidaens}
\date{}

\begin{document}
\maketitle

\begin{abstract}
This paper clarifies the status of the 2$\times$4$\times$8 table by removing a persistent
misinterpretation: that the table enumerates entities, components, or system parts.
Instead, the table is a pruning snapshot of survivability.
It records which interaction patterns remain admissible under constraint, and which
collapse or disappear.
\end{abstract}

\section{What the Table Is Not}

The 2$\times$4$\times$8 table is not a catalogue of elements, modules, or primitives.
It does not list system components, behaviors, or semantic categories.

Interpreting table entries as “things” introduces illegitimate reconstruction and
violates the reading discipline established in
\emph{L3: Reading Without Reconstruction}.

Nothing in the table denotes ontology.

\section{Why the Periodic Table Analogy Exists}

The chemical periodic table did not succeed because it listed everything that exists.
It succeeded because it revealed which configurations are stable enough to persist.

Most atomic configurations never appear.
They decay too quickly, fail to bind, or collapse immediately.

The table is therefore a pruning snapshot of survivability, not an inventory.

The 2$\times$4$\times$8 table plays the same role for system interactions.

\section{Stability as the Primary Criterion}

In chemistry, stability is governed by electron configuration.
In systems, stability is governed by admissibility under constraint.

An interaction appears in the table not because it is meaningful,
but because it can survive repeated closure, propagation, and reverse pruning.

Visibility is a side-effect.
Persistence is the criterion.

\section{Why Most of the Table Is Dark}

Most admissible interactions are invisible while constraints are satisfied.
This is a direct consequence of null dominance.

Latent states do not appear.
Infrastructure disappears once stable.
Reverse structure becomes legible only at violation or closure.

This asymmetry explains why failure dominates attention despite being rare,
and why debugging is harder than construction.

\section{Stable Regions, Not Cells}

A “stable element” in the table is not a single cell.
It is a region of trajectories that repeatedly:
\begin{itemize}
  \item close cleanly,
  \item infer invariants,
  \item background into infrastructure,
  \item and disappear without forcing reset.
\end{itemize}

Stability is therefore path-dependent, not locational.

\section{The Seam and the Closed Loop}

Two special operators structure the entire table:
the admission seam and the return seam.

Every stable system trajectory:
emerges,
commits,
propagates,
saturates,
is read backward,
collapses,
and makes emergence possible again.

This loop is structural and unavoidable.
It is not a narrative.

\section{Consequences for Interpretation}

Treating the table as an inventory produces predictable errors:
\begin{itemize}
  \item overinterpretation of visible events,
  \item neglect of infrastructure,
  \item confusion between survivability and importance.
\end{itemize}

Treating the table as a survivability map avoids these errors.

\section{Relation to the COP Corpus}

This paper reframes the table introduced in
\emph{L2: A 2$\times$4$\times$8 Admissibility Framework}
and clarifies its use in
\emph{L5: Operational Admissibility in a Bounded Structure}.

Diagrammatic grammar remains the domain of
\emph{L4: Diagrammatic Addressing in the COP Framework}.

\section{Conclusion}

The periodic table is not about elements.
It is about what survives.

The 2$\times$4$\times$8 table applies the same logic to system interactions.
It records survivability under constraint, not meaning, mechanism, or intent.

\bibliographystyle{unsrt}
\bibliography{references}

\end{document}
