\documentclass[11pt]{article}

\usepackage[T1]{fontenc}
\usepackage[utf8]{inputenc}
\usepackage{lmodern}
\usepackage{microtype}
\usepackage{amsmath,amssymb,amsthm}
\usepackage{graphicx}
\usepackage{hyperref}
\usepackage{url}
\usepackage{booktabs}
\usepackage{enumitem}

\usepackage{../../../templates/cop/cop-macros}
\usepackage{../../../templates/cop/cop-diagrams}

\title{L5: Operational Admissibility in a Bounded Structure}
\author{Pascal Sparidaens}
\date{}

\begin{document}
\maketitle

\begin{abstract}
This paper provides an operational example of the 2$\times$4$\times$8 admissibility framework.
The example demonstrates how admissibility constraints behave when applied to a bounded structure,
without reconstructing domain semantics or introducing interpretive narratives.
\end{abstract}

\section{Introduction}

This paper continues the development of the COP framework by demonstrating how admissibility
operates on a constrained structure.

It assumes familiarity with the admissibility rules established in
\emph{L2: A 2$\times$4$\times$8 Admissibility Framework} and the reading discipline specified in
\emph{L3: Reading Without Reconstruction}.

No claims are made regarding physical causation, ontology, or reinterpretation of existing
scientific frameworks.
The focus is exclusively on operational behavior under constraint.

\section{Admissibility Constraints}

Admissibility is defined as the set of operations that remain coherent under structural constraints.
Forward and reverse constraints may operate concurrently, and no operation permits reconstruction
of excluded histories.

The framework assumes:
\begin{itemize}
  \item Writes are additive.
  \item No operation deletes history.
  \item Return is structural, not causal.
\end{itemize}

These assumptions are not derived.
They are enforced.

\section{Operational Example: A Bounded Structure}

This section provides a concrete operational example of the 2$\times$4$\times$8 admissibility framework.
The purpose is not to explain a physical system, nor to reinterpret an established domain,
but to demonstrate how admissibility constraints behave when applied to a familiar, finite structure.

The example is intentionally chosen to be recognizable while remaining semantically neutral.
The structure serves only as a constraint surface on which admissibility can be exercised.

\subsection{The Host Structure as a Constraint Surface}

Consider a bounded, finite structure with a fixed number of admissible positions and well-defined
adjacency relations.

Such structures are common across domains and include tabular lattices, phase diagrams,
and other discretized representations.
No semantic interpretation of positions is required.

The only properties assumed are:
\begin{itemize}
  \item The structure is finite.
  \item Adjacency relations are constrained.
  \item Certain extensions are impossible by construction.
  \item Saturation occurs at the boundaries.
\end{itemize}

These properties are sufficient to host admissibility operations.

\subsection{Forward Restriction and Saturation}

We begin with a forward admissibility operation (D = 1).
A forward operation corresponds to an irreversible restriction of future admissible states.

Repeated forward restriction remains admissible only while the structure can accommodate
further differentiation without contradiction.
Because the structure is finite, a point is inevitably reached at which no further forward
restriction is admissible.

Saturation is not an event.
It does not correspond to a detectable transition.
It is a structural condition that becomes legible only when an additional forward move is attempted
and excluded.

\subsection{Concurrent Reverse Exclusion}

While forward saturation occurs locally, reverse admissibility (D = 2) may already be active
elsewhere.

Reverse operations do not reconstruct prior actions.
They operate solely by excluding histories that are incompatible with present constraints.

At saturation, multiple forward trajectories may remain admissible,
but certain past interpretations become impossible.
Forward and reverse constraints therefore operate concurrently,
without implying reversibility or symmetry.

\subsection{Return as Structural Necessity}

When forward restriction can no longer proceed and reverse exclusion has eliminated all remaining
incompatible histories, the system reaches a point at which no further distinction is admissible.

At this point, the only admissible operation is return.
Return does not correspond to a privileged state, transition, or reset.
It is the dissolution of distinctions that can no longer be maintained coherently.

Importantly, return is not caused by saturation.
It is revealed by it.

\subsection{Boundary of Interpretation}

This example does not propose a new interpretation of any physical, chemical, or mathematical system.
Any such interpretation would constitute an illegal reconstruction and falls outside the
admissibility of the framework.

The example is complete insofar as it demonstrates how admissibility behaves when applied to
a bounded structure.
That suffices.

\section{Conclusion}

This paper has demonstrated how admissibility constraints operate on a bounded structure
without introducing interpretive narratives or reconstructive explanations.

The example prepares the ground for subsequent diagrammatic and applied treatments of the framework,
while remaining fully non-interpretive.

\bibliographystyle{unsrt}
\bibliography{references}

\end{document}
