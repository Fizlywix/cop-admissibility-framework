\documentclass[11pt]{article}

\usepackage[T1]{fontenc}
\usepackage[utf8]{inputenc}
\usepackage{lmodern}
\usepackage{microtype}
\usepackage{amsmath,amssymb,amsthm}
\usepackage{graphicx}
\usepackage{hyperref}
\usepackage{url}
\usepackage{booktabs}
\usepackage{enumitem}

\usepackage{../../../templates/cop/cop-macros}
\usepackage{../../../templates/cop/cop-diagrams}

\title{L1: How to Read the COP Corpus}
\author{Pascal Sparidaens}
\date{}

\begin{document}
\maketitle

\begin{abstract}
This paper provides a reading guide for the COP corpus.
It introduces no new structure, notation, or claims.
Its sole purpose is to specify how the remaining papers (L2--L8) are to be read,
and to prevent illegitimate inference across the corpus.
\end{abstract}

\section{Purpose}

This paper is not a theoretical contribution.
It is an interface.

The COP corpus is intentionally non-linear.
Different papers serve different constraint roles, and cannot be read as a
single progressive argument.

This paper specifies:
\begin{itemize}
  \item what each paper does,
  \item what it does not do,
  \item and how the papers relate without collapsing into a narrative.
\end{itemize}

No part of the corpus requires belief, agreement, or interpretation.

\section{What the COP Corpus Is}

The COP corpus defines an admissibility framework.
It constrains which operations remain coherent under structural limits.

The corpus does not:
\begin{itemize}
  \item explain systems,
  \item predict outcomes,
  \item reconstruct histories,
  \item or provide ontological commitments.
\end{itemize}

Correct use consists in excluding illegitimate moves.

\section{The Role of the Light Papers}

The Light papers (L1--L8) define the usable surface of the framework.

Each paper has a single function.
No paper subsumes another.

\subsection{L2: A 2$\times$4$\times$8 Admissibility Framework}

L2 defines the coordinate system.
It introduces the axes, roles, and admissibility rules.

It contains no guidance on interpretation or use.
All subsequent papers depend on it structurally.

\subsection{L3: Reading Without Reconstruction}

L3 constrains how the framework may be read.
It catalogues illegal inferences and forbids narrative reconstruction.

Any reading discipline not consistent with L3 is invalid.

\subsection{L4: Diagrammatic Addressing in the COP Framework}

L4 defines the diagrammatic grammar.
It specifies how diagrams function as constraint objects.

The figures are not illustrative.
They do not explain the framework.
They are governed by admissibility rules.

\subsection{L5: Operational Admissibility in a Bounded Structure}

L5 demonstrates how admissibility behaves under constraint.
It provides an operational example without interpretation.

The example is a host surface, not a model.

\subsection{L6: The Periodic Table Is Not About Elements}

L6 removes a persistent misreading:
that the table enumerates entities or components.

The table records survivability under constraint,
not meaning, mechanism, or ontology.

\subsection{L7: Structural Visibility Under Stress}

L7 explains why failure is visible and stability is not.
Visibility is treated as a consequence of constraint pressure,
not as a feature of observers or narratives.

\subsection{L8: Applied Constraint Surfaces}

L8 specifies how the framework may be used without becoming explanatory.
It treats admissibility as a surface for disciplined operation.

No examples are given.
Stopping conditions are explicit.

\section{Reading Order}

The corpus does not require a single reading order.
However, the following constraints apply:
\begin{itemize}
  \item L2 must be read before any other paper.
  \item L3 must be read before attempting interpretation.
  \item L4 must be read before using diagrams.
  \item L5--L8 may be read in any order after L2--L4.
\end{itemize}

Violating these constraints produces predictable misreadings.

\section{What Is Deliberately Missing}

The corpus does not include:
\begin{itemize}
  \item worked examples,
  \item applications to specific domains,
  \item comparisons to existing theories,
  \item or claims of explanatory power.
\end{itemize}

These omissions are intentional.
They preserve admissibility discipline.

\section{Shadow Papers}

The Shadow corpus (S1--S8) documents paradoxes, limits, and non-totality.
Shadow papers do not extend the framework.
They mark where admissibility necessarily breaks down.

The Shadow corpus is not required for use,
but prevents overextension.

\section{Conclusion}

The COP corpus is not a theory to be learned.
It is a discipline to be followed.

Correct reading excludes more than it includes.
When used correctly, the framework disappears into constraint,
leaving only illegitimate moves ruled out.

\bibliographystyle{unsrt}
\bibliography{references}

\end{document}
