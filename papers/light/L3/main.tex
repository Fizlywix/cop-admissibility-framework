\documentclass[11pt]{article}

\usepackage[T1]{fontenc}
\usepackage[utf8]{inputenc}
\usepackage{lmodern}
\usepackage{microtype}
\usepackage{amsmath,amssymb,amsthm}
\usepackage{graphicx}
\usepackage{hyperref}
\usepackage{url}
\usepackage{booktabs}
\usepackage{enumitem}

\usepackage{../../../templates/cop/cop-macros}
\usepackage{../../../templates/cop/cop-diagrams}

\title{L3: Reading Without Reconstruction}
\author{Pascal Sparidaens}
\date{}

\begin{document}
\maketitle

\begin{abstract}
Formal frameworks are most often misused not at the level of definition,
but at the level of reading.
This paper specifies a discipline of inference for the 2$\times$4$\times$8
admissibility framework introduced in \emph{L2: A 2$\times$4$\times$8 Admissibility Framework}.
Rather than presenting applications or outcomes, it catalogues illegal
inferences and shows how the framework explicitly forbids them.
The aim is not interpretation, but the prevention of false certainty.
\end{abstract}

\section{Scope and Relation to the Corpus}

This paper is not a standalone theory of interpretation.
It is a methodological companion to
\emph{L2: A 2$\times$4$\times$8 Admissibility Framework}.

It introduces no new notation, roles, or extensions.
Its sole purpose is to constrain how the existing framework may be read,
and to prevent reconstruction errors once the coordinate system is fixed.

Any reading of this paper outside that context is out of scope.

\section{The Source of Error After Formalization}

Most failures of formal reasoning occur after a framework is accepted.
They arise when readers import narrative, intent, causality, or completeness
that the structure does not license.

Once a coordinate system is fixed, misuse no longer takes the form of
symbolic error, but of illegitimate inference.
This paper addresses that failure mode directly.

\section{Forward Writing and Backward Exclusion}

In the framework, forward operations restrict future admissibility.
Backward reasoning does not recover actions; it excludes incompatible histories.

From a final or closed state, one may infer:
\begin{itemize}
  \item what could not have occurred,
  \item which invariants must have held,
  \item which possibilities are excluded.
\end{itemize}

One may not infer:
\begin{itemize}
  \item a unique causal chain,
  \item intent or motive,
  \item intermediate trajectories.
\end{itemize}

Attempting to do so constitutes an illegal reconstruction.

\section{The Error of Narrative Completion}

A common failure mode is narrative completion:
the impulse to fill structural gaps with plausible stories.

The framework explicitly forbids this.
Multiple histories may remain admissible indefinitely.
Ambiguity is not a flaw to be resolved, but a constraint to be respected.

\section{Closure Without Event}

Another frequent misreading is the assumption that closure requires a
visible action.

Within the framework, closure may occur through the exhaustion of
alternatives rather than through an event.
Such closures are often legible only in reverse, and only as invariant
recognition.

\section{Where Reading Must Stop}

Certain coordinates permit exclusion without permitting positive
reconstruction.
When a region of the table yields only impossibility pruning,
reasoning must stop.

Continuing beyond this point introduces unlicensed certainty
and violates admissibility.

\section{Conclusion}

The 2$\times$4$\times$8 framework is not a tool for explanation.
It is a discipline for preventing epistemic overreach.

Correct use consists not in what the framework allows one to say,
but in what it forces one to leave unsaid.

\bibliographystyle{unsrt}
\bibliography{references}

\end{document}
