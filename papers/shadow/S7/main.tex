\documentclass[11pt]{article}

\usepackage[T1]{fontenc}
\usepackage[utf8]{inputenc}
\usepackage{lmodern}
\usepackage{microtype}
\usepackage{amsmath,amssymb,amsthm}
\usepackage{graphicx}
\usepackage{hyperref}
\usepackage{url}
\usepackage{booktabs}
\usepackage{enumitem}

\usepackage{../../../templates/cop/cop-macros}
\usepackage{../../../templates/cop/cop-diagrams}

\title{S7: Coordinated Multiplicity and Limitation}
\author{Pascal Sparidaens}
\date{}

\begin{document}
\maketitle

\begin{abstract}
This Shadow paper documents a single admissibility boundary:
the tension between unity as coordinated multiplicity (forward admissibility)
and the requirement that healthy systems include limiters (reverse legibility).
It introduces no new structure, notation, or claims.
Its sole purpose is to prevent misreadings in which coordination is treated as total integration,
or limitation is treated as failure rather than necessity.
\end{abstract}

\section{Purpose}

This paper is not a contribution.
It is a constraint marker.

S7 binds one Shadow axis:
\begin{itemize}
  \item \textbf{Forward constraint:} COP-12 (Unity Is Coordinated Multiplicity),
  \item \textbf{Reverse constraint:} COP-14 (Healthy Systems Require Limiters).
\end{itemize}

S7 exists to exclude illegitimate moves that arise when unity is pushed toward total coherence,
or when limits are interpreted as defects to be eliminated.

\section{What This Paper Is Not}

This paper does not:
\begin{itemize}
  \item define optimal coordination schemes,
  \item prescribe integration architectures,
  \item treat scalability as an unqualified good,
  \item justify limit removal,
  \item or equate growth with health.
\end{itemize}

Any reading that turns S7 into a theory of optimization or integration is invalid.

\section{Shadow Axis S7}

\subsection{Forward: COP-12 Unity Is Coordinated Multiplicity}

Coordinated multiplicity constrains forward admissibility:
unity arises from coordination among distinct parts,
not from their fusion or homogenization.

Operationally:
\begin{itemize}
  \item multiplicity is preserved,
  \item coordination is selective,
  \item unity is an effect, not a prerequisite.
\end{itemize}

Total integration is not admissible.
Coordination always leaves remainder.

\subsection{Reverse: COP-14 Healthy Systems Require Limiters}

COP-14 constrains reverse legibility:
limits are structural requirements, not failures of design.

Operationally:
\begin{itemize}
  \item constraints protect viability,
  \item saturation signals health boundaries,
  \item unchecked expansion degrades coherence.
\end{itemize}

Reverse reading may detect limit activation,
but may not interpret it as malfunction.

\subsection{The Boundary}

The boundary enforced by S7 is the following:

\begin{quote}
\textbf{Unity does not require total integration.} \\
\textbf{Limits are conditions of health, not obstacles to remove.}
\end{quote}

Treating coordination as fusion destroys multiplicity.
Treating limits as failures invites systemic collapse.

\section{Local \texttt{+} and Global Constraint}

\subsection{Local phenomenology}

Forward operation proceeds only along \texttt{+} branches:
what manifests is partial coordination among distinct components.
Local experience registers cooperation with friction, not seamless unity.

\subsection{Global diagnosability}

Global boundary violations become detectable as:
\begin{itemize}
  \item pressure toward full integration,
  \item erosion of local autonomy,
  \item runaway scaling demands,
  \item suppression of limiting signals,
  \item systemic brittleness following expansion.
\end{itemize}

These are not local inefficiencies.
They are boundary warnings.

\section{Illegitimate Moves (Forbidden Inferences)}

S7 forbids the following moves:

\begin{enumerate}
  \item \textbf{Total integration:}
  treating unity as requiring complete coherence or fusion.
  \item \textbf{Limit removal:}
  interpreting constraints as design flaws.
  \item \textbf{Scale absolutism:}
  assuming larger coordination is inherently superior.
  \item \textbf{Remainder denial:}
  attempting to eliminate uncoordinated residue.
  \item \textbf{Health misreading:}
  treating saturation or slowdown as failure.
\end{enumerate}

These moves convert coordinated multiplicity
into fragile over-integration.

\section{Detection Signals}

S7 becomes visible under stress through:

\begin{itemize}
  \item increasing integration overhead,
  \item loss of subsystem independence,
  \item resistance to throttling or caps,
  \item collapse following rapid expansion,
  \item narratives of “one system to rule them all.”
\end{itemize}

These signals mark boundary pressure, not lack of coordination.

\section{Stop Conditions}

Correct use stops at the boundary.

\begin{itemize}
  \item If unity is pushed toward total integration, stop (COP-12).
  \item If limits are treated as defects, stop (COP-14).
  \item If expansion overrides viability checks, stop.
\end{itemize}

S7 does not optimize coordination.
It prevents illegitimate over-integration.

\section{Figures (Constraint Objects)}

Figures in Shadow papers are constraint markers only.
They do not explain or illustrate processes.

\begin{figure}[h]
  \centering
  % Placeholder for coordinated multiplicity overlay
  \includegraphics[width=0.92\linewidth]{figures/fig3-return-exposed_S7-overlay.pdf}
  \caption{S7 constraint overlay. Coordinated multiplicity preserves distinct components while preventing total integration beyond limits.}
  \label{fig:s7-overlay}
\end{figure}

\begin{figure}[h]
  \centering
  % Placeholder for limiter activation slice
  \includegraphics[width=0.72\linewidth]{figures/state-cube_S7-slice.pdf}
  \caption{Optional constraint slice. Limiters activate to preserve system health; they are not indicators of failure.}
  \label{fig:s7-cube}
\end{figure}

\section{Conclusion}

S7 exists to enforce a single discipline:
coordinate without collapsing multiplicity,
and respect limits as conditions of health.

Nothing further is added.

\bibliographystyle{unsrt}
\bibliography{references}

\end{document}

