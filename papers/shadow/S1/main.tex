\documentclass[11pt]{article}

\usepackage[T1]{fontenc}
\usepackage[utf8]{inputenc}
\usepackage{lmodern}
\usepackage{microtype}
\usepackage{amsmath,amssymb,amsthm}
\usepackage{graphicx}
\usepackage{hyperref}
\usepackage{url}
\usepackage{booktabs}
\usepackage{enumitem}

\usepackage{../../../templates/cop/cop-macros}
\usepackage{../../../templates/cop/cop-diagrams}

\title{S1: Survivability and Residue}
\author{Pascal Sparidaens}
\date{}

\begin{document}
\maketitle

\begin{abstract}
This Shadow paper documents a single admissibility boundary:
the tension between survivability (forward continuity) and residue (reverse legibility).
It introduces no new structure, notation, or claims.
Its role is to prevent overextension: attempts to preserve everything, explain everything,
or reconstruct what cannot be reconstructed.
\end{abstract}

\section{Purpose}

This paper is not a theoretical contribution.
It is a boundary marker.

S1 binds a single Shadow axis:
\begin{itemize}
  \item \textbf{Forward constraint:} COP-01 (Survivability),
  \item \textbf{Reverse constraint:} COP-02 (Residue Primacy).
\end{itemize}

S1 does not add content.
It only excludes illegitimate moves that appear when survivability is misread as preservation,
or when residue is misread as a recoverable record.

\section{What This Paper Is Not}

This paper does not:
\begin{itemize}
  \item provide a model of systems,
  \item define new operations,
  \item recommend policies,
  \item explain failures by narrative,
  \item or license reconstruction of history.
\end{itemize}

Any reading that turns S1 into explanation is invalid by design.

\section{Shadow Axis S1}

\subsection{Forward: COP-01 Survivability}

Survivability constrains forward admissibility:
local damage may occur, but cascades must be prevented.
Survivability is not optimization.
It is the admissibility condition that permits continued operation under bounded stress.

Operationally:
\begin{itemize}
  \item local failure is admissible,
  \item propagation without routing is not,
  \item containment is a primitive (not an outcome).
\end{itemize}

\subsection{Reverse: COP-02 Residue Primacy}

Residue Primacy constrains reverse legibility:
events do not remain available as objects.
Only residue (what constrains future admissibility) may support inference.

Operationally:
\begin{itemize}
  \item witnesses are not a truth substrate,
  \item replay is not licensed,
  \item ``what happened'' is not admissible as a target object.
\end{itemize}

\subsection{The Boundary}

The boundary is simple:

\begin{quote}
\textbf{Survivability requires loss.}
\textbf{Residue requires irreversibility.}
\end{quote}

Attempts to eliminate loss eliminate survivability.
Attempts to eliminate irreversibility eliminate residue and replace it with narrative.

S1 exists to mark that these failures are not mistakes.
They are structural.

\section{Local \texttt{+} and Global \texttt{---}}

This corpus is used forward under visibility constraints.

\subsection{Local phenomenology}

Forward use is experienced only on \texttt{+} branches:
what manifests, binds, writes, or propagates is what is locally observable.
The \texttt{-} branch is structurally active (it constrains), but is not locally experienced as content.

\subsection{Global diagnosability}

The fully blocked configuration \texttt{---} is not a locally traversable state.
However, it can be globally detectable as systemwide non-response:
a collapse of admissible continuation, not an experienced event.

S1 forbids treating global \texttt{---} as an object of reconstruction.
It is a diagnosable boundary condition, not a narrated endpoint.

\section{Illegitimate Moves (Forbidden Inferences)}

S1 forbids the following moves:

\begin{enumerate}
  \item \textbf{Preservation as survivability:}
  treating survivability as a requirement to save all local states.
  \item \textbf{Completeness as safety:}
  assuming safety increases monotonically with total observability.
  \item \textbf{History as retrievable:}
  assuming that because residue exists, events remain recoverable.
  \item \textbf{Witness elevation:}
  treating testimony, intent, or narrative as a primary substrate.
  \item \textbf{Loss denial:}
  treating loss as an error to be corrected rather than a required constraint.
  \item \textbf{Reverse made forward:}
  using residue-based inference as an operational control policy.
\end{enumerate}

Any system or reader attempting these moves will converge to fragility:
either cascade (COP-01 violation) or hallucinated reconstruction (COP-02 violation).

\section{Detection Signals}

S1 becomes visible only under stress.
Typical signals include:

\begin{itemize}
  \item \textbf{Cascade sensitivity:} small local failures trigger global degradation.
  \item \textbf{Archive inflation:} retention, logging, and tracing grow without increasing stability.
  \item \textbf{Replay demand:} insistence on recovering ``what really happened''.
  \item \textbf{Narrative substitution:} explanation replaces constraint-checking.
  \item \textbf{Global silence:} apparent systemwide non-response (\texttt{---}) becomes detectable,
        followed by attempts to narrate it.
\end{itemize}

These are not anomalies.
They are boundary indicators.

\section{Stop Conditions}

Correct use stops at the boundary.

\begin{itemize}
  \item If the next move requires reconstructing events, stop (COP-02).
  \item If the next move requires preserving everything, stop (COP-01).
  \item If global \texttt{---} is detected, stop: diagnose constraints; do not narrate.
\end{itemize}

S1 does not instruct repair.
It prevents illegitimate expansion.

\section{Figures (Constraint Objects)}

Figures in Shadow papers are not explanatory.
They mark admissibility boundaries.

\begin{figure}[h]
  \centering
  % Replace this with your diagram macro or generated overlay asset.
  \includegraphics[width=0.92\linewidth]{figures/fig3-return-exposed_S1-overlay.pdf}
  \caption{S1 overlay on the return-exposed addressing figure. The highlighted transition marks the survivability gate (forward) and the corresponding residue constraint (reverse).}
  \label{fig:s1-overlay}
\end{figure}

\begin{figure}[h]
  \centering
  % Optional: a minimal state-cube slice or axis marker. Keep it non-explanatory.
  \includegraphics[width=0.72\linewidth]{figures/state-cube_S1-slice.pdf}
  \caption{Optional constraint slice: local \texttt{+} phenomenology versus global diagnosability. This is a boundary marker, not a process diagram.}
  \label{fig:s1-cube}
\end{figure}

\section{Conclusion}

S1 exists to enforce a single discipline:
survivability is obtained by permitting loss,
and residue is obtained by forbidding reconstruction.

Nothing further is added.

\bibliographystyle{unsrt}
\bibliography{references}

\end{document}
