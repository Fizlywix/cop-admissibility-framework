\documentclass[11pt]{article}

\usepackage[T1]{fontenc}
\usepackage[utf8]{inputenc}
\usepackage{lmodern}
\usepackage{microtype}
\usepackage{amsmath,amssymb,amsthm}
\usepackage{graphicx}
\usepackage{hyperref}
\usepackage{url}
\usepackage{booktabs}
\usepackage{enumitem}

\usepackage{../../../templates/cop/cop-macros}
\usepackage{../../../templates/cop/cop-diagrams}

\title{S6: Commitment and Action}
\author{Pascal Sparidaens}
\date{}

\begin{document}
\maketitle

\begin{abstract}
This Shadow paper documents a single admissibility boundary:
the tension between commitment as a forward stabilizing act
and the constraint that not all actions exist as write-capable operations.
It introduces no new structure, notation, or claims.
Its sole purpose is to prevent misreadings in which agreement substitutes for commitment,
or intention substitutes for action.
\end{abstract}

\section{Purpose}

This paper is not a contribution.
It is a constraint marker.

S6 binds one Shadow axis:
\begin{itemize}
  \item \textbf{Forward constraint:} COP-08 (Commitment Replaces Consensus),
  \item \textbf{Reverse constraint:} COP-10 (Not All Actions Exist).
\end{itemize}

S6 exists to exclude illegitimate moves that arise when commitment is treated
as collective agreement, or when intention is treated as an action.

\section{What This Paper Is Not}

This paper does not:
\begin{itemize}
  \item prescribe decision procedures,
  \item define legitimate authority,
  \item enumerate possible actions,
  \item evaluate intentions,
  \item or equate consent with effect.
\end{itemize}

Any reading that turns S6 into a theory of decision-making or agency is invalid.

\section{Shadow Axis S6}

\subsection{Forward: COP-08 Commitment Replaces Consensus}

Commitment constrains forward admissibility:
stability is achieved by binding acts, not by shared agreement.

Operationally:
\begin{itemize}
  \item commitment precedes understanding,
  \item disagreement does not block operation,
  \item binding reduces option space.
\end{itemize}

Consensus is not required for continuation.
Commitment is sufficient.

\subsection{Reverse: COP-10 Not All Actions Exist}

COP-10 constrains reverse legibility:
only some actions produce effects.
Others do not exist as write-capable operations.

Operationally:
\begin{itemize}
  \item intention is not action,
  \item expression is not execution,
  \item permission does not imply effect.
\end{itemize}

Reverse reading may detect failed attempts,
but may not treat them as actions.

\subsection{The Boundary}

The boundary enforced by S6 is the following:

\begin{quote}
\textbf{Commitment is not agreement.} \\
\textbf{Action is not intention.}
\end{quote}

Treating consensus as commitment delays binding.
Treating intention as action inflates agency.

\section{Local \texttt{+} and Global Constraint}

\subsection{Local phenomenology}

Forward operation proceeds only along \texttt{+} branches:
what manifests are binding commitments that reduce future options.
Local experience registers commitment as constraint, not consensus.

\subsection{Global diagnosability}

Global boundary violations become detectable as:
\begin{itemize}
  \item endless deliberation without binding,
  \item proliferation of declared but ineffective actions,
  \item symbolic gestures replacing operational change,
  \item instability masked as participation.
\end{itemize}

These are not local failures.
They are signs that admissibility has been exceeded.

\section{Illegitimate Moves (Forbidden Inferences)}

S6 forbids the following moves:

\begin{enumerate}
  \item \textbf{Consensus substitution:}
  treating agreement as equivalent to commitment.
  \item \textbf{Intent inflation:}
  treating declared intent as executed action.
  \item \textbf{Action overcounting:}
  counting non-binding acts as operational changes.
  \item \textbf{Permission fallacy:}
  assuming authorization guarantees effect.
  \item \textbf{Participation masking:}
  using involvement to conceal lack of commitment.
\end{enumerate}

These moves replace binding with appearance.

\section{Detection Signals}

S6 becomes visible under stress through:

\begin{itemize}
  \item prolonged decision processes,
  \item accumulation of non-binding resolutions,
  \item repeated “attempts” without effect,
  \item emphasis on voice over consequence,
  \item erosion of operational clarity.
\end{itemize}

These signals mark boundary pressure, not democratic deficit.

\section{Stop Conditions}

Correct use stops at the boundary.

\begin{itemize}
  \item If agreement substitutes for commitment, stop (COP-08).
  \item If intention substitutes for action, stop (COP-10).
  \item If binding is postponed indefinitely, stop.
\end{itemize}

S6 does not define action.
It prevents illegitimate substitution.

\section{Figures (Constraint Objects)}

Figures in Shadow papers are constraint markers only.
They do not explain or illustrate processes.

\begin{figure}[h]
  \centering
  % Placeholder for commitment overlay
  \includegraphics[width=0.92\linewidth]{figures/fig3-return-exposed_S6-overlay.pdf}
  \caption{S6 constraint overlay. Binding commitments reduce option space, while non-existent actions are excluded under reverse reading.}
  \label{fig:s6-overlay}
\end{figure}

\begin{figure}[h]
  \centering
  % Placeholder for action-existence slice
  \includegraphics[width=0.72\linewidth]{figures/state-cube_S6-slice.pdf}
  \caption{Optional constraint slice. Only write-capable actions exist operationally; intention alone leaves no trace.}
  \label{fig:s6-cube}
\end{figure}

\section{Conclusion}

S6 exists to enforce a single discipline:
bind before agreement,
and count only actions that exist.

Nothing further is added.

\bibliographystyle{unsrt}
\bibliography{references}

\end{document}

