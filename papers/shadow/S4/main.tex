\documentclass[11pt]{article}

\usepackage[T1]{fontenc}
\usepackage[utf8]{inputenc}
\usepackage{lmodern}
\usepackage{microtype}
\usepackage{amsmath,amssymb,amsthm}
\usepackage{graphicx}
\usepackage{hyperref}
\usepackage{url}
\usepackage{booktabs}
\usepackage{enumitem}

\usepackage{../../../templates/cop/cop-macros}
\usepackage{../../../templates/cop/cop-diagrams}

\title{S4: Stability Bands and Responsibility}
\author{Pascal Sparidaens}
\date{}

\begin{document}
\maketitle

\begin{abstract}
This Shadow paper documents a single admissibility boundary:
the tension between operation within stability bands (forward admissibility)
and network-routed responsibility (reverse legibility).
It introduces no new structure, notation, or claims.
Its sole purpose is to prevent misreadings in which stability is treated as control,
or responsibility is treated as locally assignable.
\end{abstract}

\section{Purpose}

This paper is not a contribution.
It is a constraint marker.

S4 binds one Shadow axis:
\begin{itemize}
  \item \textbf{Forward constraint:} COP-04 (Stability Band),
  \item \textbf{Reverse constraint:} ICOP-04 (Responsibility Must Be Network-Routed).
\end{itemize}

S4 exists to exclude illegitimate moves that arise when stability is confused with dominance,
or when responsibility is collapsed onto individual nodes.

\section{What This Paper Is Not}

This paper does not:
\begin{itemize}
  \item prescribe optimal operating points,
  \item define safe equilibria,
  \item allocate blame or credit,
  \item localize responsibility,
  \item or justify intervention by authority.
\end{itemize}

Any reading that turns S4 into a governance, control, or accountability model is invalid.

\section{Shadow Axis S4}

\subsection{Forward: COP-04 Stability Band}

The stability band constrains forward admissibility:
operation is viable only within bounded ranges.
Outside the band, admissibility collapses.

Operationally:
\begin{itemize}
  \item stability is range-bound, not point-defined,
  \item remaining within bounds matters more than optimization,
  \item pushing limits reduces global survivability.
\end{itemize}

Stability does not imply control.
It indicates tolerated variance.

\subsection{Reverse: ICOP-04 Network-Routed Responsibility}

Network-routed responsibility constrains reverse legibility:
effects propagate through networks, not individuals.
Responsibility follows propagation paths, not intent or position.

Operationally:
\begin{itemize}
  \item outcomes are network-produced,
  \item attribution is distributed,
  \item local blame is structurally misleading.
\end{itemize}

Reverse inference may trace paths,
but may not collapse responsibility onto single actors.

\subsection{The Boundary}

The boundary enforced by S4 is the following:

\begin{quote}
\textbf{Stability is not authority.} \\
\textbf{Responsibility is not local.}
\end{quote}

Treating stability as control licenses overreach.
Treating responsibility as local obscures systemic causation.

\section{Local \texttt{+} and Global Constraint}

\subsection{Local phenomenology}

Forward operation proceeds only along \texttt{+} branches:
what manifests is continued operation within acceptable bounds.
Local actors experience margins, not total system behavior.

\subsection{Non-local diagnosability}

Boundary violations are not locally attributable.
They become detectable only non-locally as cascading effects,
diffuse harm, or pressure toward centralized control.

Such signals are diagnostic only.
They must not be moralized or personalized.

\section{Illegitimate Moves (Forbidden Inferences)}

S4 forbids the following moves:

\begin{enumerate}
  \item \textbf{Control inference:}
  treating stable operation as evidence of dominance or mastery.
  \item \textbf{Blame localization:}
  assigning responsibility to individual nodes for network effects.
  \item \textbf{Optimization pressure:}
  pushing systems to band edges for performance gain.
  \item \textbf{Authority justification:}
  using stability metrics to legitimize intervention.
  \item \textbf{Network erasure:}
  ignoring propagation paths in favor of personal culpability.
\end{enumerate}

These moves collapse bounded operation into illegitimate governance.

\section{Detection Signals}

S4 becomes visible under stress through:

\begin{itemize}
  \item tightening tolerances despite rising impact,
  \item repeated near-boundary excursions,
  \item diffuse harm with singular blame narratives,
  \item centralization in response to instability,
  \item moralization replacing structural analysis.
\end{itemize}

These signals mark boundary pressure, not moral failure.

\section{Stop Conditions}

Correct use stops at the boundary.

\begin{itemize}
  \item If stability is treated as control, stop (COP-04).
  \item If responsibility is localized, stop (ICOP-04).
  \item If intervention replaces constraint analysis, stop.
\end{itemize}

S4 does not assign responsibility.
It prevents illegitimate attribution.

\section{Figures (Constraint Objects)}

Figures in Shadow papers are constraint markers only.
They do not explain or illustrate processes.

\begin{figure}[h]
  \centering
  \includegraphics[width=0.92\linewidth]{figures/fig3-return-exposed_S4-overlay.pdf}
  \caption{S4 constraint overlay. The highlighted region marks admissible operation within stability bands and the distributed nature of responsibility under reverse reading.}
  \label{fig:s4-overlay}
\end{figure}

\begin{figure}[h]
  \centering
  \includegraphics[width=0.72\linewidth]{figures/state-cube_S4-slice.pdf}
  \caption{Optional constraint slice. Stability bands constrain operation; responsibility propagates across the network. This figure marks the boundary only.}
  \label{fig:s4-cube}
\end{figure}

\section{Conclusion}

S4 exists to enforce a single discipline:
operate within stability bands,
and route responsibility through networks rather than nodes.

Nothing further is added.

\bibliographystyle{unsrt}
\bibliography{references}

\end{document}
