\documentclass[11pt]{article}

\usepackage[T1]{fontenc}
\usepackage[utf8]{inputenc}
\usepackage{lmodern}
\usepackage{microtype}
\usepackage{amsmath,amssymb,amsthm}
\usepackage{graphicx}
\usepackage{hyperref}
\usepackage{url}
\usepackage{booktabs}
\usepackage{enumitem}

\usepackage{../../../templates/cop/cop-macros}
\usepackage{../../../templates/cop/cop-diagrams}

\title{S2: Routed Tension and Effects}
\author{Pascal Sparidaens}
\date{}

\begin{document}
\maketitle

\begin{abstract}
This Shadow paper documents a single admissibility boundary:
the tension between routed coexistence (forward admissibility)
and effects-first inference (reverse legibility).
It introduces no new structure, notation, or claims.
Its sole function is to prevent collapse into causal reconstruction
or forced resolution of structural tension.
\end{abstract}

\section{Purpose}

This paper is not a contribution.
It is a constraint marker.

S2 binds one Shadow axis:
\begin{itemize}
  \item \textbf{Forward constraint:} COP-02 (Routed Tension),
  \item \textbf{Reverse constraint:} ICOP-02 (Effects-First Ontology).
\end{itemize}

S2 exists to exclude illegitimate moves that arise when tension is treated
as a problem to be solved, or when effects are treated as sufficient causes.

\section{What This Paper Is Not}

This paper does not:
\begin{itemize}
  \item explain conflicts,
  \item resolve contradictions,
  \item infer causes from outcomes,
  \item propose equilibria,
  \item or license synthesis.
\end{itemize}

Any reading that turns S2 into a method for resolution is invalid.

\section{Shadow Axis S2}

\subsection{Forward: COP-02 Routed Tension}

Routed tension constrains forward admissibility:
incompatible structures may coexist provided their interaction
is constrained by routing rather than resolution.

Operationally:
\begin{itemize}
  \item contradiction is admissible,
  \item collapse is not,
  \item separation is structural, not communicative.
\end{itemize}

Tension is not a failure mode.
It is a required operating condition.

\subsection{Reverse: ICOP-02 Effects-First Ontology}

Effects-first ontology constrains reverse legibility:
observable effects do not license causal reconstruction.

Operationally:
\begin{itemize}
  \item effects may be registered,
  \item causes are not recoverable objects,
  \item explanation is not an admissible target.
\end{itemize}

Reverse inference is diagnostic only.
It does not authorize intervention.

\subsection{The Boundary}

The boundary enforced by S2 is the following:

\begin{quote}
\textbf{Tension must be routed, not resolved.} \\
\textbf{Effects must be registered, not explained.}
\end{quote}

Attempts to resolve tension collapse plurality.
Attempts to explain effects reconstruct narratives
that exceed admissibility.

\section{Local \texttt{+} and Non-Local Constraint}

\subsection{Local phenomenology}

Forward operation proceeds only along \texttt{+} branches:
what manifests is what remains distinct under routing.
Tension is locally experienced only as constrained coexistence,
not as conflict requiring resolution.

\subsection{Non-local diagnosability}

Failure of routing is not locally experienced.
It becomes detectable only non-locally, under accumulated pressure,
as interpretive drift, narrative closure, or forced convergence.

Such signals indicate boundary pressure.
They are not events and must not be narrated.

\section{Illegitimate Moves (Forbidden Inferences)}

S2 forbids the following moves:

\begin{enumerate}
  \item \textbf{Resolution demand:}
  treating tension as an error that must be eliminated.
  \item \textbf{Causal inversion:}
  inferring necessary causes from observed effects.
  \item \textbf{Synthesis pressure:}
  forcing incompatible structures into coherence.
  \item \textbf{Narrative smoothing:}
  replacing routed separation with explanatory story.
  \item \textbf{Effect-as-origin:}
  treating outcomes as sufficient grounds for design.
\end{enumerate}

These moves collapse routed tension into illegitimate unity.

\section{Detection Signals}

S2 becomes visible under stress through:

\begin{itemize}
  \item escalation of coordination overhead,
  \item insistence on ``root causes'',
  \item premature harmonization,
  \item explanation replacing const
