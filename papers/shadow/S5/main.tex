\documentclass[11pt]{article}

\usepackage[T1]{fontenc}
\usepackage[utf8]{inputenc}
\usepackage{lmodern}
\usepackage{microtype}
\usepackage{amsmath,amssymb,amsthm}
\usepackage{graphicx}
\usepackage{hyperref}
\usepackage{url}
\usepackage{booktabs}
\usepackage{enumitem}

\usepackage{../../../templates/cop/cop-macros}
\usepackage{../../../templates/cop/cop-diagrams}

\title{S5: Structural Unity and Agency}
\author{Pascal Sparidaens}
\date{}

\begin{document}
\maketitle

\begin{abstract}
This Shadow paper documents a single admissibility boundary:
the tension between structural unity with functional plurality (forward admissibility)
and the production of agency (reverse legibility).
It introduces no new structure, notation, or claims.
Its sole purpose is to prevent misreadings in which unity is treated as uniformity,
or agency is treated as an inherent or assignable property.
\end{abstract}

\section{Purpose}

This paper is not a contribution.
It is a constraint marker.

S5 binds one Shadow axis:
\begin{itemize}
  \item \textbf{Forward constraint:} COP-05 (Structural Unity, Functional Plurality),
  \item \textbf{Reverse constraint:} ICOP-05 (Agency Is Produced, Not Granted).
\end{itemize}

S5 exists to exclude illegitimate moves that arise when structural unity is mistaken
for homogeneity, or when agency is attributed independently of operational conditions.

\section{What This Paper Is Not}

This paper does not:
\begin{itemize}
  \item define roles or hierarchies,
  \item allocate authority,
  \item describe actors or subjects,
  \item explain motivation or intent,
  \item or prescribe empowerment mechanisms.
\end{itemize}

Any reading that turns S5 into a theory of actors, rights, or control is invalid.

\section{Shadow Axis S5}

\subsection{Forward: COP-05 Structural Unity, Functional Plurality}

Structural unity with functional plurality constrains forward admissibility:
a single structural frame may support multiple, non-interchangeable functions.

Operationally:
\begin{itemize}
  \item unity does not imply sameness,
  \item plurality does not imply fragmentation,
  \item roles are constrained by structure, not negotiated.
\end{itemize}

Functional plurality is admissible only insofar as the underlying structure remains intact.

\subsection{Reverse: ICOP-05 Agency Is Produced}

ICOP-05 constrains reverse legibility:
agency is not a given property.
It emerges only under specific operational conditions.

Operationally:
\begin{itemize}
  \item agency follows write-capability,
  \item agency can appear and disappear,
  \item attribution of agency is context-dependent.
\end{itemize}

Reverse reading may detect the presence of agency,
but may not treat it as intrinsic or permanent.

\subsection{The Boundary}

The boundary enforced by S5 is the following:

\begin{quote}
\textbf{Unity does not create agency.} \\
\textbf{Agency does not precede structure.}
\end{quote}

Treating unity as a basis for agency
produces illegitimate authority.
Treating agency as inherent
obscures the structural conditions that produce it.

\section{Local \texttt{+} and Global Constraint}

\subsection{Local phenomenology}

Forward operation proceeds only along \texttt{+} branches:
what manifests are distinct functional roles operating within a shared structure.
Agency appears locally only where write-conditions are satisfied.

\subsection{Non-local diagnosability}

Boundary violations are not locally attributable.
They become detectable only non-locally as role confusion,
claims of inherent authority, or persistence of agency without operational backing.

Such signals are diagnostic only.
They must not be moralized or narrativized.

\section{Illegitimate Moves (Forbidden Inferences)}

S5 forbids the following moves:

\begin{enumerate}
  \item \textbf{Uniformity inference:}
  treating structural unity as functional sameness.
  \item \textbf{Intrinsic agency:}
  assigning agency independent of write conditions.
  \item \textbf{Role flattening:}
  collapsing distinct functions into interchangeable units.
  \item \textbf{Authority naturalization:}
  treating produced agency as inherent right.
  \item \textbf{Agency persistence:}
  assuming agency remains when conditions no longer hold.
\end{enumerate}

These moves collapse admissible plurality
into illegitimate hierarchy or chaos.

\section{Detection Signals}

S5 becomes visible under stress through:

\begin{itemize}
  \item disputes over role legitimacy,
  \item claims of universal agency,
  \item attempts to equalize authority structurally,
  \item confusion between access and entitlement,
  \item structural strain caused by role overreach.
\end{itemize}

These signals mark boundary pressure, not social failure.

\section{Stop Conditions}

Correct use stops at the boundary.

\begin{itemize}
  \item If unity is treated as uniformity, stop (COP-05).
  \item If agency is treated as inherent, stop (ICOP-05).
  \item If roles are collapsed or universalized, stop.
\end{itemize}

S5 does not define actors.
It prevents illegitimate attribution of agency.

\section{Figures (Constraint Objects)}

Figures in Shadow papers are constraint markers only.
They do not explain or illustrate processes.

\begin{figure}[h]
  \centering
  \includegraphics[width=0.92\linewidth]{figures/fig3-return-exposed_S5-overlay.pdf}
  \caption{S5 constraint overlay. The marked structure supports multiple distinct functions without collapsing into uniformity or fragmentation.}
  \label{fig:s5-overlay}
\end{figure}

\begin{figure}[h]
  \centering
  \includegraphics[width=0.72\linewidth]{figures/state-cube_S5-slice.pdf}
  \caption{Optional constraint slice. Agency appears only where write conditions are satisfied; it is not inherent or persistent.}
  \label{fig:s5-cube}
\end{figure}

\section{Conclusion}

S5 exists to enforce a single discipline:
maintain unity without uniformity,
and recognize agency only where it is produced.

Nothing further is added.

\bibliographystyle{unsrt}
\bibliography{references}

\end{document}
