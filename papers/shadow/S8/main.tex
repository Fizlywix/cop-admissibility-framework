\documentclass[11pt]{article}

\usepackage[T1]{fontenc}
\usepackage[utf8]{inputenc}
\usepackage{lmodern}
\usepackage{microtype}
\usepackage{amsmath,amssymb,amsthm}
\usepackage{graphicx}
\usepackage{hyperref}
\usepackage{url}
\usepackage{booktabs}
\usepackage{enumitem}

\usepackage{../../../templates/cop/cop-macros}
\usepackage{../../../templates/cop/cop-diagrams}

\title{S8: Backgrounding and Null Dominance}
\author{Pascal Sparidaens}
\date{}

\begin{document}
\maketitle

\begin{abstract}
This Shadow paper documents the terminal admissibility boundary of the COP corpus:
the relation between null dominance (pre-structural constraint)
and background operation (post-structural constraint).
It introduces no new structure, notation, or claims.
Its sole purpose is to prevent totalization, closure, and final interpretation
of the framework itself.
\end{abstract}

\section{Purpose}

This paper is not a conclusion.
It is a boundary on conclusions.

S8 binds the terminal Shadow axis:
\begin{itemize}
  \item \textbf{Pre-structural constraint:} COP-00 (Null Dominance),
  \item \textbf{Post-structural constraint:} COP-15 (Background Operation).
\end{itemize}

S8 exists to exclude illegitimate moves in which the framework
is treated as complete, total, self-explanatory, or final.

\section{What This Paper Is Not}

This paper does not:
\begin{itemize}
  \item summarize the corpus,
  \item unify the preceding papers,
  \item provide a meta-theory,
  \item explain why the framework “works,”
  \item or license final interpretation.
\end{itemize}

Any reading that treats S8 as closure is invalid.

\section{Shadow Axis S8}

\subsection{Pre-Structural: COP-00 Null Dominance}

Null dominance constrains admissibility before any structure appears:
what functions correctly is not foregrounded.

Operationally:
\begin{itemize}
  \item absence of signal is admissible,
  \item silence does not require explanation,
  \item correct operation leaves no trace.
\end{itemize}

Null dominance forbids the assumption that visibility implies importance.

\subsection{Post-Structural: COP-15 Background Operation}

Background operation constrains admissibility after structure is in place:
what sustains the system does not participate in it.

Operationally:
\begin{itemize}
  \item infrastructure is not an actor,
  \item background processes do not enter the foreground,
  \item stability is maintained without representation.
\end{itemize}

Background operation forbids instrumentalization of the framework itself.

\subsection{The Boundary}

The boundary enforced by S8 is the following:

\begin{quote}
\textbf{What enables operation must not dominate it.} \\
\textbf{What sustains structure must not appear within it.}
\end{quote}

Foregrounding null dominance produces mystification.
Foregrounding background operation produces totalization.

\section{Local \texttt{+} and Global \texttt{---}}

\subsection{Local phenomenology}

Forward use of the corpus proceeds only along \texttt{+} branches:
what appears is what is locally admissible and operational.
Null dominance and background operation are not locally experienced.

\subsection{Global diagnosability}

At the global level, violation of S8 becomes detectable as:
\begin{itemize}
  \item attempts to explain the framework itself,
  \item treatment of constraints as objects,
  \item elevation of background mechanisms to principles,
  \item claims of completeness or finality,
  \item systemic non-response following totalization (\texttt{---}).
\end{itemize}

The configuration \texttt{---} is not a state to be reached.
It is a global diagnostic of overextension.

\section{Illegitimate Moves (Forbidden Inferences)}

S8 forbids the following moves:

\begin{enumerate}
  \item \textbf{Framework totalization:}
  treating the COP corpus as a complete theory.
  \item \textbf{Meta-explanation:}
  explaining why the framework itself must be true.
  \item \textbf{Background foregrounding:}
  turning infrastructure into an actor.
  \item \textbf{Null mystification:}
  treating absence as a hidden force or principle.
  \item \textbf{Final synthesis:}
  attempting to close the corpus with a unifying claim.
\end{enumerate}

These moves collapse constraint into ideology.

\section{Detection Signals}

S8 becomes visible under stress through:

\begin{itemize}
  \item demands for a final explanation,
  \item questions about ultimate meaning or purpose,
  \item attempts to “complete” the framework,
  \item narrative wrapping of constraints,
  \item paralysis following total interpretive closure.
\end{itemize}

These signals indicate boundary violation, not incompleteness.

\section{Stop Conditions}

Correct use stops at the boundary.

\begin{itemize}
  \item If the framework is being explained, stop (COP-00).
  \item If the framework is being operationalized as an object, stop (COP-15).
  \item If closure is demanded, stop.
\end{itemize}

S8 does not conclude the corpus.
It prevents conclusion.

\section{Figures (Constraint Objects)}

Figures in Shadow papers are constraint markers only.
They do not explain, illustrate, or unify.

\begin{figure}[h]
  \centering
  % Placeholder for cap overlay (COP-00 / COP-15)
  \includegraphics[width=0.92\linewidth]{figures/fig3-return-exposed_S8-overlay.pdf}
  \caption{S8 constraint overlay. COP-00 and COP-15 act as caps: pre-structural null dominance and post-structural background operation.}
  \label{fig:s8-overlay}
\end{figure}

\begin{figure}[h]
  \centering
  % Placeholder for global diagnosability marker
  \includegraphics[width=0.72\linewidth]{figures/state-cube_S8-slice.pdf}
  \caption{Optional constraint slice. Global non-response (\texttt{---}) is diagnostically visible but never locally traversed.}
  \label{fig:s8-cube}
\end{figure}

\section{Conclusion}

S8 exists to enforce a single discipline:
do not close what functions only by remaining open.

Nothing further is added.

\bibliographystyle{unsrt}
\bibliography{references}

\end{document}

