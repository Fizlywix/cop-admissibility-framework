\documentclass[11pt]{article}

\usepackage[T1]{fontenc}
\usepackage[utf8]{inputenc}
\usepackage{lmodern}
\usepackage{microtype}
\usepackage{amsmath,amssymb,amsthm}
\usepackage{graphicx}
\usepackage{hyperref}
\usepackage{url}
\usepackage{booktabs}
\usepackage{enumitem}

\usepackage{../../../templates/cop/cop-macros}
\usepackage{../../../templates/cop/cop-diagrams}

\title{A1: If You Are Looking for Answers, Close This Document}
\author{Pascal Sparidaens}
\date{}

\begin{document}
\maketitle

\begin{abstract}
This document introduces no structure.
It exists to prevent a specific category error:
treating the COP corpus as a source of answers.
Readers seeking explanations, conclusions, or truths
are instructed to stop reading immediately.
\end{abstract}

\section{Purpose}

This document functions as a gate.

It does not prepare the reader.
It filters the reader.

The COP corpus is frequently misapproached as:
\begin{itemize}
  \item an explanatory framework,
  \item a theory of reality,
  \item a model of systems,
  \item or a source of answers.
\end{itemize}

It is none of these.

\section{If You Are Looking for Answers}

If you are looking for:
\begin{itemize}
  \item what exists,
  \item how things work,
  \item why events occur,
  \item what is true,
  \item what should be done,
\end{itemize}
close this document.

Nothing in the corpus will satisfy that demand.

\section{What This Corpus Does Instead}

The COP corpus does not provide answers.
It provides constraints.

It does not tell you:
\begin{itemize}
  \item what to believe,
  \item what to conclude,
  \item what to accept,
  \item or what to reject.
\end{itemize}

It specifies only which moves are illegitimate.

\section{On Discomfort}

Readers often experience discomfort
when expected explanations are withheld.

This discomfort is not a signal of depth,
mystery, or hidden meaning.

It is the result of encountering a stopping condition.

\section{No Hidden Layer}

There is no deeper explanation later in the corpus.

There is no final synthesis.
There is no moment where ambiguity resolves.
There is no section where meaning is revealed.

Continued reading will not produce answers.
It will only remove options.

\section{Correct Motivation}

You may continue reading only if you are interested in:
\begin{itemize}
  \item disciplined exclusion,
  \item admissibility boundaries,
  \item structural stopping conditions,
  \item and the prevention of illegitimate inference.
\end{itemize}

Curiosity about limits is admissible.
Demand for answers is not.

\section{Non-Commitment Clause}

Reading the corpus requires no belief.

Agreement is irrelevant.
Disagreement is irrelevant.

Only use is relevant.

\section{Final Instruction}

If at any point you begin to ask:
\begin{quote}
``Yes, but what does this \emph{mean}?''
\end{quote}
stop reading.

The framework has already done its job.

\bibliographystyle{unsrt}
\bibliography{references}

\end{document}
