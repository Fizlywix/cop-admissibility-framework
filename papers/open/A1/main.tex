\documentclass[11pt]{article}

\usepackage[T1]{fontenc}
\usepackage[utf8]{inputenc}
\usepackage{lmodern}
\usepackage{microtype}
\usepackage{amsmath,amssymb,amsthm}
\usepackage{graphicx}
\usepackage{hyperref}
\usepackage{url}
\usepackage{booktabs}
\usepackage{enumitem}

\usepackage{../../../templates/cop/cop-macros}
\usepackage{../../../templates/cop/cop-diagrams}

\title{A1: If You Are Looking for Answers, Close This Document}
\author{Pascal Sparidaens}
\date{}

\begin{document}
\maketitle

\begin{abstract}
This document introduces no structure.
It defines the status of the COP corpus.
The corpus is neither a theory nor a method.
It is a lens that makes certain moves visibly illegitimate.
It produces no answers, prescriptions, or outputs.
\end{abstract}

\section{What This Is}

The COP corpus is a constraint lens.

When applied to a line of thought, argument, or explanation,
it reveals where articulation exceeds admissibility.

The lens does not generate conclusions.
It does not select correct alternatives.
It does not recommend actions.

It only marks where continuation is illegitimate.

\section{What a Lens Does}

A lens:
\begin{itemize}
  \item alters visibility without altering the object,
  \item reveals limits without resolving them,
  \item and disappears when no longer applied.
\end{itemize}

The COP corpus functions exactly in this way.

Any attempt to treat the lens itself as an object,
output, or authority constitutes misuse.

\section{What This Is Not}

The COP corpus is not:
\begin{itemize}
  \item a theory of reality,
  \item a model of systems,
  \item a method for decision-making,
  \item a programming construct,
  \item or a source of truth.
\end{itemize}

It cannot be executed, followed, or obeyed.

\section{No Output Clause}

The COP corpus produces no outputs.

If you believe it has produced:
\begin{itemize}
  \item an answer,
  \item a rule,
  \item a conclusion,
  \item or a prescription,
\end{itemize}
you have already misapplied it.

The only admissible outcome is exclusion of illegitimate moves.

\section{Correct and Incorrect Use}

Correct use consists in noticing:
\begin{itemize}
  \item when explanation turns into narration,
  \item when interpretation replaces constraint,
  \item when totality is implicitly assumed,
  \item and when continuation is forced past saturation.
\end{itemize}

Incorrect use consists in asking what should replace the excluded move.

No replacement is provided.

\section{On Discomfort}

Application of the lens often results in discomfort.

This discomfort signals that a stopping condition has been reached.

Relief, certainty, or guidance indicate misuse.

\section{Abandonment Condition}

The lens must be dropped once illegitimacy is visible.

Continued use beyond that point
reintroduces the very overextension it excludes.

Abandonment is therefore part of correct use.

\section{Final Instruction}

If at any point you find yourself asking:
\begin{quote}
``So what does the COP framework say is correct?''
\end{quote}
close this document.

The lens has already done all it can do.

\bibliographystyle{unsrt}
\bibliography{references}

\end{document}
