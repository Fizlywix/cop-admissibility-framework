\documentclass[11pt]{article}

\usepackage[T1]{fontenc}
\usepackage[utf8]{inputenc}
\usepackage{lmodern}
\usepackage{microtype}
\usepackage{amsmath,amssymb,amsthm}
\usepackage{graphicx}
\usepackage{hyperref}
\usepackage{url}
\usepackage{booktabs}
\usepackage{enumitem}

\usepackage{../../../templates/cop/cop-macros}
\usepackage{../../../templates/cop/cop-diagrams}

\title{Shadow Appendix: Paradox Mapping Across Admissibility Surfaces}
\author{Pascal Sparidaens}
\date{}

\begin{document}
\maketitle

\begin{abstract}
This appendix introduces no new structure.
It maps well-known paradoxes across scientific, philosophical, and formal domains
onto the Shadow axes of the COP framework.
The purpose is not resolution, explanation, or comparison,
but localization of where admissibility necessarily breaks.
\end{abstract}

\section{Purpose}

Paradoxes recur across domains because the same inadmissible moves
are repeatedly attempted under different narratives.

This appendix treats paradoxes as diagnostic artifacts:
signals that admissibility has been exceeded.

No paradox is solved.
Each is placed.

\section{Method}

Each paradox is mapped to:
\begin{itemize}
  \item a dominant Shadow axis (S1--S8),
  \item the illegitimate operation producing it,
  \item and the reason resolution is forbidden.
\end{itemize}

No semantic interpretation is required.

\section{Temporal Paradoxes}

\subsection{Zeno’s Paradoxes}

\textbf{Shadow Axis: S2 (Resolution Demand)}

Illegitimate move:
forcing completion of an infinite subdivision.

Why it persists:
time is treated as reconstructable via articulation.

Admissible response:
stop subdivision; motion is not narratable.

\subsection{Grandfather Paradox}

\textbf{Shadow Axis: S3 (Reverse Reconstruction)}

Illegitimate move:
using reverse legibility to generate forward causation.

Admissible response:
history is not writable backward.

\section{Identity Paradoxes}

\subsection{Ship of Theseus}

\textbf{Shadow Axis: S1 (Total Preservation)}

Illegitimate move:
demanding identity persistence under complete replacement.

Admissible response:
identity is not a preserved object.

\subsection{Personal Identity Over Time}

\textbf{Shadow Axis: S4 (Responsibility Localization)}

Illegitimate move:
treating agency as a stable, local entity.

Admissible response:
agency is distributed and situational.

\section{Logical and Semantic Paradoxes}

\subsection{Liar Paradox}

\textbf{Shadow Axis: S8 (Visibility Privileging)}

Illegitimate move:
forcing self-reference into total articulation.

Admissible response:
self-reference saturates legibility.

\subsection{Russell’s Paradox}

\textbf{Shadow Axis: S7 (Expansion Past Saturation)}

Illegitimate move:
constructing total sets without boundary constraints.

Admissible response:
totality claims are inadmissible.

\section{Physical Paradoxes}

\subsection{Wave--Particle Duality}

\textbf{Shadow Axis: S3 (Coherence via Explanation)}

Illegitimate move:
demanding a single explanatory picture.

Admissible response:
multiple constraint-consistent descriptions coexist.

\subsection{Measurement Problem}

\textbf{Shadow Axis: S8 (Foregrounding Infrastructure)}

Illegitimate move:
treating measurement as a privileged explanatory act.

Admissible response:
measurement is a constraint interaction, not a revelation.

\section{Computational Paradoxes}

\subsection{Halting Problem}

\textbf{Shadow Axis: S2 (Resolution Demand)}

Illegitimate move:
demanding global prediction of local computation.

Admissible response:
non-decidability is structural.

\subsection{Self-Improving Systems}

\textbf{Shadow Axis: S7 (Unbounded Extension)}

Illegitimate move:
assuming indefinite admissible expansion.

Admissible response:
optimization saturates.

\section{Existential and Moral Paradoxes}

\subsection{Free Will vs Determinism}

\textbf{Shadow Axis: S5 (Agency Naturalization)}

Illegitimate move:
treating agency as a metaphysical property.

Admissible response:
agency is an operational attribution.

\subsection{Problem of Evil}

\textbf{Shadow Axis: S1 (Preservation Fantasy)}

Illegitimate move:
demanding total preservation under constraint.

Admissible response:
loss is structural.

\section{Non-Resolution Principle}

No paradox listed here admits resolution within the COP framework.

Attempted resolutions invariably:
\begin{itemize}
  \item add hidden structure,
  \item reintroduce narrative closure,
  \item or exceed admissibility.
\end{itemize}

Paradoxes persist because they are correctly signaling limits.

\section{Closure}

Paradoxes do not indicate failure of understanding.
They indicate violation of admissibility.

The correct response to a paradox is not explanation,
but cessation of the illegitimate move that produced it.

\bibliographystyle{unsrt}
\bibliography{references}

\end{document}
