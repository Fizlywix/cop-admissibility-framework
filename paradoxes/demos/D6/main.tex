\documentclass[11pt]{article}

\usepackage[T1]{fontenc}
\usepackage[utf8]{inputenc}
\usepackage{lmodern}
\usepackage{microtype}
\usepackage{amsmath,amssymb,amsthm}
\usepackage{graphicx}
\usepackage{hyperref}
\usepackage{url}
\usepackage{booktabs}
\usepackage{enumitem}

\usepackage{../../../templates/cop/cop-macros}
\usepackage{../../../templates/cop/cop-diagrams}

\title{D6: Death as Transfer, Not Negation}
\author{Pascal Sparidaens}
\date{}

\begin{document}
\maketitle

\begin{abstract}
This demonstration examines death as a mode of structural transfer
rather than as absolute negation or terminal erasure.
Using classical paradoxes of death and persistence as reference,
it shows how continuity may be preserved across system boundaries,
and how collapse occurs when death is forced into finality.
No resolution is proposed.
The paradoxes are used solely to orient admissible perception.
\end{abstract}

\section{Purpose}

This paper demonstrates a single admissibility distinction:
death may occur as pruning, handoff, or loss of legibility
without implying total disappearance or final negation.

Within COP terms, this corresponds to admissible return and redistribution
without escalation into absolute termination.

The purpose is not to deny loss,
but to show how continuity may persist
when death is read structurally rather than absolutely.

\section{What This Demonstration Is Not}

This demonstration does not:
\begin{itemize}
  \item deny the reality of death,
  \item claim immortality,
  \item preserve individual identity,
  \item assert continuation of experience,
  \item or resolve existential questions.
\end{itemize}

Any reading that treats death as requiring total negation
misreads the demonstration.

\section{What Becomes Visible}

\subsection{Death as Pruning and Redistribution}

When read admissibly:
\begin{itemize}
  \item components are removed from one system,
  \item resources are released into surrounding systems,
  \item structure may persist elsewhere without identity.
\end{itemize}

Under COP forward admissibility,
death appears as local collapse
and non-local continuation.

\subsection{Why Negation Is Not Required}

Negation is a global claim.
Death is a local event.

Nothing in the cessation of a system’s operation
requires that all structure, influence, or constraint vanish.
Loss of legibility does not imply loss of effect.

\section{Demonstration: Paradoxes of Death}

Classical death paradoxes demonstrate a tension,
not an absence.

\begin{itemize}
  \item Death is described as non-experience.
  \item Concern is framed as meaningless after death.
\end{itemize}

The paradox arises only when:
\begin{itemize}
  \item death is equated with absolute nothingness,
  \item continuity is restricted to subjective presence,
  \item structural transfer is ignored.
\end{itemize}

When death is read as redistribution across systems,
the paradox reframes without resolution.

\section{How Collapse Occurs}

Collapse occurs when:
\begin{itemize}
  \item death is treated as total erasure,
  \item continuity is demanded to remain personal,
  \item existence is equated with experience,
  \item structural effects are dismissed.
\end{itemize}

At this point, admissible loss
is replaced by metaphysical finality.

\section{Admissible Reading}

This demonstration permits the following reading:

\begin{quote}
Death may terminate a system locally
while allowing structure, influence, or resources
to persist beyond that system.
\end{quote}

Loss and continuation coexist
without contradiction.

\section{Stop Condition}

This demonstration stops at the point where:
\begin{itemize}
  \item death is equated with absolute negation,
  \item continuity is required to be experiential,
  \item explanation replaces structural observation.
\end{itemize}

Beyond this point, death collapses into metaphysical claim.

\section{Conclusion}

Paradoxes of death do not negate continuity.
They expose the error of treating death as final negation.

Death redistributes.
Continuity shifts.

Nothing further is claimed.

\bibliographystyle{unsrt}
\bibliography{references}

\end{document}
