\documentclass[11pt]{article}

\usepackage[T1]{fontenc}
\usepackage[utf8]{inputenc}
\usepackage{lmodern}
\usepackage{microtype}
\usepackage{amsmath,amssymb,amsthm}
\usepackage{graphicx}
\usepackage{hyperref}
\usepackage{url}
\usepackage{booktabs}
\usepackage{enumitem}

\usepackage{../../../templates/cop/cop-macros}
\usepackage{../../../templates/cop/cop-diagrams}

\title{D4: Gradual Change Without Threshold}
\author{Pascal Sparidaens}
\date{}

\begin{document}
\maketitle

\begin{abstract}
This demonstration examines gradual change as an admissible mode of transformation
that does not require discrete thresholds or sharp boundaries.
Using the Sorites paradox as reference,
it shows how accumulation and depletion remain coherent under COP forward admissibility,
and how collapse occurs when continuity is forced into binary classification.
No resolution is proposed.
The paradox is used solely to orient admissible perception.
\end{abstract}

\section{Purpose}

This paper demonstrates a single admissibility distinction:
change may occur continuously without crossing a definable threshold.

Within COP terms, this corresponds to admissible progression under local accumulation
without escalation into categorical rupture.

The purpose is not to define when change “becomes” something else,
but to show how continuity remains valid
until sharp boundaries are imposed.

\section{What This Demonstration Is Not}

This demonstration does not:
\begin{itemize}
  \item define exact thresholds,
  \item deny measurable differences,
  \item resolve vagueness formally,
  \item privilege binary classification,
  \item or stabilize categories.
\end{itemize}

Any reading that treats gradual change as requiring a precise cutoff
misreads the demonstration.

\section{What Becomes Visible}

\subsection{Gradual Accumulation and Depletion}

When read admissibly:
\begin{itemize}
  \item change proceeds by small, locally coherent increments,
  \item no single step is decisive on its own,
  \item accumulation and loss remain intelligible without labels.
\end{itemize}

Under COP forward admissibility,
local change is valid without categorical transition.

\subsection{Why Thresholds Are Not Required}

Thresholds are global declarations.
Gradual change is a local process.

Nothing in continuous accumulation or depletion
requires a moment at which identity must flip.
Continuity persists without discrete boundaries.

\section{Demonstration: The Sorites Paradox}

The Sorites paradox demonstrates a tension,
not a contradiction.

\begin{itemize}
  \item A heap changes grain by grain.
  \item No single grain determines heapness.
\end{itemize}

The paradox arises only when:
\begin{itemize}
  \item a binary classification is demanded,
  \item continuity is forced into categorical language.
\end{itemize}

When change is allowed to remain gradual,
the paradox does not obstruct coherence.

\section{How Collapse Occurs}

Collapse occurs when:
\begin{itemize}
  \item gradual variation is forced into discrete states,
  \item classification replaces observation,
  \item identity is demanded from accumulation,
  \item continuity is denied without a threshold.
\end{itemize}

At this point, local admissibility is replaced
by global labeling pressure.

\section{Admissible Reading}

This demonstration permits the following reading:

\begin{quote}
Change may proceed continuously and remain valid
without crossing a definable threshold or category boundary.
\end{quote}

Accumulation and depletion are intelligible
without binary resolution.

\section{Stop Condition}

This demonstration stops at the point where:
\begin{itemize}
  \item a precise cutoff is demanded,
  \item gradual change is denied legitimacy,
  \item categorical identity replaces continuity.
\end{itemize}

Beyond this point, description collapses into classification.

\section{Conclusion}

The Sorites paradox does not negate gradual change.
It exposes the error of forcing thresholds onto continuity.

Change persists locally.
Boundaries are optional.

Nothing further is claimed.

\bibliographystyle{unsrt}
\bibliography{references}

\end{document}
