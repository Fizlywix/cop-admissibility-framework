\documentclass[11pt]{article}

\usepackage[T1]{fontenc}
\usepackage[utf8]{inputenc}
\usepackage{lmodern}
\usepackage{microtype}
\usepackage{amsmath,amssymb,amsthm}
\usepackage{graphicx}
\usepackage{hyperref}
\usepackage{url}
\usepackage{booktabs}
\usepackage{enumitem}

\usepackage{../../../templates/cop/cop-macros}
\usepackage{../../../templates/cop/cop-diagrams}

\title{D7: Knowledge Without Completion}
\author{Pascal Sparidaens}
\date{}

\begin{document}
\maketitle

\begin{abstract}
This demonstration examines knowledge as a locally coherent,
operationally sufficient activity that does not require global completion
or total consistency.
Using classical incompleteness and self-reference paradoxes as reference,
it shows how knowing remains admissible under partial closure,
and how collapse occurs when knowledge is forced into total explanation.
No resolution is proposed.
The paradoxes are used solely to orient admissible perception.
\end{abstract}

\section{Purpose}

This paper demonstrates a single admissibility distinction:
knowledge may function and remain valid
without being complete, closed, or final.

Within COP terms, this corresponds to admissible local coherence
without escalation into total explanation or universal consistency.

The purpose is not to deny truth or rigor,
but to show how knowing remains operational
until completion is demanded.

\section{What This Demonstration Is Not}

This demonstration does not:
\begin{itemize}
  \item deny the possibility of truth,
  \item reject formal systems,
  \item resolve incompleteness,
  \item guarantee consistency everywhere,
  \item or assert epistemic relativism.
\end{itemize}

Any reading that treats knowledge as requiring total completion
misreads the demonstration.

\section{What Becomes Visible}

\subsection{Knowledge as Local Coherence}

When read admissibly:
\begin{itemize}
  \item knowledge operates within bounded domains,
  \item consistency is enforced locally,
  \item usefulness does not depend on global closure.
\end{itemize}

Under COP forward admissibility,
knowledge is sufficient wherever it works,
without reference to what lies outside its domain.

\subsection{Why Completion Is Not Required}

Completion is a global demand.
Knowledge is a local activity.

Nothing in the use, application, or maintenance of knowledge
requires that all truths be derivable,
all questions be answerable,
or all contradictions be resolved.

Knowing proceeds without total reach.

\section{Demonstration: Incompleteness and Self-Reference}

Incompleteness paradoxes demonstrate a boundary,
not a failure.

\begin{itemize}
  \item Formal systems remain usable despite incompleteness.
  \item Undecidable statements do not invalidate decidable ones.
\end{itemize}

The paradox arises only when:
\begin{itemize}
  \item knowledge is expected to be self-contained,
  \item truth is equated with provability everywhere,
  \item systems are forced to close over themselves.
\end{itemize}

When knowledge is allowed to remain partial,
the paradox does not obstruct operation.

\section{How Collapse Occurs}

Collapse occurs when:
\begin{itemize}
  \item completeness is demanded as a condition of validity,
  \item undecidability is treated as failure,
  \item explanation is required to exhaust all structure,
  \item knowing is conflated with total description.
\end{itemize}

At this point, admissible knowledge
is replaced by epistemic overreach.

\section{Admissible Reading}

This demonstration permits the following reading:

\begin{quote}
Knowledge may remain valid and effective
without being complete, final, or universally closed.
\end{quote}

Partial coherence is sufficient.
Total explanation is optional.

\section{Stop Condition}

This demonstration stops at the point where:
\begin{itemize}
  \item knowledge is denied due to incompleteness,
  \item undecidability is used to negate validity,
  \item explanation replaces operational sufficiency.
\end{itemize}

Beyond this point, knowing collapses into totalization.

\section{Conclusion}

Paradoxes of incompleteness do not negate knowledge.
They expose the error of demanding completion.

Knowledge functions locally.
Completion is not required.

Nothing further is claimed.

\bibliographystyle{unsrt}
\bibliography{references}

\end{document}
