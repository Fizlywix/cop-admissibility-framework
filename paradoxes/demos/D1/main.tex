\documentclass[11pt]{article}

\usepackage[T1]{fontenc}
\usepackage[utf8]{inputenc}
\usepackage{lmodern}
\usepackage{microtype}
\usepackage{amsmath,amssymb,amsthm}
\usepackage{graphicx}
\usepackage{hyperref}
\usepackage{url}
\usepackage{booktabs}
\usepackage{enumitem}

\usepackage{../../../templates/cop/cop-macros}
\usepackage{../../../templates/cop/cop-diagrams}

\title{D1: Persistence Without Substance}
\author{Pascal Sparidaens}
\date{}

\begin{document}
\maketitle

\begin{abstract}
This Shadow paper documents a single admissibility boundary:
the distinction between persistence as structural continuity
and identity as material or essential sameness.
It introduces no new structure, notation, or claims.
Its sole purpose is to permit persistence without substance
while excluding illegitimate demands for preserved essence.
\end{abstract}

\section{Purpose}

This paper is not an explanation.
It is a lens boundary.

D1 binds one admissibility tension:
persistence may remain valid under complete material turnover,
while identity-as-substance must not be inferred.

D1 exists to prevent the collapse of continuity
into essentialism, origin fixation, or material identity.

\section{What This Paper Is Not}

This paper does not:
\begin{itemize}
  \item define identity,
  \item resolve paradoxes,
  \item preserve objects,
  \item privilege material continuity,
  \item or assign essence.
\end{itemize}

Any reading that treats persistence as requiring preserved substance
is invalid.

\section{Admissible View}

\subsection{YES: Persistence as Structural Continuity}

Persistence is admissible as long as:
\begin{itemize}
  \item relational structure remains coherent,
  \item functional constraints remain satisfied,
  \item turnover does not break continuity of operation.
\end{itemize}

Material replacement does not negate persistence.
Continuity is enforced by constraint compatibility,
not by retention of parts.

\subsection{NO: Identity as Preserved Substance}

The following inferences are illegitimate:
\begin{itemize}
  \item that an object ceases to persist once its components change,
  \item that sameness requires original material,
  \item that identity is anchored in physical essence.
\end{itemize}

Substance fixation replaces admissible continuity
with metaphysical demand.

\section{Canonical Paradox: The Ship of Theseus}

The Ship of Theseus is not a problem to be solved.
It is a diagnostic boundary marker.

\begin{itemize}
  \item The ship persists structurally under replacement.
  \item The paradox appears only when substance is treated as identity.
\end{itemize}

D1 permits the ship to persist.
D1 forbids the demand that persistence resolve into sameness of matter.

\section{Local and Non-Local Reading}

\subsection{Local admissibility}

Locally, replacement is admissible.
No replacement event invalidates persistence on its own.

\subsection{Non-local failure mode}

The paradox becomes visible only when:
\begin{itemize}
  \item accumulation of replacements is re-narrated,
  \item origin is treated as explanatory anchor,
  \item continuity is forced into sameness.
\end{itemize}

These are diagnostic signals only.

\section{Illegitimate Moves (Forbidden Inferences)}

D1 forbids:
\begin{enumerate}
  \item \textbf{Essence anchoring:}
  treating original material as identity.
  \item \textbf{Total replacement collapse:}
  declaring persistence void after full turnover.
  \item \textbf{Origin privilege:}
  using beginnings to justify present validity.
  \item \textbf{Material accounting:}
  treating inventories as ontological evidence.
\end{enumerate}

These moves replace structural continuity
with metaphysical assertion.

\section{Detection Signals}

Boundary pressure becomes visible as:
\begin{itemize}
  \item insistence on original components,
  \item debates over “real” versus “copy,”
  \item archival fixation on beginnings,
  \item substitution of material history for constraint analysis.
\end{itemize}

These signals indicate misuse, not failure.

\section{Stop Conditions}

Correct use stops at the boundary.

\begin{itemize}
  \item If persistence is denied due to material turnover, stop.
  \item If identity is grounded in substance, stop.
  \item If origin is used as explanation, stop.
\end{itemize}

D1 does not preserve objects.
It preserves admissible continuity.

\section{Conclusion}

D1 enforces a single discipline:
persistence without substance,
continuity without essence.

Nothing further is added.

\bibliographystyle{unsrt}
\bibliography{references}

\end{document}
