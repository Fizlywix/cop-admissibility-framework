\documentclass[11pt]{article}

\usepackage[T1]{fontenc}
\usepackage[utf8]{inputenc}
\usepackage{lmodern}
\usepackage{microtype}
\usepackage{amsmath,amssymb,amsthm}
\usepackage{graphicx}
\usepackage{hyperref}
\usepackage{url}
\usepackage{booktabs}
\usepackage{enumitem}

\usepackage{../../../templates/cop/cop-macros}
\usepackage{../../../templates/cop/cop-diagrams}

\title{D3: Branching Without Totality}
\author{Pascal Sparidaens}
\date{}

\begin{document}
\maketitle

\begin{abstract}
This demonstration examines branching as an admissible mode of continuation
that does not imply totality, completeness, or exhaustive realization.
Using classical branching paradoxes and many-path interpretations as reference,
it shows how multiplicity remains coherent under COP forward admissibility,
and how collapse occurs when branching is forced into total inclusion.
No resolution is proposed.
The paradoxes are used solely to orient admissible perception.
\end{abstract}

\section{Purpose}

This paper demonstrates a single admissibility distinction:
branching may occur without producing a total set.

Within COP terms, this corresponds to the visibility of COP-02 (1 $\rightarrow$ VEEL)
without escalation to ALL.

The purpose is not to decide which branch is real,
nor to assert that all branches exist,
but to show how branching remains coherent
until totality is demanded.

\section{What This Demonstration Is Not}

This demonstration does not:
\begin{itemize}
  \item assert the existence of a multiverse,
  \item deny the existence of alternatives,
  \item rank branches by reality,
  \item claim that all possibilities are realized,
  \item or resolve ontological debates.
\end{itemize}

Any reading that treats branching as requiring total inclusion
misreads the demonstration.

\section{What Becomes Visible}

\subsection{Branching as Local Differentiation}

When read admissibly:
\begin{itemize}
  \item branching produces multiple viable continuations,
  \item each branch remains locally coherent,
  \item no branch requires knowledge of the others.
\end{itemize}

Under COP-02, VEEL denotes multiplicity without totality.
Branching increases structure without exhausting possibility.

\subsection{Why Totality Does Not Follow}

Totality is a global claim.
Branching is a local operation.

Nothing in the act of branching
requires that all branches be realized,
catalogued, or preserved.

Branching remains admissible
as long as it does not escalate into ALL.

\section{Demonstration: Branching Paradoxes}

Branching paradoxes appear wherever:
\begin{itemize}
  \item multiple futures are acknowledged,
  \item no single outcome is privileged,
  \item yet a demand is made to account for all possibilities.
\end{itemize}

Examples include:
\begin{itemize}
  \item many-path decision problems,
  \item many-worlds interpretations,
  \item counterfactual proliferation.
\end{itemize}

The paradox arises not from branching itself,
but from the demand that branching imply total realization.

\section{How Collapse Occurs}

Collapse occurs when:
\begin{itemize}
  \item VEEL is reinterpreted as ALL,
  \item branching is treated as exhaustive enumeration,
  \item possibility is mistaken for existence,
  \item completeness is demanded of continuation.
\end{itemize}

At this point, COP-07 pressure becomes visible:
the system saturates under totalization.

\section{Admissible Reading}

This demonstration permits the following reading:

\begin{quote}
Branching produces multiple coherent continuations,
without implying that all continuations must exist or be realized.
\end{quote}

Multiplicity is admissible.
Totality is not required.

\section{Stop Condition}

This demonstration stops at the point where:
\begin{itemize}
  \item branching is equated with total inclusion,
  \item VEEL is forced into ALL,
  \item explanation replaces structural observation.
\end{itemize}

Beyond this point, continuation collapses into enumeration.

\section{Conclusion}

Branching is a mode of continuation, not a claim of totality.
COP permits VEEL.
It does not require ALL.

Nothing further is claimed.

\bibliographystyle{unsrt}
\bibliography{references}

\end{document}
