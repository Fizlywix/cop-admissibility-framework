\documentclass[11pt]{article}

\usepackage[T1]{fontenc}
\usepackage[utf8]{inputenc}
\usepackage{lmodern}
\usepackage{microtype}
\usepackage{amsmath,amssymb,amsthm}
\usepackage{graphicx}
\usepackage{hyperref}
\usepackage{url}
\usepackage{booktabs}
\usepackage{enumitem}

\usepackage{../../../templates/cop/cop-macros}
\usepackage{../../../templates/cop/cop-diagrams}

\title{S1: Residue Without Instruction}
\author{Pascal Sparidaens}
\date{}

\begin{document}
\maketitle

\begin{abstract}
This demonstration examines residue as what remains after process,
without encoding instruction, intent, or complete history.
It shows how traces constrain inference
without authorizing reconstruction,
and how collapse occurs when residue is treated as directive.
No archive is proposed.
Residue is presented solely as bounded evidence.
\end{abstract}

\section{Purpose}

This paper demonstrates a simple distinction:
what remains after activity
does not prescribe how it occurred.

Its purpose is not to deny evidential value,
but to show what becomes visible
when trace is separated
from instruction.

\section{What This Demonstration Is Not}

This demonstration does not:
\begin{itemize}
  \item deny the relevance of evidence,
  \item reject investigation,
  \item prohibit interpretation,
  \item privilege ignorance,
  \item or eliminate inquiry.
\end{itemize}

Any reading that treats residue
as encoded intent
misunderstands its role.

\section{What Becomes Visible}

\subsection{Residue as Constrained Effect}

When residue is present:
\begin{itemize}
  \item effects persist,
  \item causes do not remain legible,
  \item loss is embedded.
\end{itemize}

Traces register impact,
not procedure.

They bound inference
without completing it.

\subsection{Why Instruction Is Not Preserved}

Instruction is a forward relation.
Residue is a backward condition.

Nothing in remaining structure
requires that generative rules
be recoverable.

Processes complete
without leaving manuals.

\section{Demonstration: Partial Records}

Consider a system
that leaves incomplete records.

\begin{itemize}
  \item Outcomes remain observable.
  \item Internal operations disappear.
\end{itemize}

Multiple histories
fit the same residue.

When a single narrative
is imposed,
evidence is overextended.

\section{How Collapse Occurs}

Collapse occurs when:
\begin{itemize}
  \item traces are treated as blueprints,
  \item evidence is equated with intent,
  \item outcomes are re-narrated as plans,
  \item gaps are filled with certainty.
\end{itemize}

Under these conditions,
inference becomes fiction.

Residue is converted into story.

\section{Admissible Reading}

This demonstration permits the following reading:

\begin{quote}
What remains after a process
constrains interpretation
without authorizing reconstruction.
\end{quote}

This reading preserves evidence
without totalizing it.

\section{Stop Condition}

This demonstration stops at the point where:
\begin{itemize}
  \item traces are treated as instructions,
  \item absence is interpreted as message,
  \item residue is assumed complete,
  \item inference outruns evidence.
\end{itemize}

Beyond this point,
legibility collapses into narration.

\section{Conclusion}

Residue limits inference.
It does not direct it.

Traces persist.
Histories dissolve.

Nothing further is claimed.

\bibliographystyle{unsrt}
\bibliography{references}

\end{document}
