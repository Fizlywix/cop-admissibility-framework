\documentclass[11pt]{article}

\usepackage[T1]{fontenc}
\usepackage[utf8]{inputenc}
\usepackage{lmodern}
\usepackage{microtype}
\usepackage{amsmath,amssymb,amsthm}
\usepackage{graphicx}
\usepackage{hyperref}
\usepackage{url}
\usepackage{booktabs}
\usepackage{enumitem}

\usepackage{../../../templates/cop/cop-macros}
\usepackage{../../../templates/cop/cop-diagrams}

\title{S4: Decoupling Without Map}
\author{Pascal Sparidaens}
\date{}

\begin{document}
\maketitle

\begin{abstract}
This demonstration examines decoupling as the loss of relational structure
without preservation of prior connectivity.
It shows how systems may fragment
without leaving recoverable maps,
and how collapse occurs when disconnected states
are treated as evidence of former organization.
No reconstruction is proposed.
Decoupling is presented solely as loss of relational legibility.
\end{abstract}

\section{Purpose}

This paper demonstrates a simple distinction:
disconnection does not preserve
the structure of what was connected.

Its purpose is not to deny past relations,
but to show what becomes visible
when links dissolve
without traceable pattern.

\section{What This Demonstration Is Not}

This demonstration does not:
\begin{itemize}
  \item reconstruct former networks,
  \item privilege surviving nodes,
  \item infer original structure,
  \item assign central causes,
  \item or restore lost relations.
\end{itemize}

Any reading that treats fragments
as maps of former systems
misunderstands decoupling.

\section{What Becomes Visible}

\subsection{Fragmentation as Relational Loss}

When decoupling occurs:
\begin{itemize}
  \item connections disappear,
  \item coordination dissolves,
  \item isolation increases.
\end{itemize}

Remaining elements
no longer encode
their former relations.

Separation is structural,
not descriptive.

\subsection{Why Maps Do Not Survive}

Maps require stable links.
Decoupling removes them.

Nothing in disconnected states
requires that prior pathways
remain inferable.

Relational history
vanishes with connectivity.

\section{Demonstration: Network Breakdown}

Consider a system
after widespread disconnection.

\begin{itemize}
  \item Components persist.
  \item Relations are absent.
\end{itemize}

Multiple past configurations
fit the same fragmented state.

When fragments are arranged
into a single narrative,
structure is invented.

\section{How Collapse Occurs}

Collapse occurs when:
\begin{itemize}
  \item fragments are treated as diagrams,
  \item isolation is read as evidence,
  \item blame is centralized,
  \item networks are reconstructed from endpoints.
\end{itemize}

Under these conditions,
loss is converted into explanation.

Absence becomes structure.

\section{Admissible Reading}

This demonstration permits the following reading:

\begin{quote}
After decoupling,
remaining elements do not authorize
reconstruction of former relational structure.
\end{quote}

This reading preserves fragmentation
without fictional maps.

\section{Stop Condition}

This demonstration stops at the point where:
\begin{itemize}
  \item disconnected parts are re-linked narratively,
  \item origins are inferred from fragments,
  \item isolation is moralized,
  \item history is rebuilt without residue.
\end{itemize}

Beyond this point,
decoupling collapses into mythology.

\section{Conclusion}

Decoupling destroys maps.
Fragments remain.

Relations dissolve.
Reconstruction fails.

Nothing further is claimed.

\bibliographystyle{unsrt}
\bibliography{references}

\end{document}
