\documentclass[11pt]{article}

\usepackage[T1]{fontenc}
\usepackage[utf8]{inputenc}
\usepackage{lmodern}
\usepackage{microtype}
\usepackage{amsmath,amssymb,amsthm}
\usepackage{graphicx}
\usepackage{hyperref}
\usepackage{url}
\usepackage{booktabs}
\usepackage{enumitem}

\usepackage{../../../templates/cop/cop-macros}
\usepackage{../../../templates/cop/cop-diagrams}

\title{S5: Stillness Without Closure}
\author{Pascal Sparidaens}
\date{}

\begin{document}
\maketitle

\begin{abstract}
This demonstration examines stillness as a state
in which processes cease
without resolving underlying structure.
It shows how systems may enter inactive regimes
without achieving completion or finality,
and how collapse occurs when cessation is treated as conclusion.
No endpoint is proposed.
Stillness is presented solely as suspended activity.
\end{abstract}

\section{Purpose}

This paper demonstrates a simple distinction:
stopping does not imply finishing.

Its purpose is not to deny equilibrium,
but to show what becomes visible
when inactivity is separated
from resolution.

\section{What This Demonstration Is Not}

This demonstration does not:
\begin{itemize}
  \item declare an end state,
  \item resolve tensions,
  \item establish final order,
  \item confirm success,
  \item or close processes.
\end{itemize}

Any reading that treats stillness
as completion
misunderstands its status.

\section{What Becomes Visible}

\subsection{Cessation as Structural Condition}

When stillness occurs:
\begin{itemize}
  \item activity diminishes,
  \item transitions halt,
  \item dynamics pause.
\end{itemize}

Underlying constraints
remain in place.

Unresolved relations persist.

\subsection{Why Closure Is Not Implied}

Closure is a narrative act.
Stillness is a physical condition.

Nothing in reduced activity
requires that tensions,
potentials,
or incompatibilities
be settled.

Silence is not synthesis.

\section{Demonstration: Inactive Regimes}

Consider a system
that enters prolonged inactivity.

\begin{itemize}
  \item Change becomes negligible.
  \item Interaction weakens.
\end{itemize}

No internal contradiction
is thereby resolved.

When inactivity
is read as achievement,
unfinished structure is concealed.

\section{How Collapse Occurs}

Collapse occurs when:
\begin{itemize}
  \item stillness is equated with success,
  \item inactivity is moralized,
  \item stagnation is declared final,
  \item unresolved tension is ignored.
\end{itemize}

Under these conditions,
pause becomes doctrine.

Suspension replaces inquiry.

\section{Admissible Reading}

This demonstration permits the following reading:

\begin{quote}
Systems may enter states of minimal activity
without resolving internal structure
or achieving final closure.
\end{quote}

This reading preserves suspension
without finality.

\section{Stop Condition}

This demonstration stops at the point where:
\begin{itemize}
  \item inactivity is treated as resolution,
  \item equilibrium is declared ultimate,
  \item silence is taken as proof,
  \item pause becomes justification.
\end{itemize}

Beyond this point,
stillness collapses into conclusion.

\section{Conclusion}

Stillness suspends motion.
It does not complete it.

Processes pause.
Structure remains.

Nothing further is claimed.

\bibliographystyle{unsrt}
\bibliography{references}

\end{document}
