\documentclass[11pt]{article}

\usepackage[T1]{fontenc}
\usepackage[utf8]{inputenc}
\usepackage{lmodern}
\usepackage{microtype}
\usepackage{amsmath,amssymb,amsthm}
\usepackage{graphicx}
\usepackage{hyperref}
\usepackage{url}
\usepackage{booktabs}
\usepackage{enumitem}

\usepackage{../../../templates/cop/cop-macros}
\usepackage{../../../templates/cop/cop-diagrams}
\usepackage{graphicx}

\title{Q7: Diagrammatic Constraint Topology}
\author{Pascal Sparidaens}
\date{}

\begin{document}
\maketitle

\begin{abstract}
This document examines the COP diagrams
as topological representations of constraint relations,
not as models, explanations, or metaphors.
It shows how the diagrams function
as spatial encodings of admissible flow and opacity,
and how collapse occurs
when diagrams are treated as descriptive systems.
No visual ontology is proposed.
The diagrams are presented solely as constraint topology.
\end{abstract}

\section{Purpose}

This paper demonstrates a simple distinction:
the diagrams encode relations,
not interpretations.

Its purpose is not to explain the framework visually,
but to show what becomes visible
when structural constraints
are rendered spatially.

\section{What This Demonstration Is Not}

This demonstration does not:
\begin{itemize}
  \item illustrate theory,
  \item represent reality,
  \item model behavior,
  \item depict processes,
  \item or provide explanation.
\end{itemize}

Any reading that treats the diagrams
as representational models
misunderstands their function.

\section{What Becomes Visible}

\subsection{Diagrams as Constraint Surfaces}

When the diagrams are used correctly:
\begin{itemize}
  \item admissible paths are exposed,
  \item branching limits are visible,
  \item opacity regions are marked.
\end{itemize}

The diagrams do not describe systems.

They display
where movement remains admissible.

\subsection{Why Representation Is Not Intended}

Representation is a semantic act.
Topology is a structural act.

Nothing in admissibility
requires pictorial accuracy.

Spatial layout functions
as constraint mapping,
not depiction.

\section{Demonstration: Diagram Series}

\subsection{Forward-Only Topology}

\begin{figure}[h]
\centering
\includegraphics[width=0.9\linewidth]{figures/fig1-forward-only.jpg.jpg}
\caption{Forward-only constraint topology}
\end{figure}

This diagram encodes
forward admissibility
without reverse legibility.

Branching and routing
remain viable.

Return paths
are structurally absent.

\subsection{Forward--Reverse Topology}

\begin{figure}[h]
\centering
\includegraphics[width=0.9\linewidth]{figures/fig2-forward-reverse.jpg}
\caption{Forward--reverse constraint topology}
\end{figure}

This diagram encodes
partial reverse accessibility
under constraint.

Forward continuation
remains dominant.

Reverse reconstruction
is bounded.

\subsection{Return-Exposed Topology}

\begin{figure}[h]
\centering
\includegraphics[width=0.9\linewidth]{figures/fig3-return-exposed.jpg}
\caption{Return-exposed constraint topology}
\end{figure}

This diagram encodes
explicit return boundaries.

Re-entry is visible.

Restoration
is structurally blocked.

\section{How Collapse Occurs}

Collapse occurs when:
\begin{itemize}
  \item diagrams are treated as models,
  \item spatial relations are interpreted literally,
  \item topology is reified,
  \item drawings are narrativized.
\end{itemize}

Under these conditions,
constraint becomes metaphor.

Structure becomes illustration.

\section{Admissible Reading}

This demonstration permits the following reading:

\begin{quote}
The COP diagrams function
as topological encodings of admissible constraint relations,
not as representational models or explanations.
\end{quote}

This reading preserves structure
without visual authority.

\section{Stop Condition}

This demonstration stops at the point where:
\begin{itemize}
  \item diagrams are used to justify claims,
  \item spatial layout is equated with causation,
  \item drawings become evidence,
  \item topology becomes doctrine.
\end{itemize}

Beyond this point,
diagrams collapse into symbolism.

\section{Conclusion}

The diagrams encode limits.
They do not depict systems.

Topology constrains movement.
Meaning remains external.

Nothing further is claimed.

\bibliographystyle{unsrt}
\bibliography{references}

\end{document}
