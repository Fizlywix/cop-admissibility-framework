\documentclass[11pt]{article}

\usepackage[T1]{fontenc}
\usepackage[utf8]{inputenc}
\usepackage{lmodern}
\usepackage{microtype}
\usepackage{amsmath,amssymb,amsthm}
\usepackage{graphicx}
\usepackage{hyperref}
\usepackage{url}
\usepackage{booktabs}
\usepackage{enumitem}

\usepackage{../../../templates/cop/cop-macros}
\usepackage{../../../templates/cop/cop-diagrams}

\title{Q2: The ICOP List as Legibility Index}
\author{Pascal Sparidaens}
\date{}

\begin{document}
\maketitle

\begin{abstract}
This document examines the ICOP list
as a structural index of legibility limits,
not as a theory of ignorance or hidden structure.
It shows how the list functions
as a boundary map for admissible reverse reading,
and how collapse occurs when the list
is treated as a source of explanation.
No negative ontology is proposed.
The ICOP list is presented solely as a legibility register.
\end{abstract}

\section{Purpose}

This paper demonstrates a simple distinction:
the ICOP list records
where inference stops.

Its purpose is not to discourage investigation,
but to show what becomes visible
when limits of reconstruction
are made explicit.

\section{What This Demonstration Is Not}

This demonstration does not:
\begin{itemize}
  \item deny the value of evidence,
  \item promote skepticism,
  \item assert unknowability,
  \item privilege mystery,
  \item or prohibit inquiry.
\end{itemize}

Any reading that treats the ICOP list
as a philosophy of ignorance
misunderstands its role.

\section{What Becomes Visible}

\subsection{The ICOP List as Boundary Register}

When the ICOP list is used correctly:
\begin{itemize}
  \item illegible reconstructions are excluded,
  \item speculative histories are blocked,
  \item reverse overreach is constrained.
\end{itemize}

The list does not specify
what cannot be known.

It specifies
what cannot be supported.

\subsection{Why Explanation Is Not Provided}

Explanation is a narrative act.
Legibility is a structural condition.

Nothing in reverse admissibility
requires positive account.

The ICOP list functions
by termination,
not by description.

\section{Demonstration: Reverse Screening}

Consider a system
with partial residue.

\begin{itemize}
  \item Some inferences remain bounded.
  \item Others are excluded.
\end{itemize}

No hidden story is recovered.

Only legibility
is evaluated.

When the list
is used as an explanatory device,
constraint becomes speculation.

\section{How Collapse Occurs}

Collapse occurs when:
\begin{itemize}
  \item the ICOP list is treated as hidden ontology,
  \item opacity is interpreted as mystery,
  \item limits are moralized,
  \item ignorance becomes doctrine.
\end{itemize}

Under these conditions,
the framework becomes anti-epistemology.

Structure is replaced by resignation.

\section{Admissible Reading}

This demonstration permits the following reading:

\begin{quote}
The ICOP list functions
as an index of legibility limits,
not as a source of explanation or hidden truth.
\end{quote}

This reading preserves inquiry
without overreach.

\section{Stop Condition}

This demonstration stops at the point where:
\begin{itemize}
  \item opacity is romanticized,
  \item limits are treated as revelations,
  \item non-knowledge becomes authority,
  \item termination becomes ideology.
\end{itemize}

Beyond this point,
the framework collapses into mystification.

\section{Conclusion}

The ICOP list records boundaries.
It does not conceal meaning.

Legibility has limits.
Inquiry persists.

Nothing further is claimed.

\bibliographystyle{unsrt}
\bibliography{references}

\end{document}
