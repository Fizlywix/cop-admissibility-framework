\documentclass[11pt]{article}

\usepackage[T1]{fontenc}
\usepackage[utf8]{inputenc}
\usepackage{lmodern}
\usepackage{microtype}
\usepackage{amsmath,amssymb,amsthm}
\usepackage{graphicx}
\usepackage{hyperref}
\usepackage{url}
\usepackage{booktabs}
\usepackage{enumitem}

\usepackage{../../../templates/cop/cop-macros}
\usepackage{../../../templates/cop/cop-diagrams}

\title{Q8: Formula and Index as Admissibility Algebra}
\author{Pascal Sparidaens}
\date{}

\begin{document}
\maketitle

\begin{abstract}
This document examines the COP formula
and its associated index system
as an algebra of admissibility cycles,
not as a descriptive model or universal theory.
It shows how directional, ledger, and role coordinates
encode constraint interactions,
and how collapse occurs
when the formula is treated as ontology.
No metaphysical system is proposed.
The formula is presented solely as constraint algebra.
\end{abstract}

\section{Purpose}

This paper demonstrates a simple distinction:
the formula and index encode relations,
not meanings.

Its purpose is not to explain reality,
but to show what becomes visible
when admissibility is expressed
through coordinated operations.

\section{What This Demonstration Is Not}

This demonstration does not:
\begin{itemize}
  \item describe the universe,
  \item explain causation,
  \item predict outcomes,
  \item unify disciplines,
  \item or authorize applications.
\end{itemize}

Any reading that treats the formula
as a universal theory
misunderstands its role.

\section{What Becomes Visible}

\subsection{The Gate Formula as Cycle Encoding}

The admissibility cycle
may be expressed abstractly as:

\begin{quote}
\[
ALL0 ((POSSIBILITIES+POINTZERO=RETURN) =
(({-}Order \; || \; {+}Chaos) \triangleright C)
\otimes
(({-}Chaos \; || \; {+}Order) \triangleright C)
=
(POSSIBILITY-ZEROPOINT=RETURN)) 0ALL
\]
\end{quote}

This expression encodes
forward seam closure (ALL0),
reverse seam closure (0ALL),
and bidirectional pruning
through admissibility narrowing.

It does not describe processes.

It specifies closure relations.

\subsection{Index Coordinates as Constraint Positions}

Each operation is located
by three indices:

\begin{itemize}
  \item Direction (D): constructive or inferential,
  \item Ledger relation (W): latent, write, propagate, closure,
  \item Role (R): 1--8 functional positions.
\end{itemize}

Together,
these coordinates define
a position in admissibility space.

They do not define entities.

They locate constraints.

\section{Demonstration: The 2×4×8 Structure}

\subsection{Directional Axis}

\begin{itemize}
  \item D=1: Forward construction (narrows futures),
  \item D=2: Reverse inference (narrows pasts).
\end{itemize}

This axis encodes
asymmetry of continuation and reading.

\subsection{Ledger Axis}

\begin{itemize}
  \item W=1: Latent option-space,
  \item W=2: Commit / exclusion,
  \item W=3: Propagation / trace,
  \item W=4: Closure / invariant inference.
\end{itemize}

This axis encodes
degrees of inscription and erasure.

\subsection{Role Axis}

\begin{itemize}
  \item R=1--8: Emergence through Elimination.
\end{itemize}

This axis encodes
functional differentiation
without hierarchy.

\section{Visibility as Derived Function}

Each coordinate
admits a visibility tag:

\begin{itemize}
  \item N: Null,
  \item V: Visible,
  \item B: Breach-visible,
  \item S: Seam.
\end{itemize}

Visibility is determined by:

\begin{quote}
Vis(D,W,R)
\end{quote}

It is not intrinsic.

It is a function of constraint satisfaction.

Most reverse structure
remains suppressed
until closure or failure.

\section{The Minimal Admissibility Loop}

Within the 64-cell space,
a minimal cycle always exists:

\begin{enumerate}
  \item Forward seam (1.4.1),
  \item Visible commits and propagation,
  \item Knife-edge saturation (1.4.5),
  \item Reverse tracing and exclusion,
  \item Reverse seam (2.4.8),
  \item Return to ALL0.
\end{enumerate}

This loop expresses
closure without restoration.

It is a structural cycle,
not a narrative.

\section{How Collapse Occurs}

Collapse occurs when:
\begin{itemize}
  \item the formula is reified,
  \item coordinates are interpreted as entities,
  \item cycles are moralized,
  \item algebra becomes cosmology.
\end{itemize}

Under these conditions,
constraint becomes metaphysics.

Structure becomes doctrine.

\section{Admissible Reading}

This demonstration permits the following reading:

\begin{quote}
The COP formula and index system
function as an algebra of admissibility cycles,
encoding directional, ledger, and role constraints
without describing systems or reality.
\end{quote}

This reading preserves formal power
without ontological authority.

\section{Stop Condition}

This demonstration stops at the point where:
\begin{itemize}
  \item the formula is treated as explanatory,
  \item coordinates become categories of being,
  \item cycles are interpreted as laws,
  \item algebra becomes worldview.
\end{itemize}

Beyond this point,
formal structure collapses into metaphysics.

\section{Conclusion}

The formula encodes cycles.
The index locates constraints.

Coordinates regulate admissibility.
Meaning remains external.

Nothing further is claimed.

\bibliographystyle{unsrt}
\bibliography{references}

\end{document}
