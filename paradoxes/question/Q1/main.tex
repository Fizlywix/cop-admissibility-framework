\documentclass[11pt]{article}

\usepackage[T1]{fontenc}
\usepackage[utf8]{inputenc}
\usepackage{lmodern}
\usepackage{microtype}
\usepackage{amsmath,amssymb,amsthm}
\usepackage{graphicx}
\usepackage{hyperref}
\usepackage{url}
\usepackage{booktabs}
\usepackage{enumitem}

\usepackage{../../../templates/cop/cop-macros}
\usepackage{../../../templates/cop/cop-diagrams}

\title{Q1: The COP List as Constraint Index}
\author{Pascal Sparidaens}
\date{}

\begin{document}
\maketitle

\begin{abstract}
This document examines the COP list
as a structural index of admissibility constraints,
not as a theory, model, or explanatory system.
It shows how the list functions
as a boundary map for viable operation and legibility,
and how collapse occurs when the list
is treated as descriptive or interpretive.
No ontology is proposed.
The COP list is presented solely as a constraint register.
\end{abstract}

\section{Purpose}

This paper demonstrates a simple distinction:
the COP list records limits,
not meanings.

Its purpose is not to explain the framework,
but to show what becomes visible
when the list is treated
as an index of inadmissible moves
rather than as a set of claims.

\section{What This Demonstration Is Not}

This demonstration does not:
\begin{itemize}
  \item describe reality,
  \item explain systems,
  \item predict outcomes,
  \item authorize applications,
  \item or provide interpretation.
\end{itemize}

Any reading that treats the COP list
as a descriptive theory
misunderstands its role.

\section{What Becomes Visible}

\subsection{The COP List as Negative Map}

When the COP list is used correctly:
\begin{itemize}
  \item inadmissible operations are excluded,
  \item illegible inferences are blocked,
  \item extension boundaries are marked.
\end{itemize}

The list does not specify
what systems are.

It specifies
what systems cannot support.

\subsection{Why Description Is Not Provided}

Description is a semantic act.
Constraint is a structural act.

Nothing in admissibility
requires positive characterization.

The COP list functions
by delimitation,
not by representation.

\section{Demonstration: Index Function}

Consider a system
evaluated against the COP list.

\begin{itemize}
  \item Some moves remain viable.
  \item Others are excluded.
\end{itemize}

No explanation is produced.

Only admissibility
is determined.

When the list
is used as a descriptive tool,
constraint becomes narrative.

\section{How Collapse Occurs}

Collapse occurs when:
\begin{itemize}
  \item the COP list is treated as ontology,
  \item entries are interpreted as laws,
  \item constraints are moralized,
  \item limits are converted into doctrine.
\end{itemize}

Under these conditions,
the framework becomes ideology.

Structure is replaced by belief.

\section{Admissible Reading}

This demonstration permits the following reading:

\begin{quote}
The COP list functions
as an index of admissibility constraints,
not as a description of systems or reality.
\end{quote}

This reading preserves structure
without authority.

\section{Stop Condition}

This demonstration stops at the point where:
\begin{itemize}
  \item the list is treated as explanatory,
  \item entries are reified,
  \item constraint becomes prescription,
  \item limits become identity.
\end{itemize}

Beyond this point,
the framework collapses into doctrine.

\section{Conclusion}

The COP list records boundaries.
It does not describe worlds.

Constraints guide operation.
Meaning remains external.

Nothing further is claimed.

\bibliographystyle{unsrt}
\bibliography{references}

\end{document}
