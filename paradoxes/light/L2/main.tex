\documentclass[11pt]{article}

\usepackage[T1]{fontenc}
\usepackage[utf8]{inputenc}
\usepackage{lmodern}
\usepackage{microtype}
\usepackage{amsmath,amssymb,amsthm}
\usepackage{graphicx}
\usepackage{hyperref}
\usepackage{url}
\usepackage{booktabs}
\usepackage{enumitem}

\usepackage{../../../templates/cop/cop-macros}
\usepackage{../../../templates/cop/cop-diagrams}

\title{L2: Multiplicity Without Totality}
\author{Pascal Sparidaens}
\date{}

\begin{document}
\maketitle

\begin{abstract}
This demonstration examines multiplicity as a mode of continuation
based on local branching under constraint
without escalation into total inclusion.
It shows how systems generate multiple viable paths
without requiring global enumeration,
and how collapse occurs when multiplicity is forced into completeness.
No total set is proposed.
Multiplicity is presented solely as distributed continuation.
\end{abstract}

\section{Purpose}

This paper demonstrates a simple distinction:
systems may branch into multiple continuations
without producing a total set.

Its purpose is not to decide which paths are real,
but to show what becomes visible
when differentiation remains local
rather than global.

\section{What This Demonstration Is Not}

This demonstration does not:
\begin{itemize}
  \item assert that all possibilities exist,
  \item deny the reality of alternatives,
  \item privilege any branch as final,
  \item construct exhaustive catalogs,
  \item or resolve ontological debates.
\end{itemize}

Any reading that treats multiplicity
as implying completeness
misunderstands its structure.

\section{What Becomes Visible}

\subsection{Multiplicity as Local Differentiation}

When multiplicity is present:
\begin{itemize}
  \item branches form under constraint,
  \item each path remains locally coherent,
  \item no branch contains the whole.
\end{itemize}

Differentiation increases continuation
without producing total inclusion.

\subsection{Why Totality Does Not Follow}

Totality is a global claim.
Branching is a local operation.

Nothing in the formation of alternatives
requires that all alternatives
be realized, preserved, or enumerated.

Multiplicity remains admissible
as long as it is not globalized.

\section{Demonstration: Selective Branching}

Consider a system
that differentiates across multiple layers.

\begin{itemize}
  \item Each layer routes alternatives selectively.
  \item No layer receives all branches.
\end{itemize}

Paths persist only where constraints permit.

When branching is treated
as requiring complete tracking,
differentiation collapses into accounting.

\section{How Collapse Occurs}

Collapse occurs when:
\begin{itemize}
  \item multiplicity is equated with completeness,
  \item branches are treated as a total set,
  \item possibility is mistaken for inclusion,
  \item differentiation is forced into inventory.
\end{itemize}

Under these conditions,
continuation is replaced by enumeration.

Multiplicity becomes unmanageable.

\section{Admissible Reading}

This demonstration permits the following reading:

\begin{quote}
Systems may generate multiple viable continuations
without requiring that all continuations
form a complete or accessible set.
\end{quote}

This reading preserves differentiation
without totalization.

\section{Stop Condition}

This demonstration stops at the point where:
\begin{itemize}
  \item branching is demanded to be exhaustive,
  \item alternatives must be fully listed,
  \item paths are treated as collectively complete,
  \item accounting replaces routing.
\end{itemize}

Beyond this point,
continuation collapses into catalog.

\section{Conclusion}

Multiplicity enables continuation.
Totality obstructs it.

Branches persist locally.
No set is complete.

Nothing further is claimed.

\bibliographystyle{unsrt}
\bibliography{references}

\end{document}
