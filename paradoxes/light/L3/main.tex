\documentclass[11pt]{article}

\usepackage[T1]{fontenc}
\usepackage[utf8]{inputenc}
\usepackage{lmodern}
\usepackage{microtype}
\usepackage{amsmath,amssymb,amsthm}
\usepackage{graphicx}
\usepackage{hyperref}
\usepackage{url}
\usepackage{booktabs}
\usepackage{enumitem}

\usepackage{../../../templates/cop/cop-macros}
\usepackage{../../../templates/cop/cop-diagrams}

\title{L3: Routed Tension, Not Resolved Conflict}
\author{Pascal Sparidaens}
\date{}

\begin{document}
\maketitle

\begin{abstract}
This demonstration examines tension as a structural condition
that supports stability through routing
rather than through elimination or resolution.
It shows how opposing forces may remain co-present
when constrained pathways are maintained,
and how collapse occurs when one pole is permitted to dominate.
No synthesis is proposed.
Tension is presented solely as a load-bearing relation.
\end{abstract}

\section{Purpose}

This paper demonstrates a simple distinction:
systems may remain stable
by routing opposing forces
without resolving them.

Its purpose is not to reconcile conflicts,
but to show what becomes visible
when opposition is structured
rather than suppressed.

\section{What This Demonstration Is Not}

This demonstration does not:
\begin{itemize}
  \item promote permanent disagreement,
  \item deny cooperation,
  \item valorize polarization,
  \item require equilibrium,
  \item or privilege compromise.
\end{itemize}

Any reading that treats tension
as a problem to be solved
misunderstands its function.

\section{What Becomes Visible}

\subsection{Tension as Structural Load}

When tension is routed:
\begin{itemize}
  \item opposing forces remain active,
  \item neither pole dominates globally,
  \item interaction is constrained by pathways.
\end{itemize}

Stability arises
from maintained opposition,
not from uniformity.

\subsection{Why Resolution Is Not Required}

Resolution is a global outcome.
Tension is a local relation.

Nothing in sustained operation
requires that opposing forces
be unified, dissolved, or eliminated.

Systems persist
by distributing opposition,
not by erasing it.

\section{Demonstration: Constrained Opposition}

Consider a system
containing opposing tendencies.

\begin{itemize}
  \item Each tendency is locally effective.
  \item Each is limited by complementary constraint.
\end{itemize}

Interaction occurs
through structured channels.

When one tendency is allowed
to bypass routing,
opposition converts into dominance.

\section{How Collapse Occurs}

Collapse occurs when:
\begin{itemize}
  \item one pole suppresses the other,
  \item routing channels are removed,
  \item opposition is treated as error,
  \item dominance is equated with stability.
\end{itemize}

Under these conditions,
balance is replaced by control.

Tension becomes destructive
once it is no longer constrained.

\section{Admissible Reading}

This demonstration permits the following reading:

\begin{quote}
Systems remain stable
when opposing forces are routed
through constrained pathways,
without requiring resolution or synthesis.
\end{quote}

This reading preserves opposition
without escalation.

\section{Stop Condition}

This demonstration stops at the point where:
\begin{itemize}
  \item conflict must be eliminated,
  \item dominance is treated as order,
  \item routing is replaced by suppression,
  \item uniformity is equated with coherence.
\end{itemize}

Beyond this point,
tension collapses into hierarchy.

\section{Conclusion}

Tension sustains structure.
Resolution concentrates power.

Systems persist through routing.
Opposition remains active.

Nothing further is claimed.

\bibliographystyle{unsrt}
\bibliography{references}

\end{document}
