\documentclass[11pt]{article}

\usepackage[T1]{fontenc}
\usepackage[utf8]{inputenc}
\usepackage{lmodern}
\usepackage{microtype}
\usepackage{amsmath,amssymb,amsthm}
\usepackage{graphicx}
\usepackage{hyperref}
\usepackage{url}
\usepackage{booktabs}
\usepackage{enumitem}

\usepackage{../../../templates/cop/cop-macros}
\usepackage{../../../templates/cop/cop-diagrams}

\title{L5: Experience Exists Inside a Stability Band}
\author{Pascal Sparidaens}
\date{}

\begin{document}
\maketitle

\begin{abstract}
This demonstration examines experience as an emergent condition
that appears only within a bounded range of stability.
It shows how systems support agency and complexity
only inside a constrained middle zone,
and how collapse occurs when conditions move beyond this band.
No universal openness is proposed.
Experience is presented solely as a stabilized regime.
\end{abstract}

\section{Purpose}

This paper demonstrates a simple distinction:
experience remains viable
only within a limited stability range.

Its purpose is not to define optimal conditions,
but to show what becomes visible
when complexity is treated
as a bounded phenomenon.

\section{What This Demonstration Is Not}

This demonstration does not:
\begin{itemize}
  \item guarantee freedom everywhere,
  \item deny constraint,
  \item privilege equilibrium,
  \item promise permanence,
  \item or assert universal safety.
\end{itemize}

Any reading that treats experience
as globally available
misunderstands its dependence.

\section{What Becomes Visible}

\subsection{Stability Bands and Viability}

When a stability band exists:
\begin{itemize}
  \item processes remain regulated,
  \item variation remains bounded,
  \item interaction remains sustainable.
\end{itemize}

Within this range,
complex structures persist.

Outside it,
structure degrades.

\subsection{Why Extremes Are Not Viable}

Extremes are global conditions.
Experience is locally sustained.

Nothing in agency or awareness
requires unlimited openness
or total rigidity.

Both extremes
eliminate viable interaction.

\section{Demonstration: Middle-Zone Dynamics}

Consider a system
operating under variable pressure.

\begin{itemize}
  \item Below the band, activity dissipates.
  \item Above the band, instability dominates.
\end{itemize}

Only intermediate regimes
support persistent complexity.

When regulation weakens,
chaos increases.

When regulation hardens,
adaptation disappears.

\section{How Collapse Occurs}

Collapse occurs when:
\begin{itemize}
  \item constraint is removed entirely,
  \item rigidity becomes absolute,
  \item variation escapes regulation,
  \item control suppresses adaptation.
\end{itemize}

Under these conditions,
experience becomes unsustainable.

Viability narrows to zero.

\section{Admissible Reading}

This demonstration permits the following reading:

\begin{quote}
Experience and complexity persist
only within a bounded stability range,
neither fully open nor fully rigid.
\end{quote}

This reading preserves emergence
without idealization.

\section{Stop Condition}

This demonstration stops at the point where:
\begin{itemize}
  \item openness is treated as universally safe,
  \item rigidity is equated with order,
  \item constraint is denied legitimacy,
  \item extremes are normalized.
\end{itemize}

Beyond this point,
viability collapses into extremity.

\section{Conclusion}

Experience is conditional.
Stability is bounded.

Complexity inhabits the middle.
Extremes erase it.

Nothing further is claimed.

\bibliographystyle{unsrt}
\bibliography{references}

\end{document}
