\documentclass[11pt]{article}

\usepackage[T1]{fontenc}
\usepackage[utf8]{inputenc}
\usepackage{lmodern}
\usepackage{microtype}
\usepackage{amsmath,amssymb,amsthm}
\usepackage{graphicx}
\usepackage{hyperref}
\usepackage{url}
\usepackage{booktabs}
\usepackage{enumitem}

\usepackage{../../../templates/cop/cop-macros}
\usepackage{../../../templates/cop/cop-diagrams}

\title{L1: Redundancy Without Optimality}
\author{Pascal Sparidaens}
\date{}

\begin{document}
\maketitle

\begin{abstract}
This demonstration examines redundancy as a mode of continuation
based on parallel and overlapping channels
rather than on optimal configuration.
It shows how systems persist through distributed admissibility,
and how collapse occurs when a single path is optimized to dominance.
No ideal arrangement is proposed.
Redundancy is presented solely as a condition of viable routing.
\end{abstract}

\section{Purpose}

This paper demonstrates a simple distinction:
systems may persist through multiple constrained channels
without requiring an optimal path.

Its purpose is not to defend inefficiency,
but to show what becomes visible
when continuation is distributed
rather than concentrated.

\section{What This Demonstration Is Not}

This demonstration does not:
\begin{itemize}
  \item oppose refinement,
  \item reject improvement,
  \item promote duplication for its own sake,
  \item deny coordination,
  \item or privilege disorder.
\end{itemize}

Any reading that treats redundancy
as mere surplus
misunderstands its structure.

\section{What Becomes Visible}

\subsection{Redundancy as Parallel Routing}

When redundancy is present:
\begin{itemize}
  \item multiple channels carry partial load,
  \item no channel is sufficient on its own,
  \item failure remains locally bounded.
\end{itemize}

Continuation is preserved
through overlapping admissibility,
not through a dominant pathway.

\subsection{Why Optimality Is Not Required}

Optimality is a global attribution.
Continuation is a local condition.

Nothing in sustained operation
requires that one configuration
outperform all others.

Viability depends on routing,
not perfection.

\section{Demonstration: Distributed Channels}

Consider a system
with several constrained pathways.

\begin{itemize}
  \item Each channel transmits selectively.
  \item No channel carries total continuity.
\end{itemize}

When one channel degrades,
others remain admissible.

When channels are reduced
to a single optimized path,
propagation becomes unavoidable.

\section{How Collapse Occurs}

Collapse occurs when:
\begin{itemize}
  \item one pathway is optimized to dominance,
  \item filtering is removed for efficiency,
  \item routing is centralized,
  \item recovery depends on ideal configuration.
\end{itemize}

Under these conditions,
local failure becomes global disruption.

Optimization converts routing
into fragility.

\section{Admissible Reading}

This demonstration permits the following reading:

\begin{quote}
Systems remain viable
when continuation is distributed
across constrained channels,
without privileging any optimal path.
\end{quote}

This reading preserves routing
without idealization.

\section{Stop Condition}

This demonstration stops at the point where:
\begin{itemize}
  \item redundancy is eliminated for efficiency,
  \item one channel becomes mandatory,
  \item recovery is tied to perfection,
  \item distribution is replaced by control.
\end{itemize}

Beyond this point,
routing collapses into dependency.

\section{Conclusion}

Redundancy sustains continuation.
Optimality concentrates risk.

Systems persist through overlap.
No path is sufficient.

Nothing further is claimed.

\bibliographystyle{unsrt}
\bibliography{references}

\end{document}
