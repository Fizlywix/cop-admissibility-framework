\documentclass[11pt]{article}

\usepackage[T1]{fontenc}
\usepackage[utf8]{inputenc}
\usepackage{lmodern}
\usepackage{microtype}
\usepackage{amsmath,amssymb,amsthm}
\usepackage{graphicx}
\usepackage{hyperref}
\usepackage{url}
\usepackage{booktabs}
\usepackage{enumitem}

\usepackage{../../../templates/cop/cop-macros}
\usepackage{../../../templates/cop/cop-diagrams}

\title{L1: Redundancy Without Optimality}
\author{Pascal Sparidaens}
\date{}

\begin{document}
\maketitle

\begin{abstract}
This demonstration examines redundancy as a structural condition
that supports continuation without reference to optimization.
It shows how systems persist through parallel and overlapping capacity,
and how collapse occurs when efficiency is treated as a governing principle.
No ideal configuration is proposed.
Redundancy is presented solely as an admissible mode of stability.
\end{abstract}

\section{Purpose}

This paper demonstrates a simple distinction:
redundancy enables persistence
without requiring optimal arrangement.

Its purpose is not to defend inefficiency,
but to show what becomes visible
when surplus capacity is treated
as a condition of viability
rather than as waste.

\section{What This Demonstration Is Not}

This demonstration does not:
\begin{itemize}
  \item argue against technical improvement,
  \item oppose refinement or skill,
  \item celebrate disorder,
  \item promote maximal duplication,
  \item or reject coordination.
\end{itemize}

Any attempt to read this paper
as an attack on optimization
misunderstands its scope.

\section{What Becomes Visible}

\subsection{Redundancy as Structural Insurance}

When redundancy is present:
\begin{itemize}
  \item failure remains local,
  \item interruption does not propagate,
  \item function persists under disturbance.
\end{itemize}

Multiple pathways preserve continuity
without centralized control.

\subsection{Why Optimality Is Not Required}

Optimality is a global evaluation.
Stability is a local condition.

No functioning system requires
that every component
operate at maximal efficiency.

Persistence depends on overlap,
not perfection.

\section{Demonstration: Parallel Capacity}

Consider a system
with multiple equivalent channels.

\begin{itemize}
  \item If one channel degrades,
        others continue operation.
  \item No single pathway
        carries total responsibility.
\end{itemize}

Performance remains acceptable
despite partial failure.

When channels are reduced
to a single optimal path,
the system becomes fragile.

\section{How Collapse Occurs}

Collapse occurs when:
\begin{itemize}
  \item redundancy is treated as error,
  \item surplus is eliminated systematically,
  \item resilience is exchanged for efficiency.
\end{itemize}

Under these conditions,
minor disturbances
produce disproportionate disruption.

Optimization converts robustness
into vulnerability.

\section{Admissible Reading}

This demonstration permits the following reading:

\begin{quote}
Systems remain viable
when overlapping capacity is preserved,
even if no configuration is maximally efficient.
\end{quote}

This reading prioritizes continuation
over ideal performance.

\section{Stop Condition}

This demonstration stops at the point where:
\begin{itemize}
  \item redundancy is framed as failure,
  \item stability is subordinated to metrics,
  \item control replaces distributed support.
\end{itemize}

Beyond this point,
evaluation displaces observation.

\section{Conclusion}

Redundancy sustains systems.
Optimality narrows them.

Persistence depends on surplus.
Efficiency is optional.

Nothing further is claimed.

\bibliographystyle{unsrt}
\bibliography{references}

\end{document}
