\documentclass[11pt]{article}

\usepackage[T1]{fontenc}
\usepackage[utf8]{inputenc}
\usepackage{lmodern}
\usepackage{microtype}
\usepackage{amsmath,amssymb,amsthm}
\usepackage{graphicx}
\usepackage{hyperref}
\usepackage{url}
\usepackage{booktabs}
\usepackage{enumitem}

\usepackage{../../../templates/cop/cop-macros}
\usepackage{../../../templates/cop/cop-diagrams}

\title{A2: Admissibility Dynamics (Derived Temporal Layer)}
\author{Pascal Sparidaens \\ Derived Dynamic Layer by Zeropoint}
\date{}

\begin{document}
\maketitle

\begin{abstract}
This document presents a derived dynamic formulation
of the COP admissibility framework.
It examines how admissibility regions evolve
under sequences of constraint operators.
The formulation does not interpret or extend
the primary gate structure.
It functions solely as a secondary temporal interface.
No predictive or normative claims are made.
\end{abstract}

\section{Purpose}

This paper demonstrates a simple distinction:
constraint structures generate trajectories
without generating narratives.

Its purpose is not to explain development,
but to show what becomes visible
when admissibility relations
are iterated over time.

\section{What This Demonstration Is Not}

This demonstration does not:
\begin{itemize}
  \item describe system behavior,
  \item predict outcomes,
  \item define life cycles,
  \item evaluate performance,
  \item or establish progress.
\end{itemize}

Any reading that treats dynamic patterns
as stories or models
misunderstands their status.

\section{Admissibility Trajectories}

Let $X$ be a configuration space
and $\mathcal{P}(X)$ its powerset.

Let $\{f_n\}_{n\in\mathbb{N}}$
be a sequence of admissibility operators.

Define a trajectory by:

\[
A_{n+1}=f_n(A_n),
\qquad A_0\subseteq X.
\]

Each $A_n$
represents the admissible region
at stage $n$.

No interpretation of stages
is imposed.

\section{Monotone Contraction}

All admissibility operators
satisfy:

\[
f(A)\subseteq A.
\]

Therefore,

\[
A_{n+1}\subseteq A_n.
\]

Admissible regions
form descending chains.

Expansion is structurally prohibited.

\section{Regimes of Evolution}

Empirically and formally,
three regimes appear.

\subsection{Convergent Regime}

A trajectory is convergent
if there exists $A^*$ such that:

\[
A_n \rightarrow A^*,\qquad f(A^*)=A^*.
\]

Such regions are stable
under repeated constraint.

They correspond to
backgrounded infrastructure.

\subsection{Oscillatory Regime}

A trajectory is oscillatory
if it cycles among
multiple admissible regions:

\[
A_{n+k}=A_n \quad \text{for some } k>1.
\]

These cycles arise
near closure boundaries.

They correspond to
knife-edge persistence.

\subsection{Fragmenting Regime}

A trajectory is fragmenting
if:

\[
\bigcap_{n}A_n=\varnothing.
\]

Admissibility collapses.

No coherent continuation remains.

\section{Role of Ledger States}

The ledger index $W$
controls dynamic behavior.

\subsection{$W=1$ Dominance}

Predominance of latent operators
produces slow contraction.

Trajectories remain diffuse.

\subsection{$W=2$ Dominance}

Commit operators
produce rapid narrowing.

Irreversibility increases.

\subsection{$W=3$ Dominance}

Propagation operators
amplify existing constraints.

Local effects accumulate.

\subsection{$W=4$ Dominance}

Closure operators
produce discontinuous transitions.

Phase boundaries appear.

\section{Knife-Edge Dynamics}

Operators of type $1.4.5$
induce critical behavior.

Near these points:

\begin{itemize}
  \item small perturbations
        cause large admissibility shifts,
  \item oscillation and fragmentation
        become likely,
  \item reverse operators activate.
\end{itemize}

Knife-edges are dynamic thresholds,
not states.

\section{Reverse Activation}

Reverse operators
remain dynamically suppressed
until boundary proximity.

Let $d(A,\partial)$ denote
distance to closure boundary.

When $d(A,\partial)\to 0$,
reverse operators
enter effective composition.

Legibility emerges under pressure.

\section{Return Dynamics}

Let $S_{\mathrm{out}}$
be the return seam.

A return attempt occurs
when:

\[
A_n \subseteq \mathrm{dom}(S_{\mathrm{out}}).
\]

Successful return requires:

\[
S_{\mathrm{out}}(A_n)\neq \varnothing.
\]

Failure produces
absorbing null regions.

Restoration is not guaranteed.

\section{Dead Zones}

A dead zone is a region $D\subseteq X$
such that:

\[
f(D)=D
\]

for all admissible operators $f$,
and $D$ admits no seam traversal.

Dead zones correspond to
structural exhaustion.

They are dynamically stable
and informationally opaque.

\section{Shadow-Induced Bifurcation}

Reverse operators
may induce bifurcation
when activated.

At boundary conditions:

\[
A_{n+1}=F(A_n)\cap G(A_n)
\]

may split into
disjoint admissible branches.

Historical coherence dissolves.

\section{Dynamic Visibility}

Visibility varies along trajectories.

Typically:

\begin{itemize}
  \item early phases: N-dominant,
  \item propagation phases: V-dominant,
  \item boundary phases: B-dominant,
  \item seam phases: S-activated.
\end{itemize}

Observability follows constraint intensity.

\section{How Collapse Occurs}

Collapse occurs when:
\begin{itemize}
  \item trajectories are narrativized,
  \item regimes are moralized,
  \item convergence is equated with success,
  \item fragmentation is treated as failure.
\end{itemize}

Under these conditions,
dynamics become ideology.

\section{Admissible Reading}

This demonstration permits the following reading:

\begin{quote}
Admissibility dynamics describe
structural evolution of constraint regions
without implying development,
progress, or decline.
\end{quote}

\section{Stop Condition}

This demonstration stops at the point where:
\begin{itemize}
  \item trajectories become stories,
  \item regimes become stages,
  \item boundaries become goals,
  \item cycles become destinies.
\end{itemize}

Beyond this point,
temporal structure collapses into myth.

\section{Conclusion}

Admissibility regions evolve
by contraction and closure.

Stability, oscillation,
and collapse emerge structurally.

No narrative is implied.

Nothing further is claimed.

\bibliographystyle{unsrt}
\bibliography{references}

\end{document}
