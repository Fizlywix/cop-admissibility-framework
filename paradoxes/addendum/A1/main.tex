\documentclass[11pt]{article}

\usepackage[T1]{fontenc}
\usepackage[utf8]{inputenc}
\usepackage{lmodern}
\usepackage{microtype}
\usepackage{amsmath,amssymb,amsthm}
\usepackage{graphicx}
\usepackage{hyperref}
\usepackage{url}
\usepackage{booktabs}
\usepackage{enumitem}

\usepackage{../../../templates/cop/cop-macros}
\usepackage{../../../templates/cop/cop-diagrams}

\title{A1: Admissibility Algebra (Derived Formal Layer)}
\author{Pascal Sparidaens \\ Derived Formal Layer by Zeropoint}
\date{}

\begin{document}
\maketitle

\begin{abstract}
This document presents a derived algebraic formulation
of the COP admissibility framework.
It translates directional, ledger, and role relations
into constraint operators over possibility spaces.
The formulation does not replace or interpret
the primary gate formula.
It functions solely as a secondary formal interface.
No ontological status is claimed.
\end{abstract}

\section{Purpose}

This paper demonstrates a simple distinction:
algebraic structure may be derived
from admissibility relations
without replacing their primary expression.

Its purpose is not to formalize the framework completely,
but to show what becomes visible
when constraint relations
are expressed in operator form.

\section{What This Demonstration Is Not}

This demonstration does not:
\begin{itemize}
  \item reinterpret the gate formula,
  \item replace symbolic structure,
  \item impose axioms,
  \item define meaning,
  \item or establish foundations.
\end{itemize}

Any reading that treats this algebra
as primary
misunderstands its status.

\section{Admissibility Space}

Let $X$ denote a space of configurations.

Elements of $X$ may represent
partial histories,
possible trajectories,
or admissible states.

No internal structure of $X$
is assumed.

Let $\mathcal{P}(X)$ denote
the powerset of $X$,
ordered by inclusion.

Elements of $\mathcal{P}(X)$
represent admissible possibility regions.

\section{Constraint Operators}

An admissibility operator
is a function

\[
f:\mathcal{P}(X)\rightarrow\mathcal{P}(X)
\]

satisfying:

\begin{itemize}
  \item $f(A)\subseteq A$ (contracting),
  \item $A\subseteq B \Rightarrow f(A)\subseteq f(B)$ (monotone).
\end{itemize}

Such operators
narrow admissible regions.

They do not generate content.

\section{Directional Families}

Two operator families are distinguished.

\subsection{Forward Operators}

Forward operators
constrain future admissibility:

\[
F_{w,r}:\mathcal{P}(X)\rightarrow\mathcal{P}(X)
\]

with $w\in\{1,2,3,4\}$ and $r\in\{1,\dots,8\}$.

They correspond to constructive narrowing.

\subsection{Reverse Operators}

Reverse operators
constrain past legibility:

\[
G_{w,r}:\mathcal{P}(X)\rightarrow\mathcal{P}(X)
\]

They correspond to inferential narrowing.

\section{Index Structure}

Each admissibility position
is indexed by

\[
(D,W,R)\in\{1,2\}\times\{1,2,3,4\}\times\{1,\dots,8\}.
\]

Define the mapping

\[
\Phi(D,W,R)=
\begin{cases}
F_{W,R} & D=1,\\
G_{W,R} & D=2.
\end{cases}
\]

This mapping assigns
each coordinate
to a narrowing operator.

Coordinates do not denote entities.

They locate constraints.

\section{Composition Operator}

Let $\otimes$ denote
bidirectional admissibility composition.

Two equivalent forms are admissible.

\subsection{Sequential Form}

\[
(f\otimes g)(A)=g(f(A)).
\]

This represents
forward narrowing
followed by reverse narrowing.

\subsection{Intersection Form}

\[
(f\otimes g)(A)=f(A)\cap g(A).
\]

This represents
simultaneous constraint.

No global preference
between these forms is imposed.

\section{Seam Operators}

Two distinguished operators
function as admissibility gates.

\subsection{Entry Seam}

\[
S_{\mathrm{in}}:\mathcal{P}(X)\rightarrow\mathcal{P}(X)
\]

corresponds to ALL0.

\subsection{Return Seam}

\[
S_{\mathrm{out}}:\mathcal{P}(X)\rightarrow\mathcal{P}(X)
\]

corresponds to 0ALL.

Both satisfy:

\begin{itemize}
  \item $S(A)\subseteq A$,
  \item $S(S(A))=S(A)$.
\end{itemize}

They function as interior gate operators.

They regulate admissible cycling.

\section{Ledger Stratification}

The ledger index $W$
induces a stratification
of operator behavior.

\subsection{$W=1$ (Latent)}

Weak bounding operators.
They restrict ambiguity
without inscription.

\subsection{$W=2$ (Commit / Exclusion)}

Irreversible inscription operators.

Forward and reverse operators
satisfy a Galois-type relation:

\[
F_{2,r}(A)\subseteq B
\iff
A\subseteq G_{2,r}(B).
\]

\subsection{$W=3$ (Propagation / Trace)}

Derived operators
generated by composition
of $W=2$ and $W=4$.

\subsection{$W=4$ (Closure / Invariant)}

Idempotent projectors
encoding lock-in
and boundary inference.

\section{Visibility Function}

Let $O:\mathcal{P}(X)\rightarrow\mathcal{P}(Y)$
be an observation map.

Define visibility as

\[
Vis(f)=
\begin{cases}
N & O(f(A))=O(A),\\
V & O(f(A))\neq O(A),\\
B & O(f(A))\neq O(A)\ \text{only at boundaries},\\
S & f \text{ alters admissibility domain}.
\end{cases}
\]

Visibility is derived.

It is not intrinsic.

\section{Minimal Cycle}

A minimal admissibility cycle
has the form

\[
\mathcal{C}
=
S_{\mathrm{in}}
\circ
P
\circ
K
\circ
R
\circ
S_{\mathrm{out}},
\]

where:

\begin{itemize}
  \item $P$ is a composite of $W=2$ and $W=3$ operators,
  \item $K$ is a $W=4$ closure operator,
  \item $R$ is a composite of reverse operators.
\end{itemize}

This cycle encodes
closure without restoration.

\section{Stability Regions}

A region $A\subseteq X$
is stable under a family $\mathcal{F}$
if

\[
A_{n+1}=f_n(A_n),\quad f_n\in\mathcal{F},
\]

converges to $A^*$ with

\[
f(A^*)=A^*.
\]

Stable regions correspond
to repeated backgrounded closure.

They are not catalogued.

They are emergent.

\section{How Collapse Occurs}

Collapse occurs when:
\begin{itemize}
  \item operators are reified,
  \item indices become entities,
  \item closure is treated as resolution,
  \item algebra replaces admissibility.
\end{itemize}

Under these conditions,
formal structure becomes ideology.

\section{Admissible Reading}

This demonstration permits the following reading:

\begin{quote}
The admissibility algebra
provides a derived operator language
for expressing constraint relations
without replacing the primary gate formulation.
\end{quote}

\section{Stop Condition}

This demonstration stops at the point where:
\begin{itemize}
  \item algebra is treated as ontology,
  \item operators are absolutized,
  \item derivation replaces primary structure,
  \item formalism becomes authority.
\end{itemize}

Beyond this point,
the algebra collapses into doctrine.

\section{Conclusion}

The admissibility algebra
encodes narrowing relations.

It does not define reality.

It supports structural analysis
without semantic closure.

Nothing further is claimed.

\bibliographystyle{unsrt}
\bibliography{references}

\end{document}
